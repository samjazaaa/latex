\documentclass[11pt, paper=a4, twocolumn]{scrartcl}

\usepackage[ngerman]{babel}
\usepackage[utf8]{inputenc}

\usepackage[T1]{fontenc}
\usepackage{mathpazo}

\usepackage{geometry}

\geometry{a4paper, top=20mm, left=15mm, right=15mm, bottom=20mm,
headsep=5mm, footskip=12mm}


\pagenumbering{gobble}

\title{\vspace{-1.25cm}Zusammenfassung Crypto\vspace{-0.25cm}}
\date{\vspace{-5ex}}

\newcommand*{\Z}{\mathbb{Z}}

\begin{document}
	\maketitle


	\section{Allgemeine Kryptosysteme}
		\begin{itemize}
			\item Krypto Schutzziele
				\begin{itemize}
					\item Vertraulichkeit (Empfänger)
					\item Integrität (Inhalt)
					\item Authentizität (Sender)
				\end{itemize}
			\item Kryptosystem $(P,C,K,E,D)$
				\begin{itemize}
					\item $P$ Klartextraum
					\item $C$ Ciphertextraum
					\item $K$ Schlüsselraum
					\item $E=\{E_k:k\in K\}$ Verschlüsselungsfunktionen
					\item $D$ Entschlüsselungsfunktionen
				\end{itemize}
			\item Symmetrische Systeme
				\begin{itemize}
					\item Nur ein Schlüssel, schneller
					\item Schlüsselaustausch und Geheimhaltung nötig
				\end{itemize}
			\item Asymmetrische Systeme
				\begin{itemize}
					\item  Kein Schlüsselaustausch nötig, höhere Sicherheit
					\item Langsamer, Anzahl der Keys
				\end{itemize}
			\item Angriffsarten
				\begin{itemize}
					\item Ciphertext-Only
					\item Known-Plaintext (Angreifer kennt P-C Paare)
					\item Chosen-Plaintext (Angreifer wählt P)
					\item Chosen-Ciphertext (Angreifer kann Cs entschlüsseln lassen)
				\end{itemize}
		\end{itemize}

	\section{Blockchiffren}
		\begin{itemize}
			\item Symmetrisch und deterministisch
			\item Verschlüsselung von Blöcken gleicher Länge
			\item Stromchiffren
				\begin{itemize}
					\item Meist Blöcke variabler Länge
					\item Kein Warten bis Block voll
				\end{itemize}
			\item Betriebsmodi
				\begin{itemize}
					\item ECB (Electronic Code Book)
						\begin{itemize}
							\item Nachricht in 
								gleich große 
								Blöcke teilen
							\item Evtl. Padding
							\item Gleiche Blöcke gleich verschlüsselt
							\item Keine Rückkopplung / Randomisierung
						\end{itemize}
					\item CBC (Cipher Block Chaining)
						\begin{itemize}
							\item Teilen / Padding
							\item Initialisierungsvektor
							\item Rückkopplung mit Ciphertext des vorherigen Blocks
							\item Pseudorandomisierung
						\end{itemize}
					\item CFB (Cipher Feedback Mode)
						\begin{itemize}
							\item Selbstsynchronisierende Stromchiffre
							\item Initialisierungsvektor
							\item Rückkopplung mit vorherigem Ciphertext in Schieberegister
						\end{itemize}
					\item OFB (Output Feedback Mode)
						\begin{itemize}
							\item Synchrone Stromchiffre
							\item Rückkopplung mit verwendetem Teil des zuvor generierten Schlüsselteils
						\end{itemize}
					\item CTR (Counter Mode)
						\begin{itemize}
							\item Zähler wird zu Initialisierungsvektor addiert
							\item IV also immer NONCE
							\item Schutz vor Replay
						\end{itemize}
				\end{itemize}
		\end{itemize}
	
	\section{Affin lineare Blockchiffren}
		\begin{itemize}
			\item Symmetrisch, deterministisch, konstante Blocklänge
			\item Blocklänge $n$ und Alphabetlänge $m$
			\item Schlüssel $K$ besteht aus Teilen $a$ und $b$
			\item Verschlüsselung: $C_i=(a*P_i+b)\;mod\;m$
			\item Entschlüsselung: $P_i=((C_i-b)*a')\;mod\;m$
			\item Sonderfälle
				\begin{itemize}
					\item $a=1$: additiv (Shift, Caesar)
					\item $b=0$: multiplikativ (Substitution)
				\end{itemize}
			\item Affine Chiffre
				\begin{itemize}
					\item Blocklänge $n=1$
					\item Schlüssel: $K=(a,b)\in \Z_m^2$
					\item $a$ und $m$ teilerfremd ($gcd(a,m)=1$)
					\item $a'$ ist inverses Element zu $a\;mod\;m$ und kann mit erweitertem eukl. Alg. berechnet werden
					\item Es gilt $a*a'=1\;mod\;m$
				\end{itemize}
			\item Angriffe auf affine Chiffre
				\begin{itemize}
					\item Häufigkeitsanalyse
					\item Ciphertext-Only möglich, da kleiner Schlüsselraum
					\item Known-Plaintext: aus zwei Klartext und Ciphertext Elementen kann Schlüssel berechnet werden
				\end{itemize}
			\item Affin lineare Blockchiffre
				\begin{itemize}
					\item Schlüsselraum: $(A,\vec{b})\in \Z_m^{(n,n)}\times \Z_m^n$
					\item Wobei $gcd(det\;A,m)=1$
					\item Enc: $A\vec{x}+\vec{b}\;mod\;m$
					\item Dec: $A'(\vec{x}-\vec{b})$
					\item $A'=(a'*adj\;A)\;mod\;m$\\
						mit $a'$ mult. Inv. von $det\;A\;mod\;m$
				\end{itemize}
			\item Known-Plaintext mit zwei Blockpaaren immer noch möglich
		\end{itemize}
	
	\section{Perfekte Geheimhaltung}
		\begin{itemize}
			\item Mit beliebig viel Ressourcen ist Klartext ohne Schlüssel nicht Aus Ciphertext rekonstruierbar
			\item Schlüsseltext darf keine Infos Wahrscheinlichkeiten) über Klartext verraten
			\item Computational (Beweis, dass bester Alg. langsam), Provable (Reduktion) und Heuristic (keine einfache Lösung bekannt) security
			\item Schlüssel- und Klartext stoch. unabh.\\
				$\forall p\in P, \forall c\in C:\;Pr(p|c)=Pr(p)$
			\item Bzw.: $Pr(p)*Pr(c|p)=Pr(p\cap c)=$\\
				$=Pr(c)*Pr(p|c)$
			\item Jeder Schlüssel genau einmal verwendet
			\item Prefekt geheim $\Leftrightarrow$\\
				$\forall (p_0,p_1)\in P, \forall c\in C:\;Pr(c|p_0)=Pr(c|p_1)$
			\item Wahrscheinlichkeit für bestimmten Schlüssel unabhängig vom Klartext
			\item Satz von Shannon: Perfekt geheim $\Leftrightarrow$ Schlüsselraum gleichverteilt und für jeden p und c genau ein k
			\item Vigenere Chiffre $c_i \equiv p_i+k_i\;mod\;26$
				\begin{itemize}
					\item Cäsar: Periode 1
					\item Beaufort: $c_i\equiv k_i-p_i\;mod\;26$
					\item Vernam (OTP)\\
						Schlüssellänge = Nachrichtlänge
					\item Compound Vigenere
				\end{itemize}
			\item OTP ist perfekt geheim
			\item Mehrfache Schlüsselverwendung angreifbar:\\
				XOR der Ciphert. ergibt XOR der Plaint.
			\item Immun gegen Ciphertext-only, Chosen Plain-/Ciphertext und Known Plaintext
			\item Substitution-Angriff
				\begin{itemize}
					\item Je 2 bit Schlüssel und Cipher, 1 bit Plain
					\item Angreifer kann Ciphertext garantiert flippen
					\item $\Rightarrow$ Permutation der Ciphertexte liefert perfekte Authentizität
				\end{itemize}
		\end{itemize}
	
	\section{DES-Algorithmus}
		\begin{itemize}
			\item Feistel-Chiffren
				\begin{itemize}
					\item Binäre Blockchiffre mit Länge $2t$
					\item Rundenzahl $r\geq 1$ mit entsprechenden Rundenschlüsseln
					\item Aufteilen in $p=(L_0,R_0)$
					\item Enc.: $(L_i,R_i)=(R_{i-1},L_{i-1}\oplus f_{K_i}(R_{i-1}))$
					\item Dec.: $(R_{i-1},L_{i-1})=(L_i,R_i\oplus f_{K_i}(L_i))$
				\end{itemize}
			\item DES
				\begin{itemize}
					\item Blocklänge $2t=64$, Runden $r=16$
					\item Schlüssel $K$ der Länge 64, in 8 Byte aufgeteilt
					\item Je letztes bit für ungerade Quersumme des Bytes
				\end{itemize}
			\item Algorithmus
				\begin{enumerate}
					\item Initiale Permutation $IP$ auf $p$ anwenden
					\item 16 Runden Feistel-Chiffre
					\item Inverse Permutation $IP^{-1}$\\
						$c=IP^{-1}(R_{16},L_{16})$
				\end{enumerate}
			\item Rundenschlüssel
				\begin{itemize}
					\item Generierung von 16 48-bit Schlüsseln aus 64-bit Schlüssel
					\item $(C_0,D_0)=PC1(k)$
					\item $C_i$ / $D_i$ entstehen durch Linksshift um 1 oder 2 bits aus vorherigem Wert
					\item $K_i=PC2(C_i,D_i)$
				\end{itemize}
			\item f-Funktion
				\begin{itemize}
					\item Expansionsfunktion verlängert von 32 auf 48 bit 
					\item $E(R)\oplus K$ bilden und in 8 Blöcke der Länge 6 aufteilen
					\item S-Boxen verkürzen Blöcke auf Länge 4\\
						$C_i=S_i(B_i)$
					\item S-Boxen sind Matrizen (Bit 0 und 6 als Zeile, Rest Spalte) und verhindern affin-linear
					\item Abschliessend noch Permutation
				\end{itemize}
			\item Sicherheit
				\begin{itemize}
					\item Triple-DES (EDE) noch sicher
					\item Brute-Force (Known-Plaintext) auf Schlüssel
					\item Lineare Kryptoanalyse (Known-P) schwer wegen S-Boxen
					\item Differentielle Kryptoanalyse komplexer als Brute-Force
				\end{itemize}
		\end{itemize}
	
	\section{AES-Algorithmus}
		\begin{itemize}
			\item Schlüssellänge 128 (4), 192 (6) oder 256 (8)
			\item Rundenzahl 10, 12 oder 14
			\item Blocklänge 128
			\item Bytes und Wörter als Polynom interpretiert
			\item Interner Zustand enthält $4*Nb=16$ Bytes (Spaltenweise)
			\item \texttt{KeyExpansion}
				\begin{itemize}
					\item Erzeugt expandierten Schlüssel mit\\
						$4*(Nr+1)$ Wörtern
					\item Ersten $Nk$ Wörter gleich Eingabeschlüssel
					\item Restliche Wörter durch Substitution, Shift und XOR generieren 
				\end{itemize}
			\item Algorithmus zur Verschlüsselung
				\begin{enumerate}
					\item \texttt{AddRoundKey}: Roundkey mit Blocklänge per XOR mit state verknüpfen
					\item \texttt{SubBytes}: Bytes gemäß S-Box substituieren (nicht linear und dynamisch berechenbar)
					\item \texttt{ShiftRows}: Zyklischer Linksshift der Zeilen (erste 1, zweite 2, \dots)
					\item \texttt{MixColums}: Spalten mit festem Polynom multiplizieren
				\end{enumerate}
			\item Entschlüsselung
				\begin{itemize}
					\item Gleiche Rundenschlüssel umgekehrt
					\item \texttt{InvSubBytes}: inverse S-Boxen
					\item \texttt{InvShiftRows}: Rechtsshift
					\item \texttt{InvMixColumns}: Mult. mit inv. Polynom
				\end{itemize}
			\item Angriffe
				\begin{itemize}
					\item Differentielle und Lineare Kryptoanalyse nicht anwendbar
					\item Saturation Attack: Chosen-Plaintext, nur bei verringerter Rundenzahl
					\item Biclique Attack: Known-Plaintext, Teilberechnungen nutzen um Komplexität minimal zu reduzieren
					\item Timing Attacks in MixColumns
					\item Power Analysis kann mit fester Instruktionsreihenfolge verhindert werden
				\end{itemize}
		\end{itemize}
	
	\section{RSA-Verfahren}
		\begin{itemize}
			\item Schlüsselerzeugung
				\begin{itemize}
					\item Wähle Zufällig Primzahlen $p,q$ sodass $n=pq$ eine $k$-bit Zahl
					\item Wähle $e$ mit $1<e<\phi (n)=(p-1)(q-1)$\\
						und $gcd(e,\phi (n))=1$
					\item Berechne $d$ mit $1<d<\phi (n)$ und\\
						$de\;mod\;\phi (n)=1$
					\item Schlüssel: $(n,e)$ bzw. $(n,d)$
				\end{itemize}
			\item $k$ groß, $e$ klein aber nicht zu klein
			\item Enc: $c=t^e\;mod\;n$
			\item Dec: $t=c^d\;mod\;n$
			\item Schnelle Exponentiation: Schrittweise Quadrate aufmultiplizieren
			\item Aufteilen der Nachricht in Blöcke und diese als Zahl interpretieren
			\item Sicherheit
				\begin{itemize}
					\item RSA-Problem: Aus $n,e,c$ soll $t$ ermittelt werden
					\item RSA-Schlüsselproblem: aus Public Key private berechnen
					\item Von Faktorisierung von $n$ abhängig
					\item Chosen-Plaintext möglich
					\item Hastad (Low-Exponent): Aus $e$ Ciphertexten zu teilerfremden $n_i$ kann $t$ berechnet werden durch chin. Restsatz
					\item Multiplikativer Homomorphismus
				\end{itemize}
			\item Padding
				\begin{itemize}
					\item Randomisiertes Padding OAEP
					\item Erzeuge Padding und bilde Hash
					\item XOR mit Plain, wieder hashen und XOR mit Padding
					\item Padding kann bei Dec wieder rausgerechnet werden
				\end{itemize}
		\end{itemize}
	
	\section{Diffie-Hellmann und ElGamal}
		\begin{itemize}
			\item Primitivwurzel $g$ entspricht erzeugendem Element
			\item Diskreter Logarithmus $a$ ist $A\equiv g^a\;mod\;p$
			\item D-H-Schlüseltausch
				\begin{enumerate}
					\item A und B einigen sich auf Primzahl $p$ und Primitivwurzel $g$
					\item Wählen Zahlen $a$ und $b$ und berechnen $A=g^a\;mod\;p$
					\item Tauschen $A$ und $B$ aus
					\item Berechnen $B^a\;mod\;p=g^{ab}\;mod\;p=K$
				\end{enumerate}
			\item $p$ muss hinreichend groß gewählt werden und bestimmte Bedingungen erfüllen
			\item $g$ Primitivwurzel oder zumindest große Ordnung und $\leq p-2$
			\item MitM Angriff möglich
			\item ElGamal
				\begin{enumerate}
					\item Berechne $A=g^a\;mod\;p$\\
						$\Rightarrow$ Pub. Key: $(p,g,A)$
					\item Enc: $c=A^bm\;mod\;p$
					\item Sende: $(B,c)$
					\item Dec: $m=B^xc$ mit $x=p-1-a$
				\end{enumerate}
			\item Sicherheit
				\begin{itemize}
					\item Schwierigkeit basiert auf diskr. Log.
					\item $b$ bei jeder Nachricht neu wählen, ansonsten Known-Plaintext
					\item Multiplikativer Homomorphismus:\\
						$c_2c_1^{-1}=m_2m_1^{-1}\Rightarrow m_2=c_2c_1^{-1}m_1\;mod\;p$ 
					\item Chosen-Cipher entspricht multiplizieren eigener Nachricht und dann teilen
				\end{itemize}
			\item Cipher doppelt so lang wie Klartext
			\item 2 Exponentiationen aber unabhängig von m $\Rightarrow$ effizienter
		\end{itemize}
	
	\section{Hashfunktionen}
		\begin{itemize}
			\item Abbildung von beliebiger auf feste Länge
			\item Kompressionsfunktion bildet von Länge $m$ auf $n$ ab und ist nie injektiv
			\item Voraussetzungen
				\begin{itemize}
					\item Effiziente Berechenbarkeit
					\item Urbildresistenz
					\item Schwache Koll. Resistenz (kein zweites Urbild mit gleichem Hash)
					\item Starke Koll. Resistenz (kein Paar)
				\end{itemize}
			\item Kompressionsfunktionen basieren auf Blockchiffren, alg. Strukturen oder Bitoperationen
			\item Merkle-Damgard Konstruktion
				\begin{itemize}
					\item Initialisierungsvektor $IV$
					\item Padding auf Vielfaches der Blocklänge
					\item Schrittweise Berechnung des Hashes
				\end{itemize}
			\item SHA-1
				\begin{itemize}
					\item 160 bit Länge
					\item Padding: 1, 0en bis $k*512-64$ dann Länge in 64 bit
					\item 5 32-bit Zustands Wörter
					\item Jeden Block als 16 32-bit Wörter schreiben und mehrere Bitoperationen zur Kompression verwenden
					\item Hash ist finaler Zustand
				\end{itemize}
		\end{itemize}
	
	\section{Digitale Signaturen}
		\begin{itemize}
			\item Sichern Authentizität und Unabstreitbarkeit
			\item Angriffsziele
				\begin{itemize}
					\item Universelle Fälschung (priv. Key)
					\item Selektive Fälsch. (Sig. zu Msg.)
					\item Existentielle Fälsch. (Finde Paar)
				\end{itemize}
			\item No-Message, Known-Message,\\
				Chosen-Message Angriffe
			\item Lamport-Diffie Einmal Signaturverfahren (LD-OTS)
				\begin{itemize}
					\item Schlüssel nur einmal verwenden
					\item Schlüssel sind Matrizen mit $k$ Zeilen und $2k$ Spalten
					\item Privater Key direkt generiert (Spaltenpaare)
					\item Öffentlicher Schlüssel durch Anwendung von Hashfunktion $h$ auf Spalten
					\item Generierung der Signatur durch Auswahl von Spalten des priv. Keys gemäß Nachricht
					\item Überprüfung durch Hashen der Spalten und Vergleich mit pub. Key
				\end{itemize}
			\item RSA
				\begin{itemize}
					\item Erzeuge RSA-Modul $n=pq$
					\item Wähle $e$ und berechne $d$
					\item Pub: $(n,e)$, Priv: $d$
					\item Signatur: $s=m^d\;mod\;n$
					\item Testen: $m=^?s^e\;mod\;n$
				\end{itemize}
			\item No Message, Mult. Homomorph. Angriff
			\item Merkle Verfahren
				\begin{itemize}
					\item Mehrere Einmal-Priv. Keys für einen Pub. Key
					\item $N=2^H$ verifizierbare Signaturen
					\item Wähle $N$ Einmal-Schlüsselpaare
					\item Merkle-Hashbaum
						\begin{itemize}
							\item Blätter: Hashwerte der Ver. Keys
							\item Inner: Hashwerte der Kombination der Kinder
						\end{itemize}
					\item Signierschlüssel: Folge der Priv. Keys
					\item Ver. Key: Wurzel $R$ des Hashbaums
					\item Signieren
						\begin{itemize}
							\item Index $i$ des verwendeten Keys
							\item Signatur $S$
							\item Authorisierungspfad für Verifikationsschlüssel (Hashes der Geschwisterknoten)
						\end{itemize}
					\item Prüfen
						\begin{itemize}
							\item Prüfe Einmal-Signatur mit geschicktem Test-Key
							\item Prüfe Key durch Hashberechnung bis zur öffentlichen Wurzel
						\end{itemize}
				\end{itemize}
		\end{itemize}
	
	\section{Public-Key-Infrastrukturen}
		\begin{itemize}
			\item Zertifikate Bestätigen Zugehörigkeit von Schlüsseln
			\item Inhalt: Aussteller, Gültigkeit, Inhaber, Key, Signatur
			\item Certification Authority stellt Certs aus
			\item Registration Authority Identifiziert Nehmer
			\item Validitation Authority stellt Dienste zur Überprüfung
			\item Domain (Besitzer) / Organization (Identität) Validation
			\item Extended Validation: Erweiterte Überprüfung
			\item Zertifikatsketten bis zur Root CA (self signed)
			\item Cross-Zertifizierung / Bridge-CAs
			\item Kompromittierte CA macht alle Zertifikate unterhalb ungültig
			\item Web of Trust
				\begin{itemize}
					\item Teilnehmer zertifizieren sich gegenseitig
					\item Schlüssel in Schlüsselservern
				\end{itemize}
		\end{itemize}
	
	\section{Primzahlerzeugung}
		\begin{itemize}
			\item Euklidischer Algorithmus zur Bestimmung des gcd
				\begin{itemize}
					\item Teile Größere Zahl mit Rest durch Kleinere
					\item Setze Größere = Kleinere und Kleinere = Rest bis Kleinere = 0
				\end{itemize}
			\item Probedivision
				\begin{itemize}
					\item Teile $n$ durch alle Primzahlen $p<\sqrt{n}$
					\item Falls $p$ restlos teilt $\Rightarrow$ zusammengesetzt
				\end{itemize}
			\item Sieb des Eratosthenes
			\item Sätze
				\begin{itemize}
					\item  Kleiner Fermat: $a^p\equiv a\;mod\;p$
					\item $\varphi(n)$: Anzahl aller kleineren teilerfremden Zahlen
					\item Fermat-Euler: $n\geq2$ und $a\in\Z$ teilerfremd $\Rightarrow a^{\varphi(n)}\equiv1\;mod\;n$
					\item Wenn $x^2\equiv1\;mod\;p$ dann gilt $x\equiv1$ oder $x\equiv-1$
					\item Fermat-Miller
						\begin{itemize}
							\item $p$ prim und $a$ teilerfremd
							\item Setze $p-1=d*2^s$
							\item Entweder $a^d\equiv1$ oder\\
								$\exists i\leq s-1(a^{d*2^i}\equiv-1)$ 
						\end{itemize}
				\end{itemize}
			\item Fermat-Test
				\begin{enumerate}
					\item Wähle Basis $a\in\Z$
					\item Falls $gcd(a,n)\neq1$ zusammengesetzt
					\item Verwende Satz von Fermat-Euler mit $\varphi(n)=n-1$
					\item Wenn Ergebnis nicht 1 dann zusammengesetzt
					\item Sonst wahrscheinlich prim
				\end{enumerate}
			\item Pseudoprimzahlen erfüllen für teilerfremdes $a$: $a^{n-1}\equiv1$
			\item $\Rightarrow$ für mehrere Basen testen
			\item Carmichael-Zahlen besitzen keine geeignete Basis
			\item Miller-Rabin Test
				\begin{enumerate}
					\item Schreibe $n-1=d*2^s$und wähle $a$ zwischen 1 und $n-1$ 
					\item Wenn $gcd(a,n)\neq1$ ist $n$ zusammengesetzt
					\item Berechne $b=a^d\;mod\;n$ (wenn 1, dann prim)
					\item Berechne $b,b^2,\dots,b^{2^{s-1}}$
					\item Wenn kein b gleich -1 dann zusammengesetzt
				\end{enumerate}
			\item Keine starken Carmichael Zahlen aber starke Pseudoprimzahlen
			\item Irrtumswahrscheinlichkeit $\frac{1}{2}$
			\item Erzeugung: zufällige $n$-stellige Zahl, erst Probedivision, dann t-mal Miller-Rabin
		\end{itemize}
	
	\section{Identifikationsmechanismen}
		\begin{itemize}
			\item Garantieren Zugriffskontrolle und Zurechenbarkeit
			\item Person = Identität?
			\item Proofer vs. Verifier
			\item Körperliche Merkmale, Besitz, Wissen
			\item Passwörter
				\begin{itemize}
					\item Hashvergleich
					\item Wörterbuchangriff
				\end{itemize}
			\item Sicherer Kanal, sichere Hashspeicherung
			\item Einmal Passwörter (evtl. mit Funktion)
			\item Public-Key-Challenge-Response: Proofer signiert Zufallszahl
			\item Zero-Knowledge: Vollständigkeit, Zuverlässigkeit, Zero-Knowledge
			\item Mit wahrscheinlichkeit $\varepsilon$ falsche Proofer ablehnen
			\item Ali-Babas-Cave
			\item Fiat-Shamir
				\begin{itemize}
					\item RSA-Modul $n=pq$
					\item Priv. Key: $s\in\Z_n$
					\item Pub. Key: $v=s^2\;mod\;n$
					\item Commitment: Proofer schickt $x=r^2\;mod\;n$
					\item Challenge: Verifier wählt $e\in\Z_2$
					\item Response: $y=r*s^e$
					\item Verifier prüft: $y^2=xv^e\;mod\;n$
				\end{itemize} 
			\item Simulation erzeugt gleiches Transkript $\Rightarrow$ perfektes Z-K Protokoll
			\item Bei Wiederverwendung von $r$ mit verschiedenen $e$ kann privater Schlüssel durch Division errechnet werden
		\end{itemize}
	
	\section{Elliptische Kurven}
		\begin{itemize}
			\item Benötigt im Verhältnis zu \glqq{}klassischer\grqq{} Crypto für gleiche Sicherheit kürzere Schlüssel
			\item EC: $y^2=x^3+ax+b$
			\item Kurve über Primkörper definieren um Kommazahlen zu vermeiden
			\item Addition
				\begin{itemize}
					\item $P+0=P$
					\item $P+(-P)=0$
					\item Ansonsten $\lambda$ berechen, je nachdem ob $P_1=P_2$
					\item Dann Formel für $x_3$ und $y_3$
				\end{itemize}
			\item Multiplikation als Reihe von Additionen
			\item Nicht alle EC sind sicher $\Rightarrow$ Vertrauen in Anbieter
			\item Seitenkanalattacken möglich
		\end{itemize}



\end{document}
