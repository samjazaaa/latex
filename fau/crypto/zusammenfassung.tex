\documentclass[11pt, paper=a4, twocolumn]{scrartcl}

\usepackage[ngerman]{babel}
\usepackage[utf8]{inputenc}

\usepackage[T1]{fontenc}

\usepackage{geometry}

\geometry{a4paper, top=20mm, left=15mm, right=15mm, bottom=20mm,
headsep=5mm, footskip=12mm}


\pagenumbering{gobble}

\title{\vspace{-1.25cm}Zusammenfassung Crypto\vspace{-0.25cm}}
\date{\vspace{-5ex}}



\begin{document}
	\maketitle


	%\section{Allgemeine Kryptosysteme}
	%	\begin{itemize}
	%		\item 
	%	\end{itemize}

	\section{Blockchiffren}
		\begin{itemize}
			\item Symmetrisch und deterministisch
			\item Verschlüsselung von Blöcken gleicher Länge
			\item Stromchiffren
				\begin{itemize}
					\item Meist Blöcke variabler Länge
					\item Kein Warten bis Block voll
				\end{itemize}
			\item Betriebsmodi
				\begin{itemize}
					\item ECB (Electronic Code Book)
						\begin{itemize}
							\item Nachricht in 
								gleich große 
								Blöcke teilen
							\item Evtl. Padding
							\item Gleiche Blöcke gleich verschlüsselt
							\item Keine Rückkopplung / Randomisierung
						\end{itemize}
					\item CBC (Cipher Block Chaining)
						\begin{itemize}
							\item Teilen / Padding
							\item Initialisierungsvektor
							\item Rückkopplung mit Ciphertext des vorherigen Blocks
							\item Pseudorandomisierung
						\end{itemize}
					\item CFB (Cipher Feedback Mode)
						\begin{itemize}
							\item Selbstsynchronisierende Stromchiffre
							\item Initialisierungsvektor
							\item Rückkopplung mit vorherigem Ciphertext in Schieberegister
						\end{itemize}
					\item OFB (Output Feedback Mode)
						\begin{itemize}
							\item Synchrone Stromchiffre
							\item Rückkopplung mit verwendetem Teil des zuvor generierten Schlüsselteils
						\end{itemize}
					\item CTR (Counter Mode)
						\begin{itemize}
							\item Zähler wird zu Initialisierungsvektor addiert
							\item IV also immer NONCE
							\item Schutz vor Replay
						\end{itemize}
				\end{itemize}
		\end{itemize}
	
	\section{Affin lineare Blockchiffren}
		\begin{itemize}
			\item dummy
		\end{itemize}
	
	\section{Perfekte Geheimhaltung}
		\begin{itemize}
			\item dummy
		\end{itemize}
	
	\section{DES-Algorithmus}
		\begin{itemize}
			\item dummy
		\end{itemize}
	
	\section{AES-Algorithmus}
		\begin{itemize}
			\item dummy
		\end{itemize}
	
	\section{RSA-Verfahren}
		\begin{itemize}
			\item dummy
		\end{itemize}
	
	\section{Diffie-Hellmann und ElGamal}
		\begin{itemize}
			\item dummy
		\end{itemize}
	
	\section{Hashfunktionen}
		\begin{itemize}
			\item dummy
		\end{itemize}
	
	\section{Digitale Signaturen}
		\begin{itemize}
			\item dummy
		\end{itemize}
	
	\section{Public-Key-Infrastrukturen}
		\begin{itemize}
			\item dummy
		\end{itemize}
	
	\section{Primzahlerzeugung}
		\begin{itemize}
			\item dummy
		\end{itemize}
	
	\section{Identifikationsmechanismen}
		\begin{itemize}
			\item dummy
		\end{itemize}
	
	\section{Elliptische Kurven}
		\begin{itemize}
			\item dummy
		\end{itemize}



\end{document}
