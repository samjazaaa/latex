\documentclass[11pt, paper=a4, twocolumn]{scrartcl}

\usepackage[ngerman]{babel}
\usepackage[utf8]{inputenc}

\usepackage[T1]{fontenc}

\usepackage{geometry}

\geometry{a4paper, top=20mm, left=15mm, right=15mm, bottom=20mm,
headsep=5mm, footskip=12mm}


\pagenumbering{gobble}

\title{\vspace{-1.25cm}Zusammenfassung IDB\vspace{-0.25cm}}
\date{\vspace{-5ex}}



\begin{document}
	\maketitle


	\section{Einführung}
		\begin{itemize}
			\item Datenunabhängigkeit
			\item Schichtenmodell
		\end{itemize}

	\section{Dateiverwaltung}
		\begin{itemize}
			\item physische vs. logische Speichergeräte
			\item Linearer vs. wahlfreier Zugriff
			\item Speicherhierarchie
			\item Magnetplattenspeicher: \\
				CHS / Zylinder, Spur, Block in Slot (Sektor)
			\item Gerätetreiber erzeugen \textbf{logische Speichergeräte}
				$\Rightarrow$ RAID, Partitionen
			\item \texttt{readBlock} und \texttt{writeBlock} adressieren die 
				Slots physisch
			\item \textbf{Vorteile:} schnell und maximale Ausnutzung 
			\item \textbf{Nachteile:} kein Schutz, Wissen über Zuordnung nötig, 
				Reservierung, starre Aufteilung
			\item Indirektion durch \textbf{Dateien} (dynamisch erweiterbare Folge von 
				Blöcken, abstrahiert von physischen Details)
			\item Blockdatei-Zugriffsschnittstelle erhält Name, Modus und 
				verwendete Blockgröße und legt Verwaltungsdatenstruktur an
			\item Operationen: \texttt{append(blocks)}, 
				\texttt{write(nr, buf)}, \texttt{read(nr, buf)}, 
				\texttt{size()}, \texttt{drop(blocks)}
			\item Verwaltungsstruktur: Abbildung Dateiname auf Folge von 
				Blöckenund Blocknummer zu physischer Slot-Adress
			\item Freispeicherverwaltung (z.B. Bitliste)
		\end{itemize}

	\section{Sätze}
		\begin{itemize}
			\item Variable, von Anwendung bestimmte, Länge 
			\item Folge von Bytes (keine innere Struktur)
			\item \textbf{Satz-Datei:} Menge von Sätzen fester / variabler 
				Länge\\~
			\item Blockung
				\begin{itemize}
					\item mehrere Sätze pro Block
					\item Jeder Satz vollständig in Block
					\item Feste Länge: Position aus Blockgröße und 
						Satzlänge errechenbar
					\item Variable Länge: Umsortieren für wenig 
						ungenutztem Platz 
				\end{itemize}
			\item Sequenzielle Satzdatei
				\begin{itemize}
					\item Folge von Sätzen sequentiell in Blockdatei 
						gespeichert
					\item Nur sequentieller Lese-/Schreibzugriff
					\item Nicht wahlfrei oder Einfügen / Ändern
					\item Stapelverarbeitung
					\item Öffnen mit Name, Modus und Länge (oder 0)
					\item \texttt{readNext(len, buf)}, 
						\texttt{writeNext(buf, len)}
				\end{itemize}
			\item Puffer
				\begin{itemize}
					\item Sätze puffern vor dem Schreiben
					\item Wechselpuffertechnik (einer wird 
						geschrieben, anderer kann verwendet 
						werden, Prefetching)
					\item Allgemeiner Puffer
						\begin{itemize}
							\item Pufferrahmen (Kacheln) für 
								Blöcke
							\item Beim Lesen n Blöcke 
								(Prefetching)
							\item Beim schreiben mehrere 
								füllen, dann schreiben
						\end{itemize}
				\end{itemize}
			\item Direktzugriff durch Satzadresse (eindeutig, unveränderlich)
			\item TID-Konzept
				\begin{itemize}
					\item Array mit Anfangsadressen aller Sätze im 
						Block am Ende
					\item Satzadresse: Blocknummer + Index
					\item Löschen: Arrayeintrag als ungültig markieren
					\item Bei Überlauf wird statt dem Satz eine TID 
						mit Verweis auf neuen Block gespeichert
				\end{itemize}
			\item Direkte Satzdatei
				\begin{itemize}
					\item \texttt{TID insert(buf, len), 
						read(TID, buf, len), 
						modify(TID, buf, len), delete(TID)}
					\item Sequentielles Lesen weiterhin möglich 
						(undefinierte Reihenfolge)
				\end{itemize}
			\item Freispeicherverwaltung
				\begin{itemize}
					\item Am Anfang des Blocks ungünstig
					\item k Blöcke am Anfang der Datei als 
						Freispeichertabelle (freie Bytes im i-ten 
						Block)
				\end{itemize}
		\end{itemize}

	\section{Schlüsselzugriff}
		\begin{itemize}
			\item Hashing
				\begin{itemize}
					\item Berechnung der Blocknummer aus Suchschlüssel
					\item Voraussetzung: Anzahl der Sätze abschätzbar
					\item Spezielle Blöcke als Buckets
					\item Geeignete Hash-Funktion (evtl. modulo)
					\item Überlauf
						\begin{itemize}
							\item Open Addressing (auslagern 
								in Nachbarbuckets)
							\item Overflow-Buckets (Pointer im 
								Primärbucket)
						\end{itemize}
					\item \texttt{inser(buf,len,key), read(key,len)}
					\item \texttt{modify-key(old,new)}
					\item Vorteile: schnell
					\item Nachteile: keine Reihenfolge, Speicherplatz 
						muss reserviert werden, nur nach einem 
						Schlüssel speicherbar
				\end{itemize}
			\item Virtuelles Hashing
				\begin{itemize}
					\item $q$ Buckets mit je $b$ Sätzen
					\item Belegungsfaktor $\beta$ = gespeichert / 
						Kapazität
					\item Wenn $\beta>\alpha=0.8$ mehr Buckets
					\item VH1
						\begin{itemize}
							\item $q$ neue Buckets dahinter
							\item Bei Einfügen in Bucket mit 
								Überläufern, Rehash der 
								Sätze mit $h2$ und Bit 
								$i$ setzen
						\end{itemize}
					\item Lineares Hashing
						\begin{itemize}
							\item $\alpha$ überschritten
								$\Rightarrow$ 1 neuer 
								Bucket am Ende
							\item Dann Bucket auf dem Zeiger 
								$p$ steht aufteilen
							\item Erst $h1$, wenn kleiner als 
								$p$ dann $h2$
							\item Nach Bucket $q-1$, $p=0$
						\end{itemize}
				\end{itemize}
			\item B-Bäume
				\begin{itemize}
					\item Knoten: Anzahl Einträge ($k\leq n\leq 2k$)
					\item Wurzel kann auch nur einen Eintrag haben
					\item Inhalt: $n$, $P_0$ k-mal ($K_i,D_i,P_i$)
					\item Pfad von Wurzel nach Blatt hat immer Länge 
						$h-1$
					\item Außer Wurzel immer mindestens $k+1$ 
						Nachfolger
					\item Wurzel ist Blatt oder hat mind. 2 Nachfolger
					\item Höchstens $2k+1$ Nachfolger
					\item Einfügen nur in Blattknoten
					\item Blatt voll $\Rightarrow$ Split
						\begin{itemize}
							\item $k$ link, $k$ rechts
							\item mittleres Element als 
								Diskriminator in 
								Elternknoten
							\item Wurzelsplit erhöht Höhe
							\item Auslastung $\geq 50\%$
						\end{itemize}
					\item Löschen im B-Baum
						\begin{itemize}
							\item Kleinstes Element aus 
								rechtem Unterbaum 
								/ größtes aus Linkem 
								nachrücken
							\item Unterlauf $\Rightarrow$ 
								Mischen
							\item Mischen: Blätter + 
								Diskriminator in einen 
								Knoten
						\end{itemize}
				\end{itemize}
			\item B*-Baum
				\begin{itemize}
					\item Sätze / TIDs nur in Blattknoten gespeichert
					\item Innerer Knoten: $n$, $P_0$, m-mal $R_i,P_i$
					\item Blattknoten: $n$, $V$, j-mal $S_i,D_i$, $N$
					\item Bei Split eins mehr rechts und größtes 
						Linkes als Diskriminator
					\item Bei Unterlauf: Wenn Nachbarsumme nicht 
						größer als 2k einfach Mischen ansonsten 
						Nachbarn neu aufteilen
				\end{itemize}
			\item Bitmap-Index
				\begin{itemize}
					\item Lohnt sich bei niedriger Selektivität 
						(Anteil passender Sätze von allen)
					\item Für jeden Schlüsselwert Bitliste
					\item Feste Reihenfolge notwendig
					\item Größe: Werte x Anzahl Sätze
					\item Einfache logische Verknüpfungen
				\end{itemize}
			\item Zugriff:\\ \texttt{insert(buf,len,key), read(key,len), 
				modKey(old,new)}
			\item Organisation
				\begin{itemize}
					\item Primär: Speicherung Satz in Block (seq., 
						direkt, Schlüssel)
					\item Sekundär: Verweis, nur möglich wenn Primär 
						Direktzugriff unterstützt
					\item Einfügen: In Satzdatei und (Key,TID) in 
						Indexdateien
					\item Suchen: read in Satzdatei, read in Index, 
						Scan über Satzdatei
				\end{itemize}
			\item Primärschlüssel: darf nur in einem Satz vorkommen
		\end{itemize}

	\section{Puffer}
		\begin{itemize}
			\item Frame: Platz für einen Block im Hauptspeicher
			\item logischer Zugriff $\neq$ physischer Zugriff
			\item Bei Verdrängung geänderte Blöcke zurückschreiben
			\item Ersetzungsverfahren
				\begin{itemize}
					\item FIFO: ungünstig bei Direktzugriff
					\item LFU: Häufig benutzte Blöcke bleiben evtl. zu 
						lange im Puffer
					\item LRU: Stapel, aufwändig
					\item CLOCK: used-Bit bei Nutzung auf 1 
				\end{itemize}
			\item Noch benutzte Blöcke dürfen nicht verdrängt werden!
			\item Schnittstelle
				\begin{itemize}
					\item \texttt{fix(File,BlockNo,Mode)} fixiert 
						Block im Buffer
					\item \texttt{unfix(bufAddr)} Freigabe zur 
						Ersetzung
					\item read durch fix ersetzen, unfix einfügen und 
						write weglassen / durch unfix ersetzen
					\item leerer Block muss auch gefixed werden
				\end{itemize}
			\item Im Fehlerfall Inkonsistenzen durch Hauptspeicherverlust
			\item Seite = Block im Puffer
			\item Evtl. mehrere Blöcke für Seite (alt und neu)	
			\item Einbringungsstrategien
				\begin{itemize}
					\item Nicht direkt schreiben sondern in extra 
						Block und dann atomar umschalten
				\end{itemize}
			\item Segment
				\begin{itemize}
					\item Entspricht Datei (Folge von Seiten)
					\item Einheit des Sperrens, Recovery und 
						Zugriffskontrolle
					\item Zuordnung Segmente-Dateien systemabhängig
					\item Segmenttypen
				\end{itemize}
			\item Seitenzuordnung
				\begin{itemize}
					\item Direkt: Seitenfolgen direkt auf Blockfolgen 
						einer Datei abbilden (N:1, 1:1)
					\item Indirekt: Mehr Flexibilität aber 
						Hilfsstruktur nötig
					\item Seitentabelle enthält jeweilige Blockadresse
					\item Datei hat Bitliste, die aktuelle Belegung 
						angibt
				\end{itemize}
			\item Seiteneinbringung
				\begin{itemize}
					\item Direkt
						\begin{itemize}
							\item Beim Verdrängen wird 
								Einlagerungsblock 
								überschrieben
							\item Ein Block pro Seite
							\item Alter Zustand vor Änderung 
								sichern (Write-Ahead-Log)
						\end{itemize}
					\item Indirekt
						\begin{itemize}
							\item Beim Verdrängen in freien 
								Block schreiben
							\item Schattenspeicher / Twin 
								Slots
						\end{itemize}
					\item Schattenspeicher
						\begin{itemize}
							\item Alle Seiten eines Segments 
								in Sicherungsintervall 
								konsistent speichern
						\end{itemize}
					\item Twin Slots
						\begin{itemize}
							\item Direkte Seitenzuordnung aber 
								doppelt
							\item Seiten brauchen 
								Versionsnummer
							\item Immer zwei Blöcke lesen
						\end{itemize}
				\end{itemize}
			
		\end{itemize}

	\section{Programmzugriff}
		\begin{itemize}
			\item Eingebettetes SQL
				\begin{itemize}
					\item Vorübersetzer wandelt in C um
					\item \texttt{EXEC SQL <SQL>;}
					\item INTO Klausel für Ergebnisse, 
						Programmvariablen durch Doppelpunkt 
						kennzeichnen
					\item \texttt{exec sql include sqlca;}
					\item \texttt{exec sql begin / end declare 
						section;}
					\item \texttt{exec sql declare c1 cursor for 
						<SQL>;}
					\item \texttt{exec sql open c1;}
					\item \texttt{exec sql fetch c1 into :pnr;}
					\item \texttt{if(sqlca.sqlcode==100) break;}
					\item \texttt{exec sql close c1;}
					\item Erzeugt optimierte Zugriffsmodule zu 
						Übersetzungszeit

				\end{itemize}
			\item Unterprogrammaufruf
				\begin{itemize}
					\item Call-level interface
					\item Treiber Laden
					\item \texttt{Connection con = 
						DrvrMngr.getConn(url,"foo","bar")}
					\item \texttt{Statement anw = 
						con.createStatement()}
					\item \texttt{ResultSet res = 
						anw.executeQuery(``SELECT")}
					\item \texttt{res.next() / getInt(1)}
					\item res und anw closen
					\item \texttt{PreparedStatement ins = 
						con.prepareStatement (``INSERT ?,?")}
					\item \texttt{setBoolean(1,true)}
					\item \texttt{CallableStatement ca = 
						con.prepareCall("\{ call Test (?,?)\}")}
				\end{itemize}
		\end{itemize}

	\section{Transaktionen}
		\begin{itemize}
			\item Programmfehler, Systemfehler, Gerätefehler
			\item Physische Konsistenz
				\begin{itemize}
					\item Strukturen und TIDs stimmen
					\item Indexe vollständig und stimmen mit 
						Primärdaten überein
				\end{itemize}
			\item Logische Konsistenz
				\begin{itemize}
					\item Zustand der realen Welt
					\item Referenzielle Integrität, Assertions usw.
				\end{itemize}
			\item Undo und Redo durch Logging ermöglichen
			\item Transaktion
				\begin{itemize}
					\item Folge von DB Ops. von konsistent nach 
						konsistent
					\item Fehler vor Ende: undo, danach redo
					\item commit vs. abort
					\item Atomar (alles oder nichts)
					\item Konistent (so lange wie nötig)
					\item Isolation (jede Transaktion alleine)
					\item Dauerhaft (Ergebnisse persistent)
				\end{itemize}
		\end{itemize}
	
	\section{Speicherung von Tupeln}
		\begin{itemize}
			\item Systemkatalog enthält Infos über Felder
			\item Gleicher Satztyp für Tupel derselben Relation
			\item Speicherungsstruktur in Sätzen
				\begin{itemize}
					\item Konkatenation fester Länge: speicheraufwändig
					\item Zeiger im Vorspann: unflexibel beim 
						Hinzufügen
					\item Eingebettete Längenfelder: dynamisch aber 
						keine direkte Berechnung der internen 
						Adresse aus Katalogdaten
					\item Gesamtlänge, FL, L-V, ...
					\item Eingebettete mit Zeiger:\\
						GL,FL,F,P,F,...,L,V,...
				\end{itemize}
			\item C-Stores
				\begin{itemize}
					\item Speichert Sammlung von Spaltengruppen
					\item Spaltengruppe: Projektion (nach Attr. 
						sortiert)
					\item Schreibspeicher, Lesespeicher und 
						Tuple Mover
					\item Projektionen: Verbindung zusammengehöriger 
						Attribute (verschiedener) Relationen
					\item \texttt{EMP1 (name,DEPT.floor | name)}
					\item Für jede Tabelle überdeckende 
						Projektionsmenge
					\item Speicherschlüssel ermöglicht Rekonstruktion 
						der Tupel (errechenbar aus Pos.)
					\item Verbund-Indexe: Umsortierung gemäß SK
					\item Physische Strukturen / Komprimierung
						\begin{itemize}
							\item Sortiert, wenig Werte:\\
								Tripel 
								(value,first,count), 
								organisiert in B-Baum
							\item Unsort., wenig:\\
								(val,bitmap) mit 
								Lauflängencodierung, 
								Offset-Indexe
							\item Sort., versch.:\\
								Delta-Codierung: Diff 
								zum Vorgänger in B-Baum
							\item Unsort., versch.:\\
								keine Kompr., B-Baum als 
								sekundär möglich
							\item Bei Zeichenketten Dictionary
						\end{itemize}
					\item Schreibspeicher
						\begin{itemize}
							\item Gleiche Projektionen aber 
								andere Struktur
							\item SK explizit gespeichert
							\item keine Komprimierung
						\end{itemize}
				\end{itemize}
		\end{itemize}
	
	\section{Anfrageverarbeitung}
		\begin{itemize}
			\item Unabhängige vs. korrelierte Unterabfragen
			\item Anfrageverarbeitung (Ausführungsplan) vs. Anfrageausführung
			\item Phasen der Anfrageverarbeitung
				\begin{itemize}
					\item Analyse (Parser, Zugriffskontrolle, 
						Integritätskontrolle)
					\item Optimierung (Optimizer)
					\item Code-Generierung
					\item Ausführung (Ausführungskontrolle, 
						Interpreter)
				\end{itemize}
			\item Analyse
				\begin{itemize}
					\item Lexikalische und syntaktische Analyse 
						$\Rightarrow$ Anfragebaum
					\item Semantische Analyse (Existenz, Gültigkeit, 
						Konvertierung)
					\item Zugriffs- und Integritätskontrolle
				\end{itemize}
			\item Vereinfachung
				\begin{itemize}
					\item Standardisierung in Normalform ohne 
						Redundanzen
					\item Restrukturierung: Algebraisch (Struktur) vs. 
						Planoperatoren (Transformation)
				\end{itemize}
			\item Interndarstellung einer Anfrage
				\begin{itemize}
					\item logische Operatoren mit 
						Anfrage-/Operatorbaum darstellen
					\item \texttt{SEL(R,pre(...)), PROJ(R,L), 
						CROSS(R,S), JOIN(R,S,pred)}
					\item \texttt{UNION(R,S), INTERSECT(R,S), 
						EXCEPT(R,S)} 
					\item Multimengen Operatoren
					\item \texttt{RENAME, DUP-ELIM, SUM, COUNT, GROUP, 
						G-PROJ, SORT, OUTER-JOIN}
				\end{itemize}
			\item Standardisierung und Vereinfachung
				\begin{itemize}
					\item Normalformen (CNF)
					\item Idempotenzregel, nicht erfüllbare Ausdrücke 
						usw.
					\item Konstanten-Propagierung
					\item Nutzung von semantischen 
						Integritätsbedingungen
				\end{itemize}
			\item Restrukturierung
				\begin{itemize}
					\item Verbund assoziativ und kommutativ
					\item Selektionen und Projektionen können jeweils 
						zusammengefasst werden
					\item Selektion und Verbund dürfen getauscht 
						werden
					\item Selektion und Kreuzprodukt zu Join
					\item Selektionen und Projektionen so früh wie 
						möglich
				\end{itemize}
		\end{itemize}
	
	\section{Relationale Operatoren}
		\begin{itemize}
			\item Ersetzen der logischen Operatoren durch Planoperatoren
			\item Selektion
				\begin{itemize}
					\item Relationen-Scan: Sequentiell alle Tupel 
						lesen
					\item Index-Scan: direkt auf ersten Match dann 
						sequentiell weiter
				\end{itemize}
			\item Projektion meist bei anderen Planoperatoren enthalten
			\item Sortierung
				\begin{itemize}
					\item Für ORDER BY, Joints, DISTINCT, Gruppierung
					\item blockierend $\Rightarrow$ kein Pipelining
				\end{itemize}
			\item Join allgemein
				\begin{itemize}
					\item n-Wege Join in 2-Wege Joins zerlegen
					\item Verbundattribut vs. Selektionsattribut
					\item Gleichverbund oder allgemein
				\end{itemize} 
			\item Nested-Loop-Join
				\begin{itemize}
					\item R, S nicht sortiert und keine Indexe über VA
					\item Scan über R; wenn SA, dann scan über S und 
						check VA und SA
					\item $O(n^2)$
					\item Wenn VA von S UNIQUE kann früher abgebrochen 
						werden
				\end{itemize}
			\item Nested-Loop mit Index
				\begin{itemize}
					\item $I_R(VA)$ und $I_S(VA)$ vorhanden
					\item Gleichverbund
					\item Scan über S und prüfe mit Index ob R 
						Partner hat
				\end{itemize}
			\item Sort-Merge-Verbund
				\begin{itemize}
					\item Sortiere Relationen nach VA (mit Überprüfung 
						SA)
					\item Gleichmäßig scannen und Partner prüfen
					\item $O(N\;log\;N)$
				\end{itemize}
			\item Hash-Verbund (Classic)
				\begin{itemize}
					\item Abschnittweise kleinere Relation R lesen
					\item In p Abschnitte aufteilen, die in 
						Hauptspeicher passen und SA erfüllen
					\item Aufbau einer Hashtabelle über VA
					\item Überprüfe jeden Satz in S der SA erfüllt ob 
						gehashtes VA matched
					\item $O(p\times N)$
					\item Verbesserung: Partitionierung von R nach VA
				\end{itemize}
			\item Simple Hashing mit doppelter Partitionierung
			\item Duplikat-Eliminierung durch sort and group oder hash
			\item Gruppierung durch sortieren oder hash
			\item Kostenarten: Berechnung, E/A, Speicher, Kommunikation
			\item Annahmen: Daten und Attribute gleichverteilt und Prädikate 
				unabhängig
			\item Kostenformel
				\begin{itemize}
					\item  Kosten: \#physische-Zugriffe + \\
						W $\times$ \#Aufrufe-Zielsystem
					\item CPU-bound $W_{CPU}$=\#Intr-pro-Zielaufruf / 
						\#Instr-pro-E/A
					\item I/O-bound: $W_{IO}$=\#Instr-pro-Ziel / \\ 
						(\#Instr-pro-E/A + Zugriffszeit $\times$ 
						MIPS)
				\end{itemize}
			\item Statistische Kenngrößen
				\begin{itemize}
					\item Segmente: Anzahl Daten / leeren Seiten
					\item Relationen: Tupel, Seiten, Clusterfaktor 
						(Tupel / Seite)
					\item Index: Anzahl Schlüsselwerte, Blattseiten
					\item Periodische Neubestimmung
				\end{itemize}
			\item Selektivitätsabschätzung
				\begin{itemize}
					\item Erwarteter Anteil an Tupeln, die Prädikat 
						erfüllen
					\item $A_i=a_i$: 1/\#keys falls Index, 1/10 sonst
					\item $A_i=A_k$: 1/max(\#keys) falls beide Index
					\item $A_i>a_i$: delta max / delta ges oder 1/3
				\end{itemize}
			\item Indexstrukturen reduzieren Laufzeit bei sehr geringen 
				Trefferraten / hoher Selektivität (5\%)
		\end{itemize}
	
	\section{Synchronisation}
		\begin{itemize}
			\item Datenbankkonsistenz vs. Transaktionskonsistenz 
				(nebenläufiger Ablauf)
			\item Anomalien
				\begin{itemize}
					\item lost update / dirty write: w-w-dependency\\
						$r_1[A],r_2[A],w_1[A],w_2[A],c_1,c_2$
					\item dirty read: Lesen eines geänderten Wertes, 
						dann rolback; write-read\\
						$w_1[A],r_2[A],w_2[A],a_1,c_2$
					\item Inkonsistente Analyse: Während Werte gelesen 
						werden, werden andere manipuliert; 
						read-write\\
						$r_2[A],w_1[A],w_1[B],r_2[B],c_1,c_2$
				\end{itemize}
			\item Serialisierbarkeit
				\begin{itemize}
					\item Serialisierbar wenn äquivalent zu seriellem 
						Ablauf
					\item $r_i[A] <_H w_j[A] \Leftrightarrow 
						r_i[A] <_G w_j[A]$
					\item Auch für w-r und w-w
				\end{itemize}
			\item Abhängigkeitsgraph
				\begin{itemize}
					\item Knoten Transaktionen, Kanten Konflikte
					\item Serialisierbar wenn keine Zyklen
				\end{itemize}
			\item Sperrverfahren
				\begin{itemize}
					\item Schreibsperre (x) vs. Lesesperre (s)
					\item Nur s-Sperren sind kompatibel
					\item Statisches (zu Beginn) vs. dynamisches 
						Sperren
					\item Freigabe erst am Ende der Transaktion
					\item Sperrgranulat Tupel: ineffizient, 
						Phantom-Problem
					\item Hierarchische Sperrung (DB, Segmente, 
						Relationen, Tupel)
					\item Intention Locks: IS/IX: untergeordnete 
						Objekte sind gesperrt
					\item Top Down bei Erwerb, Bottom Up bei 
						Freigabe
					\item SIX = S + IX: sperrt S, verlangt tiefer nur 
						noch IX/X; alle lesen, ein paar ändern
				\end{itemize}
			\item Optimistische Verfahren
				\begin{itemize}
					\item Zugriff ohne Sperren und bei commit 
						überprüfen, ob Überschneidungen
				\end{itemize}
			\item Deadlocks (Timeout, prevention, avoidance, detection)
		\end{itemize}
	
	\section{Recovery}
		\begin{itemize}
			\item Voraussetzung: Logging, Undo/Redo Mechanismen
			\item Physische (SQL) vs. logische Konsistenz (Transaktion)
			\item Fehler: Transaktion, System, Geräte
			\item Recovery-Klassen
				\begin{itemize}
					\item Partial Undo (R1): Transaktionsfehler
					\item Partial Redo (R2): Systemfehler
					\item Global Undo (R3): Systemfehler
					\item Global Redo (R4): Gerätefehler
				\end{itemize}
			\item Archivkopien: cold vs. hot
			\item Protokolldateien: physisch vs. logisch
			\item Einbringungsstrategien
				\begin{itemize}
					\item Steal: Bei Verdrängung schreiben, auch vor 
						TA-Ende
					\item NoSteal: Frühestens bei commit $\Rightarrow$ 
						kein Undo erforderlich aber großer Puffer
					\item NoForce: Erst bei Verdrängung schreiben, 
						auch wenn TA schon zu Ende
					\item Force: Bei commit schreiben $\Rightarrow$ 
						kein partial Redo
					\item NotAtomic: update in place
					\item Atomic: indirekte Einbringung z.B. 
						Schattenspeicher
				\end{itemize}
			\item Protokolldaten
				\begin{itemize}
					\item Physische Zustände: Before/After-Images
					\item Logische Übergänge: SQL-Ops
					\item Physische Übergänge: EXOR-Diffs.
					\item Undo Infos VOR Änderung schreiben: WAL
					\item Redo Info muss vor Abschluss der TA 
						geschrieben werden
					\item Seitenprotokollierung before/after: schnelle 
						Recovery aber hohes IO
					\item Eintragsprotokollierung: Nur geänderte Teile 
						protokollieren und Pufferung im 
						Hauptspeicher
				\end{itemize}
			\item Sicherungspunkte
				\begin{itemize}
					\item Direkter Sicherungspunkt: Alle geänderten 
						Seiten ausschreiben und einbringen
					\item Transaction-Oriented Checkpoint (TOC)
						\begin{itemize}
							\item Geänderte Seiten bei TA-Ende 
								einbringen (Force)
							\item kein Redo
						\end{itemize}
					\item Transaction-Consistent Checkpoint (TCC)
						\begin{itemize}
							\item Alle erfolgreichen TAs 
								einbringen
							\item Muss angemeldet werden und 
								setzt Lesesperre
							\item Verzögert neue TAs
							\item Besseres Undo + Redo
						\end{itemize}
					\item Action-Consistent Checkpoint
						\begin{itemize}
							\item Zu Beginn keine Änderungsops.
							\item Kürzere Totzeit aber auch 
								nur geringe Hilfe für Redo
						\end{itemize}
				\end{itemize}
			\item Systemfehler
				\begin{itemize}
					\item Direkt: nur physisches Logging
					\item Indirekt: physisch Konsistent, logisches 
						Logging möglich
					\item Restart
						\begin{itemize}
							\item Analyse-Lauf von CP bis 
								Log-Ende
							\item Redo-Lauf vorwärts
							\item Undo-Lauf rückwärts bis 
								älteste abgebrchene TA
						\end{itemize}
				\end{itemize}
			\item Geräte-Recovery: Full vs. Incremental Backup
		\end{itemize}

\end{document}
