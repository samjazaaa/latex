\documentclass[11pt, paper=a4, twocolumn]{scrartcl}

\usepackage[ngerman]{babel}
\usepackage[utf8]{inputenc}

\usepackage[T1]{fontenc}

\usepackage{geometry}

\geometry{a4paper, top=20mm, left=15mm, right=15mm, bottom=20mm,
headsep=5mm, footskip=12mm}


\pagenumbering{gobble}

\title{\vspace{-1.25cm}Einführung Psychologie\vspace{-0.25cm}}
\date{\vspace{-5ex}}



\begin{document}
	\maketitle


	\section{Psychologie als Wissenschaft}

	\begin{itemize}
		\item \textbf{wissenschaftliche} Untersuchung des \textbf{Verhaltens} von 
			Individuen und ihrer \textbf{kognitiven} Prozesse
		\item Verhalten: z.B. Lachen, Weinen, Rennen, usw.
		\item Kognition: z.B. denken, planen, schlussfolgern usw.
		\item Ziele
			\begin{enumerate}
				\item Beschreibung von Phänomenen
				\item Erklärung (Disposition vs. Umweltfaktoren)
				\item Vorhersage
				\item Kontrolle / Intervention
			\end{enumerate}
			$\Rightarrow$ 2. - 4.: \textbf{Theorie}
		\item Empiristische vs. Nativistische Sichtweise\\
			$\Rightarrow$ \textbf{Anlage-Umwelt-Kontroverse}
		\item Wilhelm Wundt: erstes experimentalpsychologisches Labor
		\item Spannungsfeld
			\begin{itemize}
				\item Strukturalismus (Struktur des Mentalen)
				\item Funktionalismus (Mehr als Summe der Teile)
				\item Gestaltpsychologie (Absicht mentaler Proz.)
			\end{itemize}
			$\Rightarrow$ Struktur \textbf{und} Funktion des Verhaltens
		\item Psychodynamische Perspektive
			\begin{itemize}
				\item Freud, Instinktgetrieben
				\item Verhalten Ausdruck unbewusster Motive
			\end{itemize}
		\item Behavioristische Perspektive
			\begin{itemize}
				\item Watson und Skinner, Umwelt-Stimulus-Interaktionen
				\item Antezedenzbedingungen $\rightarrow$ Reaktion 
					$\rightarrow$ Konsequenzen
			\end{itemize}
		\item Humanistische Perspektive
			\begin{itemize}
				\item Menschen aktiv und von Grund auf gut
				\item Verhaltensmuster, Werte und Ziele in der Lebensgeschichte
			\end{itemize}	
		\item Kognitive Perspektive
			\begin{itemize}
				\item Personen handeln weil sie denken
				\item Aufmerksamkeit, Erinnern, Gedächtnis, Sprache, Entscheiden
			\end{itemize}	
		\item Biologische Perspektive
			\begin{itemize}
				\item Biochemische Prozesse
				\item Verhalten als Ergebnis chem./el. Aktivität
			\end{itemize}
		
		\item Evolutionäre Perspektive
			\begin{itemize}
				\item Evolution, fittest $\rightarrow$ langsam
			\end{itemize}

		\item Kulturvergleichende Perspektive
			\begin{itemize}
				\item Interkulturelle Unterschiede als Ursachen / Konsequenzen
				\item Treffen Theorien auf alle Menschen zu?
			\end{itemize}
	\end{itemize}

	\section{Forschungsmethoden}

	\begin{itemize}
		\item Explicit theories vs. implicit attitudes
		\item Scientific literacy: verstehen und kritisch bewerten
		\item Theorien ermöglichen die Generierung von Hypothesen
		\item Vorgehen allgemein
			\begin{enumerate}
				\item Entdeckung (eines evtl. Zusammenhangs)
				\item Hypothese (meist aus Theorie), überprüfbar
				\item Versuchsplanung und Durchführung (Quantifizierung, Design, usw.)
				\item Datenanalyse und Schlussfolgerung
				\item Publikation der Ergebnisse
			\end{enumerate}
		\item Wichtige Begriffe
			\begin{itemize}
				\item Definition, Operationalisierung, Standardisierung
				\item (Un-)abhängige Variable
			\end{itemize}
		\item Spezifikation von Bedingungen und Prozeduren um 
			\textbf{Replizierbarkeit} zu gewährleisten
		\item Korrelationsstudien
			\begin{itemize}
				\item keine Manipulation der Variablen
				\item liefert Korrelationskoeffizienten
				\item Untersuchung von schwer manipulierbaren Merkmalen
				\item Probleme: Kausalitätsproblem, dritte Variable
			\end{itemize}
		\item Experimentalstudien
			\begin{itemize}
				\item Ein Merkmal manipulieren und Auswirkungen auf 
					andere Variablen messen
				\item Hauptmekrmale: Randomisierung, (un)abhängige 
					Variablen
				\item Kausalzusammenhänge gut ableitbar
				\item Manche Merkmale schwer kontrollierbar (Geschlecht)
			\end{itemize}
		\item Messung
			\begin{itemize}
				\item Deklarativ: Free self-description, Questionaire, 
					Checklist
				\item Non-deklarativ: Inhaltsanalyse, Interview, 
					kognitive- , behavioral- , physiological Tests
			\end{itemize}
		\item Gütekriterien
			\begin{itemize}
				\item Reliabilität (Genauigkeit)
				\item Validität (misst was es behauptet)
			\end{itemize}
		\item Korrelationskoeffizient $r$
			\begin{itemize}
				\item Zusammenhang zw. zwei Variablen
				\item $-1.00$ bis $1.00$
			\end{itemize}
	\end{itemize}

	\section{Biologische Grundlagen}

	\begin{itemize}
		\item Vererbung und Verhalten
			\begin{itemize}
				\item Natürliche Selektion (Genotyp des fittest Phänotyp 
					pflanzt sich besser fort)
				\item $\Rightarrow$ intelligenter und mobiler Phänotyp
				\item Erblichkeitskoeff. $h^2=(r_I-r_F)*2$
			\end{itemize}

		\item Nervensystem
			\begin{itemize}
				\item Zentral-NS vs. Peripheres NS
				\item Somatisches vs. Autonomes NS (periph.)
				\item Sympathikus (\glqq{}Notfall\grqq{}) vs. 
					Parasympathikus (\glqq{}Wartung\grqq{}) (Autonom)
			\end{itemize}

		\item Das Gehirn
			\begin{itemize}
				\item Großhirn (Cortex): Frontal-, Parietal-, Okzipital- 
					und Temporallappen
				\item Balken verbindet Hemisphären
				\item Motorischer (frontal), Somatosensorischer 
					(parietal), Auditorischer (Temporal) und Visueller 
					(okzipital) Cortex
				\item Hirnstamm (interne Körperprozesse)
				\item Thalamus (Weiterleitung an Cortex)
				\item Crebellum (Bewegung und Gleichgewicht)
				\item Limbisches System: Hippocampus (Gedächtnis), 
					Amygdala (Emotionen), Hypothalamus (Motivational)
				\item Endokrines System: Hypoth. $\rightarrow$ Hypophyse 
					$\rightarrow$ Hormone
			\end{itemize}

		\item Das Nervensystem in Aktion
			\begin{itemize}
				\item Neuron: Dendriten (in), Soma, Axon mt Endknöpfchen 
					(out) 
				\item Sensorische (in) vs. Motorneurone (out)
				\item Gliazellen als \glqq{}Kleber\grqq{}
				\item Aktionspotenzial: exzitatorische vs. inhibitorische 
					Inputsignale
				\item Synaptische Übertragung durch Neurotransmitter
			\end{itemize}

		\item Messmethoden
			\begin{itemize}
				\item Läsionen (auch vorübergehend durch 
					Magnetstimulation)
				\item Magnetresonanztomographie (anatomische Strukturen)
				\item Positronen-Emissions-Tomographie (radiokativ)
				\item Elektroenzephalographie (Spannungsänderungen)
			\end{itemize}

		\item Einfluss von Lebenserfahrungen
			\begin{itemize}
				\item Größe kann sich ändern (Synaptogenese vs. 
					Neurogenese)
			\end{itemize}

	\end{itemize}

	\section{Sensorische Prozesse und Wahrnehmung}

	\begin{itemize}
		\item Sensorisches Wissen
			\begin{itemize}
				\item Nahsinne vs. Fernsinne
				\item Signalentdeckungstheorie (Schwellenwerte)
				\item Webersches Gesetz (Konstante gibt unbemerkte 
					Abweichung an)
			\end{itemize}

		\item Das visuelle System
			\begin{itemize}
				\item Retina wandelt Licht in Nervenimpulse
				\item 2 + 2 Sehbahnen (Chiasma)
				\item Komplementärfarben
			\end{itemize}

		\item Hören
			\begin{itemize}
				\item Parameter: Höhe, Lautheit, Klangfarbe
				\item Tonhöhe: Ortstheorie vs. Zeittheorie
			\end{itemize}

		\item Weitere Sinne
			\begin{itemize}
				\item Olfaktorisches System
				\item Gustatorisches System
				\item Hautsinne, Gleichgewicht, Bewegung, Schmerz
			\end{itemize}

		\item Grundbegriffe Wahrnehmung
			\begin{itemize}
				\item Stufen der Wahrnehmung
					\begin{enumerate}
						\item Sensorische Prozesse
						\item Perzeptuelle Organisation
						\item Identifikation und (Wieder-)Erkennen
					\end{enumerate}
				\item Erkennen des distalen durch den proximalen Reiz
			\end{itemize}

		\item Aufmerksamkeit
			\begin{itemize}
				\item Zielgesteurt vs. Reizinduziert
			\end{itemize}

		\item Wahrnehmungsorganisation
			\begin{itemize}
				\item Figur, Grund, Scheinkonturen $\rightarrow$ 
					Gruppierungsphänomene
				\item Binokulares Tiefensehen durch Querdisparation und 
					Konvergenz
				\item Monokular durch Interposition/Okklusion, rel. Größe, 
					Texturgradient und Linearkonvergenz
			\end{itemize}

		\item Identifikation und Wiedererkennen
			\begin{itemize}
				\item Formkonstanz und bestimmte Komponenten wichtig
			\end{itemize}

	\end{itemize}

	\section{Motivation}

	\begin{itemize}
		\item Definition
			\begin{itemize}
				\item Zweck: Initiierung, Richtungsgebung, 
					Aufrechterhaltung
				\item Verbindung Biologie, Verhalten
				\item Erklärung von Verhaltensvariabilität
				\item Zuweisen von Verantwortung: Motivation + Fähigkeit 
					zur Kontrolle notwendig
				\item Erklärung von Beharrlichkeit trotz Widrigkeit
			\end{itemize}

		\item Quellen
			\begin{itemize}
				\item Triebe und Anreize: Reaktion auf physiologische 
					Bedürfnisse
				\item Bedürfnishierarche nach Maslow:\\
					Biologisch $\rightarrow$ Sicherheit $\rightarrow$ 
					Bindung $\rightarrow$ Wertschätzung $\rightarrow$ 
					Selbstverwirklichung
				\item Motive (Leistung, Macht, Anschluss)
				\item explizite vs. implizite Motive
			\end{itemize}

		\item Beispiel: Leistungsmotivation
			\begin{itemize}
				\item Genetische Faktoren vs. Motivation
				\item Implizite Motive messbar mit Thematischem 
					Apperzeptionstest
				\item Attribution von (Miss)Erfolg (Internal vs. external, 
					stabil vs. variabel)
			\end{itemize}


		\item Modelle und Theorien
			\begin{itemize}
				\item Erwartungsmodell
					\begin{itemize}
						\item Motiviert wenn erwartet wird, dass 
							Leistung zu gewünschtem Erfolg 
							führt
						\item Erwartung: erwartete 
							Erfolgswahrscheinlichkeit
						\item Valenz: wahrgenommene Wertigkeit des 
							Ausgangs
						\item Instrumentalität: Wahrnehmung der 
							Folgen der Leistung
					\end{itemize}
				\item Theorie der Selbstbestimmung
					\begin{itemize}
						\item Autonomie (Gefühl der 
							Selbstbestimmung)
						\item Kompetenz (sich erfolgreich fühlen)
						\item Soziale Einbindung (Wertschätzung)
						\item $\Rightarrow$ intrinsische Motivation
					\end{itemize}
				\item Motivationale Handlungkonflikte: Leistungswerte 
					(kompetenz) vs. Wohlbefindenswerte (Einbindung)
				\item Intrinsische vs. extrinsische Motivation
				\item Motivation vs. Volition
				\item Rubikonmodell:\\
					Abwägen $\rightarrow$ Planen $\rightarrow$ Handeln 
					$\rightarrow$ Bewerten
			\end{itemize}
	\end{itemize}

	\section{Intelligenz und Kognition}

	\begin{itemize}
		\item Definition und Modelle
			\begin{itemize}
				\item Allgemein: schlussfolgernd, abstrakt, planend, 
					problemlösend denken; Auffassen und aus Erfahrung 
					lernen
				\item Psychometrische Theorien
					\begin{itemize}
						\item Faktorenanalyse
						\item Spearman: g-Faktor (allgemein) vs. 
							s-Faktoren (spezifisch)
						\item Cartell: Kristalline (Wissen) vs. 
							Fluide (Verstehen) Intelligenz
						\item Sternberg: Analytische (abstrakt und 
							Verarbeitung) vs. Kreative (neues 
							Verarbeiten) vs. Praktische 
							(Anwenden) Intelligenz

					\end{itemize}
				\item Multiple Intelligenz nach Gardner: Laser vs. 
					Searchlight
				\item Emotionale Intelligenz: wahrnehmen, einsetzen, 
					verstehen und regulieren von Emotionen
			\end{itemize}

		\item Intelligenzdiagnostik
			\begin{itemize}
				\item Im Bezug zu vergleichbarer Personengruppe
				\item Binet: errechnet objektiven Durchschnittswert für 
					Altersgruppen und vergleicht damit 
					Einzelleistungen $\Rightarrow$ Intelligenzalter
				\item Stanford-Binet-Skala: \\
					IQ = Int.Alter / Lebensalter * 100
				\item Wechsler Intelligenzskalen
					\begin{itemize}
						\item Gesamt vs. Verbaler vs. Handlungs 
							IQ
						\item Verbal, Wahrnehmung, 
							Arbeitsgedächtnis, 
							Verarbeitungsgeschwindigkeit
					\end{itemize}
				\item Cultural Fair Intelligence Tests von Cartell
			\end{itemize}


		\item Verteilung der Intelligenz
			\begin{itemize}
				\item 100 entspricht Durchschnitt, 85-115 durschnittlich
				\item Hochbegabung: Überdurchschn. Fähigkeiten + 
					Kreativität + Zielstrebigkeit
			\end{itemize}

		\item Erblichkeit
			\begin{itemize}
				\item Substantielle genetische Komponente
				\item Flynn-Effekt: Bevölkerung wird immer besser
			\end{itemize}
	\end{itemize}
	\section{Entwicklung}

	\begin{itemize}
		\item Methoden
			\begin{itemize}
				\item Querschnitt (gleiche Testzeit) vs. Längsschnitt 
					(gleicher Jahrgang) vs. Zeitwandelstudie
					(gleiches Alter)
				\item Probleme: Kohorteneffekte (Alter konfundiert mit 
					Bedingungen), Flynn Effekt usw.
			\end{itemize}

		\item Entwicklungsaufgaben
			\begin{itemize}
				\item Aufgabe die bestimmten Lebensabschnitt zugeordnet 
					ist
				\item Erfolg bringt Glück, Misserfolg missbilligung
				\item zeitlich begrenzt vs. über mehrere Lebensabschnitte
				\item Säugling: gehen, Essen, emot. Bindung zu Eltern
				\item Mittlere Kindheit: Geschick, mit Artgenossen 
					zurechtkommen usw.
			\end{itemize}

		\item Biologische Determinanten
			\begin{itemize}
				\item Anlage vs. Umwelt
				\item Hirnentwicklung: Überangebot an Synapsen bis 2 Jahre 
				\item Aufs Überleben \glqq{}programmiert\grqq{}: Klippe, 
					Klammerreflex, Bindungsverhalten
			\end{itemize}

		\item Kognitivie Entwicklung
			\begin{itemize}
				\item Akkomodation (Umbau) vs. Assimilation (Einfügen)
				\item Objektpermanenz (Obj. unabhängig von Wahrnehmung)
				\item Egozentrismus (keine andere Persp. einnehmbar)
				\item Zentrierung (auf nur einen Wahrnehmungsinhalt)
				\item Invarianzprinzip (Eigenschaften persistent unter 
					Änderung der Form unabhängig von Menge)
			\end{itemize}

		\item Identität
			\begin{itemize}
				\item Entwicklung durch Krisen
				\item Verpflichtung: Erarbeitet vs. 
					Übernommen
				\item Keine: Moratorium vs. 
					Identitätsdiffusion
			\end{itemize}
		
		\item Bindung
			\begin{itemize}
				\item Bowlby
					\begin{itemize}
						\item Bindung: Nähe und Schutz suchen, 
							aktiviert Fürsorge
						\item Exploration: Erkundungsverhalten in 
							sicheren Situationen
						\item Fürsorgeverhaltenssystem
					\end{itemize}
				\item Bindungsmuster nach Ainsworth
					\begin{itemize}
						\item Sichere Bindung: Balance zwischen 
							Bindung und Exploration
						\item Unsicher-vermeidende: mehr 
							Exploration, Furcht vor 
							Zurückweisung
						\item Unsicher-ambivalent: mehr Bindung, 
							Angst vor Trennung
						\item $\Rightarrow$ Strange Situation Test
					\end{itemize}
			\end{itemize}

		\item Moralentwicklung
			\begin{itemize}
				\item Präkonventionelle vs. konventionelle vs. 
					postkonventionelle Moral
			\end{itemize}
	\end{itemize}
	\section{Lernen und Verhaltensanalyse}

	\begin{itemize}
		\item Lernen, Behaviorismus und Verhaltensanalyse
			\begin{itemize}
				\item erfahrungsbasiert, überdauernd, Verhaltensänderung
				\item Leistung: Verhaltensausdruck von Gelerntem
				\item Reifung als Voraussetzung
				\item Kompetenz vs. Performanz
				\item Intentional vs. inzidentielles Lernen
			\end{itemize}

		\item Klassisches Konditionieren
			\begin{itemize}
				\item UCS kombiniert mit NS löst UCR aus
				\item Nach Konditionierung führt CS zu CR
				\item Löschung durch CS ohne UCS
				\item Spontanremission nach Pause
				\item Verzögerte (direkt) vs. Spuren (kurzer Abstand) vs. 
					Simultane vs. Rückwärtskonditionierung
				\item Reizgeneralisierung (ähnlich zum CS) vs. 
					Reizdiskriminierung (unterscheiden)
				\item Kontingenz (Vorhersagekraft) entscheidend
				\item Kamins Blockierungseffekt: zweiter CS hat keine 
					Vorhersagekraft wenn einzeln hinzugefügt
				\item Kleiner Albert: Generalisierte Konditionierung von 
					Angst
			\end{itemize}

		\item Operantes Konditionieren
			\begin{itemize}
				\item Thorndikes Puzzleboxen
				\item Gesetz des Effekts: Belohnung verstärkt, Bestrafung 
					schwächt
				\item Skinnerboxen für Ratten
				\item Konsequenzen
					\begin{itemize}
						\item Angenehm: pos. Verstärkung (+) vs. 
							Negative Bestrafung (-)
						\item Unangenehm: Positive Bestrafung (+) 
							vs. Negative Verstärkung (-)
					\end{itemize}
				\item Dreifachkontingenz: Reiz, Reaktion und Konsequenz
				\item Primäre (biologisch) vs. sekundäre (erlernte 
					Verknüpfung) Verstärker
				\item Verstärkungspläne
					\begin{itemize}
						\item Kontinuierlich: schnelles Erlernen
						\item Intermittierend: dauert länger aber 
							löschungsresistenter	
					\end{itemize}
				\item Quotenpläne
					\begin{itemize}
						\item FR: jede x-te Reaktion
						\item VR: Im Mittel jede x-te
						\item FI: nur alle x Sekunden
						\item VI: Im Mittel alle x Sek.
					\end{itemize}
			\end{itemize}

		\item Einschränkungen, Erweiterungen, Anwendungen
			\begin{itemize}
				\item Instinktverschiebung $\Rightarrow$ in genetisches 
					Verhaltensrepertoire integrieren
				\item Kognitive Landkarten
				\item Belongingness
			\end{itemize}

		\item Beobachtungslernen
			\begin{itemize}
				\item Sozial-kognitive Lerntheorie
					\begin{itemize}
						\item Aneignung (Aufmerksamkeit und 
							Gedächtnis)
						\item Ausführung (Motorisch und 
							Motivational)
					\end{itemize}
				\item Bandura: Bobo Doll: Reproduktion hängt ab von 
					Konsequenzen (kompetenz vs. performanz)
			\end{itemize}

	\end{itemize}
	\section{Gedächtnis}

	\begin{itemize}
		\item Definition und Grundbegriffe
			\begin{itemize}
				\item Enkodierung, Speicherung, Abruf
				\item Gedächtnis als Voraussetzung fürs Lernen
			\end{itemize}

		\item Gedächtnisformen
			\begin{itemize}
				\item Explizites / Deklaratives Gedächtnis
					\begin{itemize}
						\item semantisch: Fakten
						\item episodisch: Ereignisse
					\end{itemize}
				\item Nondeklaratives /implizites Gedächtnis
					\begin{itemize}
						\item Prozedurale Fähigkeiten
						\item Priming, Konditionierung
					\end{itemize}
				\item Assoziatives Lernen (klassische /operante 
					Konditionierung)
				\item Nicht assoziatives: (Dis)Habituation (Gewöhnung), 
					Sensitivierung
			\end{itemize}

		\item Gedächtnisprozesse
			\begin{itemize}
				\item Enkodierung von Sensory in STM, Storage in LTM
				\item Sensorisch: Ikonisch (visuell) und Echoisch 
					(auditiv)
				\item Kurzzeit- vs. Arbeitsgedächtnis
				\item Kapazität STM: $7\pm2$
				\item Verbesserte Enkodierung durch Rehearsel und Chunking
				\item Arbeitsgedächtnis
					\begin{itemize}
						\item Zentrale Exekutive
						\item Phonologische Schleife
						\item Visuell-räumlicher Notizblock
					\end{itemize}
				\item Enkodierspezifität
				\item Erinnern: passiv (recognition) vs. aktiv (recall)
			\end{itemize}

		\item Strukturen des Langzeitgedächtnisses
			\begin{itemize}
				\item Konzept vs. Prototyp
				\item Rekonstruktiver Prozess: Vereinfachung, 
					Akzentuierung, Assimilation
				\item Loci-, Mind-Map-Methode
			\end{itemize}

	\end{itemize}
	\section{Persönlichkeits- und Sozialpsychologie}

	\begin{itemize}
		\item Eigenschaftsbasierte Persönlichkeitstheorien
			\begin{itemize}
				\item  Anhand von traits als Prädisposition $\rightarrow$ 
					verursachen Verhalten (intervenierende Variable)
				\item Allport: Kardinale, zentrale und sekundäre Traits 
				\item Catell: 16 \glqq{}Source Traits\grqq{}, darunter 
					Gegensatzpaare wie zurückhaltend - offen usw.
				\item Eysenck: 3 breite Dimensionen
					\begin{itemize}
						\item Introversion vs. Extraversion
						\item Emot. Stabilität vs. Neurotizismus
						\item Anpassung vs. Psychotizismus
					\end{itemize}
				\item McCrae und Costa: Fünf-Faktoren-Modell
					\begin{itemize}
						\item Extraversion
						\item Verträglichkeit
						\item Gewissenhaftigkeit
						\item Neurotizismus
						\item Offenheit für Erfahrungen
					\end{itemize}
				\item Stabilität genetisch, Veränderung umweltbedingt
				\item Veränderung? Entitätstheorie vs. Wachstumstheorie
			\end{itemize}

		\item Soziale Lerntheorien und kognitive Theorien
			\begin{itemize}
				\item Mischel: Kognitiv-affektive Persönlichkeitstheorie: 
					Reaktion auf Reize hängt von 5 Faktoren ab:
					\begin{enumerate}
						\item Enkodierung der Reize
						\item Erwartung über Ausgänge
						\item Affekte, Emotionen
						\item Ziele und Werte
						\item Kompetenzen und regulatorische 
							Fähigkeiten
					\end{enumerate}
				\item Bandura: Sozial-kognitive Lerntheorie
					\begin{itemize}
						\item Beobachtungslernen
						\item Reziproker Determinismus (Person, 
							Verhalten und Umwelt)
						\item Selbstwirksamkeit
					\end{itemize}

			\end{itemize}

		\item Theorien des Selbst
			\begin{itemize}
				\item Selbstkonzept: Erinnerungen, Annahmen über Motive, 
					Ideal-Selsbt usw.
				\item Selbstwertgefühl (generalisierte Bewertung)
				\item Selbstbeeinträchtigendes Verhalten (schützt 
					Selbstwertgefühl)
				\item Kulturelle Konstruktion des Selbst
					\begin{itemize}
						\item Individualistisch: independent
						\item Kollektivistisch: interdependent
					\end{itemize}
			\end{itemize}

		\item Möglichkeiten der Testung von Persönlichkeit
			\begin{itemize}
				\item Objektive Tests
					\begin{itemize}
						\item MMPI: Antwort auf viele Aussagen 
							$\Rightarrow$ 10 klinische Skalen
						\item NEO-PI: Misst 5 Faktorenmodell

					\end{itemize}
				\item Projektiv: Thematic Apperception Test, Rohrschach
			\end{itemize}



	\end{itemize}
	\section{Emotion, Stress und Gesundheit}

	\begin{itemize}
		\item Emotionen
			\begin{itemize}
				\item Emotion (kurz, Reaktion) vs. Stimmung (lang, 
					unabhängig)
				\item Emotionale Reaktion angeboren
				\item Ekman: Emotionaler Ausdruck universell: \\
					Freude, Überraschung, Ärger, Ekel, Furcht, Trauer, 
					Verachtung
				\item Physiologie: Autonomes NS: (Para) Sympathikus
				\item Amygdala als Kontrollinstanz
				\item James-Lange-Theorie
					\begin{itemize}
						\item Emotion nach körperlicher Reaktion
						\item Peripheriebetonend
					\end{itemize}	
				\item Cannon-Bard-Theorie
					\begin{itemize}
						\item Hirnaktivität aktiviert ANS und 
							emotionales Erleben
						\item Fokus auf ZNS
					\end{itemize}
				\item Theorie der kognitiven Bewertung
					\begin{itemize}
						\item Kognitive Bewertung und körperliche 
							Reaktion unabhängig
						\item Daraus Attribution der Emotion
					\end{itemize}
			\end{itemize}

		\item Stress
			\begin{itemize}
				\item Stressor: Ereignis, das Anpassungsreaktion erfordert 
				\item Distress vs. Eustress
				\item Akut (Anfang und Ende) vs. chronisch
				\item Notfallreaktion: fight or flight
				\item Hypothalamus als Zentrum: steuert ANS und aktiviert 
					Hypophyse
				\item (Nor)Adrenalin und Cortisol als Stresshormone
				\item Adaptation
					\begin{itemize}
						\item Alarmreaktion (Bereitstellung von 
							Reserven)
						\item Widerstand (moderate Erregung)
						\item Erschöpfung
					\end{itemize}
				\item Psychosomatische Störungen
				\item Stresscoping
					\begin{itemize}
						\item Problemorientiert: Verändere / Löse 
							Stressor
						\item Emotionsorientiert: Verändere dich 
							selbst
					\end{itemize}
			\end{itemize}

		\item Gesundheit
			\begin{itemize}
				\item Physisches, psychisches und soziales Wohlbefinden
				\item Gesundheitspsychologie: Untersuchung, 
					Gesundheitsförderung, Prävention, 
					Gesundheitspolitik
				\item bio-psychosoziales Modell beschreibt Wechselwirkungen
				\item Systemtheorie
				\item Gesundheitsverhalten: Erkrankungen vorbeugen / 
					entdecken
				\item Gesundheitsverhaltensmodelle
					\begin{itemize}
						\item Theorie des geplanten Verhaltens: 
							Einstellung, wahrgenommene 
							Kontrolle, Norm beeinflussen 
							Absicht und Verhalten
						\item Transtheoretisches Modell: Stufen
					\end{itemize}
			\end{itemize}
	\end{itemize}

	\section{Psychische Störungen}

	\begin{itemize}
		\item Begriffsklärung
			\begin{itemize}
				\item Abweichung von statistischer, sozialer, idealer oder 
					funktioneller Norm
				\item Leitfaden
					\begin{itemize}
						\item Definition und Formen
						\item Diagnose
						\item Epidemiologie, Komorbidität und 
							Verlauf
						\item Erklärungsmodelle
						\item Prävention und Behandlung
					\end{itemize}
			\end{itemize}

		\item Klassifikation
			\begin{itemize}
				\item DSM IV (fünf Achsen)
				\item ICD 10 
			\end{itemize}

		\item Angststörungen
			\begin{itemize}
				\item Dauererregungszustand
				\item Phobische Störungen
					\begin{itemize}
						\item Agoraphobie (komorbid mit 
							Panikstörung)
						\item Soziale Phobie
					\end{itemize}
				\item Panikstörung
				\item Generalisierte Angststörung (mindestens 6 Monate, 
					verschiedene körperliche und psychische Symptome)
			\end{itemize}

		\item Affektive Störungen
			\begin{itemize}
				\item Depression
					\begin{itemize}
						\item mindestens 2 Wochen
						\item Interessenverlust, Antriebsmangel
						\item Einstufung in leicht, mittel schwer
						\item Beeinflusste Empirie
					\end{itemize}
				\item Bipolare Störung
					\begin{itemize}
						\item Abwechselnd depressive und manische 
							Phasen
						\item Zyklothymia
					\end{itemize}
			\end{itemize}

		\item Persönlichkeitsstörungen
			\begin{itemize}
				\item Unterteilung exzentrisch, dramatisch und ängstlich 
			\end{itemize}


	\end{itemize}
	\section{Psychotherapie}

	\begin{itemize}
		\item Erklärungsmodelle: Psychodynamisch, Behavioral, Kognitiv, Biologisch

		\item Psychoanalyse
			\begin{itemize}
				\item Unbewusstes im Vordergrund
				\item Folge von unbewussten Konflikten und Erlebnissen in 
					Kindheit
				\item Therapeut hilft diese sichtbar zu machen 
					$\Rightarrow$ neutrales Blatt
				\item Langzeittherapie, 2-3 mal pro Woche
				\item Freie Assoziation
			\end{itemize}

		\item Tiefenpsychologie
			\begin{itemize}
				\item Aktuelle psychische Konflikte statt unterbewusste 
					Prozesse
				\item Therapeut unterstützt Patient bei Aktivierung 
					unbewusster Fähigkeiten
				\item Sitzen gegenüber, längerfristig, 1 mal 
					wöchentlich
			\end{itemize}

		\item Verhaltenstherapie
			\begin{itemize}
				\item Belastende Denk- und Verhaltensmuster verlernen und 
					stattdessen neue Erlernen
				\item Hilfe zur Selbsthilfe
				\item Hausaufgaben, transparente Therapieziele
				\item gegenüber, 1 mal wöchentlich, 25-80 mal
				\item Gegenkonditionierung
					\begin{itemize}
						\item Expositionstherapie (Hierarchisch 
							vs. Flooding)
						\item Entspannung
						\item Nachahmung von Modellen
						\item Aversionstherapie
					\end{itemize}
				\item Kontingenzmanagement
					\begin{itemize}
						\item Positive Verstärkung (Token)
						\item Löschungsstrategien
					\end{itemize}
				\item Kognitive Ansätze
			\end{itemize}

		\item Medikamentöse Therapie
			\begin{itemize}	
				\item Noradrenalin und Serotonin Wiederaufnahmehemmer
			\end{itemize}

		\item Wirkfaktoren
			\begin{itemize}
				\item Positive Erwartungen
				\item Vertrauen, Wärme, Akzeptanz
			\end{itemize}

	\end{itemize}

\end{document}
