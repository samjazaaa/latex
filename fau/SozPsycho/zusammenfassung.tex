\documentclass[11pt, paper=a4, twocolumn]{scrartcl}

\usepackage[ngerman]{babel}
\usepackage[utf8]{inputenc}

\usepackage[T1]{fontenc}
\usepackage{mathpazo}

\usepackage{geometry}

\geometry{a4paper, top=20mm, left=15mm, right=15mm, bottom=20mm,
headsep=5mm, footskip=12mm}


\pagenumbering{gobble}

\title{\vspace{-1.25cm}Zusammenfassung Sozialpsychologie\vspace{-0.25cm}}
\date{\vspace{-5ex}}

\newcommand*{\Z}{\mathbb{Z}}

\begin{document}
	\maketitle


	\section{Hintergrund}
		\begin{itemize}
			\item Definition
				\begin{itemize}
					\item Denken, Fühlen und Handeln
					\item Beeinflusst durch die tatsächliche, vorgestellte oder implizierte Gegenwart anderer
				\end{itemize}
			\item Personen- vs. Situationsbezogen
			\item Subjektive vs. objektive Wahrnehmung
		\end{itemize}


	\section{Methoden}
		\begin{itemize}
			\item Empirische Wissenschaft
				\begin{itemize}
					\item Verstehen und Vorhersagen $\Rightarrow$ Interventionen
					\item Theorien: komplexe erklärende Aussagen über Realitätsausschnitt
					\item Hypothesen
						\begin{itemize}
							\item Aus Theorien abgeleitet
							\item falsifizierbar
							\item Operationalisierung / Quantifizierung
							\item Hypothesentestende vs. explorative Forschung
						\end{itemize}
				\end{itemize}

			\item Deskriptive Forschung
				\begin{itemize}
					\item Wie ist das Phänomen?
					\item (Un-)Standardisierte Beobachtung
					\item Dokumentenanalyse
					\item Befragungen
					\item Einschränkungen: schwer / nur in bestimmten Kontexten beobachtbar, kein warum
				\end{itemize}

			\item Korrelative Forschung
				\begin{itemize}
					\item Korellationen (pos., neg., keine)
					\item Keine Kausation $\Rightarrow$ deskriptiv
				\end{itemize}

			\item Experimentelle Forschung
				\begin{itemize}
					\item UV manipulieren und AV messen
					\item Standardisierung und Randomisierung
					\item Design
						\begin{itemize}
							\item $\#UV1 * \#UV2 * \dots$
							\item Mediator als Zwischenvariable / Folge der UV
							\item Moderator beeinflusst Stärke des Zusammenhangs
							\item Haupteffekt vs. Interaktionseffekt
						\end{itemize}
					\item Gütekriterien
						\begin{itemize}
							\item Reliabilität (Zuverlässigkeit)
							\item Validität (das richtige gemessen?)
							\item Konfundierung von Vars. (intern)? Generalisierbarkeit (extern)?
						\end{itemize}
					\item Coverstory, Konföderierte, Manipulationscheck
					\item Grenzen: VL/VP-Effekte, Ethik
				\end{itemize}

			\item Qualitative Forschung
				\begin{itemize}
					\item Meist explorativ
					\item Induktiv Theorien entwickeln
					\item Diskursanalyse
					\item Verkuyten (2005): Triangulation; Diskurs $\rightarrow$ Text, dann Text mit (keine) Wahl $\rightarrow$ 
						Unterstützung von Multikult.
				\end{itemize}

			\item Replikationskrise
		\end{itemize}


	\section{Soziale Kognition}
		\begin{itemize}
			\item Grundsätzliches
				\begin{itemize}
					\item Denken im sozialen Kontext
					\item Fokus auf subjektives Erleben und (systematische) Fehler
				\end{itemize}
			\item Prozessstufen der soz. Informationsverarbeitung
				\begin{itemize}
					\item Informationsinput
					\item Wahrnehmung
					\item Aufmerksamkeit (nach Salienz, Relevanz, aktiviertem Wissen)
					\item Kategorisierung (Zuordnung Kategorie / Schema, Anreicherung mit Wissen)
					\item Integration (zusammenfassen und mit gespeichertem Material integrieren)
					\item Urteil, Verhalten
					\item Selektion $\Rightarrow$ Inferenz
					\item Bottom-up (oben) vs. Top-down
				\end{itemize}

			\item Automatische / Kontrollierte Verarbeitung
				\begin{itemize}
					\item Kontrolliert: absichtlich, bewusst, aufwändig $\rightarrow$ gezielt und selten
					\item Automatisch: unbewusst, mühelos $\rightarrow$ ständig und unwillkürlich
					\item Einfluss wenn: Motivation, Ressourcen, Verzerrungen bewusst
					\item Kontinuummodell (Fiske): autom. Kat. $\rightarrow$ Neu/sub $\rightarrow$ Individualisierte Wahrn.
				\end{itemize}

			\item Kognitive Schemata
				\begin{itemize}
					\item Teil der Kategorisierung
					\item Wissen über soziale Welt nach Themenbereichen
					\item Skripte (Restaurant), Selbstbild, soz. Rollen, Stereotype
					\item Funktionen
						\begin{itemize}
							\item Organisation und Sinnverleihung
							\item Mustererkennung
							\item Erinnerungshilfe
						\end{itemize}
					\item Auswirkungen
						\begin{itemize}
							\item Wahrnehmung (Gorilla im Bild passt nicht in Schema)
							\item Erinnerung (schemakonsistente Infos, siehe Cohen: Bibliothekarin vs. Kellnerin)
							\item Urteilsbildung (Zugänglichkeit des Schemas wichtig)\\
								recency \& frequency, situative vs. chronische Verfügbarkeit
						\end{itemize}
					\item Priming
						\begin{itemize}
							\item Beiläufige Voraktivierung eines vorhandenen Schemas
							\item Einfluss auf weitere Verarbeitung
							\item subliminal vs. supraliminal
							\item affektiv / semantisch / behavioral
						\end{itemize}
					\item Donald-Studie von Higgins, Rholes \& Jones 1977
						\begin{itemize}
							\item Wahrnehmungsexp. mit eingebetteten Eigenschaftsbegriffen (pos/neg)
							\item $\Rightarrow$ Beschreibung zu ambiv. Donald beurteilen (pos/neg)
							\item UVs: 2 (pos/neg) $\times$ 2 (passend ja/nein)
							\item AV: Bewertung von Donald
							\item $\Rightarrow$ Anwendbar: entspricht Bewertung; ansonsten nicht
						\end{itemize}
				\end{itemize}

			\item Persohnenwahrnehmung
				\begin{itemize}
					\item Erster Eindruck (snap) $\rightarrow$ Verhaltensbeobachtung (Implikationen) $\rightarrow$ Erwartungen
					\item Snap Judgements
						\begin{itemize}
							\item Physische Erscheinung, (para/non)verbale Merkmale
							\item Schnell mit minimaler Information
							\item Korreliert stark mit deliberativen Urteilen (vertrauenswürdige Gesichter)
							\item Hohe intersubjektive Übereinstimmung (Clips von Vorlesungen vs. Eval.)
						\end{itemize}
					\item Zentrale Dimensionen
						\begin{itemize}
							\item Asch 1946: Eigenschaften + warmherzig vs. kaltherzig $\Rightarrow$ pos./neg.
							\item Communion
								\begin{itemize}
									\item Beziehungen / Sympathie
									\item getting along
									\item prim. in Wahrn. anderer
								\end{itemize}
							\item Agency
								\begin{itemize}
									\item Respekt / Ziele erreichen
									\item getting ahead
									\item prim. für Vorhersage von pos. Selbstwert
								\end{itemize}
						\end{itemize}
					\item Verzerrungen
						\begin{itemize}
							\item Primacy/Recency-Effekte
							\item Framing (Reihenfolge, pos./neg., Vergleichsrichtung)
							\item Pluralistische Ignoranz (kollektive Fehlinterpretation durch geteilte falsche Annahme über 
								Verhaltensgründe)\\
								$\rightarrow$ Kontakt zu anderen ethnischen Gruppen
							\item Selbst erfüllende Prophezeihungen\\
								Pygmalion-Effekt (Rosenthal \& Jacobson): Schüler blühen auf
							\item Perseveranzeffekt (Überzeugungen bleiben auch ohne Grund; Ross: falsches feedback)
							\item Confirmatory Bias (suche bestätigende Infos; Snyder: introvertiert)
						\end{itemize}
					\item Urteilsheuristiken
						\begin{itemize}
							\item  Abkürzung mit geringem kogn. Aufwand für komplexe Entscheidungen
							\item Besser als Zufall aber systemathische Verzerrungen
							\item Beispiel: Experten / Nachahmungsheuristik
							\item Ankerheuristik: relativ zu Rahmen
							\item Verfügbarkeitsheuristik: je mehr Infos desto wahrscheinlicher
							\item Repräsentativitätsheuristik: wahrgenommene Ähnlichkeit
							\item Pseudodiagnostizität: Ereignis als Beleg für Vermutung trotz Alternativerklärung\\
								(self-fulf., conf. bias)
						\end{itemize}
				\end{itemize}

			\item Attribution
				\begin{itemize}
					\item Zuschreibung von Gründen zu beobachtetem Verhalten
					\item Handeln = (Anstrengun $\times$ Fähigkeit) + (Schwierigkeit + Zufall)\\
						$\Rightarrow$ intern vs. extern
					\item Kausalattribution: frei entschieden? was machen andere?
					\item Attributionsarten nach Seligman
						\begin{itemize}
							\item personal / internal vs. situational / external
							\item variabel vs. stabil
							\item global vs. spezifisch
							\item kontrollierbar vs. unkontrollierbar
						\end{itemize}
					\item Wann? $\rightarrow$ unerwartet, negativ, selbst-relevant
					\item Warum? $\rightarrow$ strukturierend, Vorhersagen
					\item Kovariationsprinzip (Kelley 1967)
						\begin{itemize}
							\item Konsensus: was machen andere?
							\item Distinktheit: was bei ähnlichen Stimuli?
							\item Konsistenz: nur einmal oder häufiger?
						\end{itemize}
					\item Optimistischer Attributionsstil
						\begin{itemize}
							\item Erfolg: personal, stabil, global
							\item Misserfolg: situational, variabel, spezifisch
						\end{itemize}
					\item Verzerrungen
						\begin{itemize}
							\item Fundamentaler Attributionsfehler
								\begin{itemize}
									\item Tendenz zu personaler Attribution
									\item Jones \& Harris: (un)freiwillig für Castro argumentiert
								\end{itemize}
							\item Perzeptuelle Salienz
								\begin{itemize}
									\item Taylor \& Fiske: Gespräch zwischen zwei Pers. aus verschiedenen Pos.
									\item Pers. in Blickfeld wirkt wichtiger
								\end{itemize}
							\item Gilbert: Zwei-Schritte-Proz.: automatische vs. aufwändigere Attribution
							\item Kulturell
								\begin{itemize}
									\item FAF vor Allem in westlich individualistischen Kulturen
									\item Masuda et al.: USA vs. Japan
									\item UV: Personen im Hintergrund
									\item AV: Emotion der Hauptperson
									\item kollektivistisch $\Rightarrow$ mehr Fokus auf Situation
								\end{itemize}
							\item Akteur-Beobachter Divergenz: andere personal, selbst situational attribuieren
							\item Selbstwertdienliche (optimistische) Attribution
						\end{itemize}
					\item Alternative Ansätze
						\begin{itemize}
							\item Kontrafaktisches Denken (Kahneman)
								\begin{itemize}
									\item Simulation von Alternativen (was wenn?)
									\item bei unerwartet, negativ, selbstrelevant
									\item Leicht veränderbare: untypisch, leicht verfügbar, kontrolliert
									\item Emot. Reaktion stärker je verfügbarer Alternative
								\end{itemize}
							\item Konversationsmodell
								\begin{itemize}
									\item Erklärung eingebettet in Konversation
									\item Konversationsregeln
									\item Sender, Empfänger, Kontext wichtig
								\end{itemize}
							\item Discursive Action Model of Attribution: Ziele / Absichten hinter Erklärungen
						\end{itemize}
				\end{itemize}
		\end{itemize}


	\section{Das Selbst}
		\begin{itemize}
			\item Selbst als Subjekt (I)
				\begin{itemize}
					\item Ausführende Funktion / handlendes Selbst
					\item Selbstregulation $\rightarrow$ bewusste Verhaltenssteuerung
				\end{itemize}
			\item Selbst als Objekt (me)
				\begin{itemize}
					\item Strukturierend / Wissen über eigene Person $\rightarrow$ Selbstkonzept
					\item Gefühle für Selbst (Selbstwertgefühl)
				\end{itemize}
			\item Selbst als Narrativ: Geschichte über Selbst (subjektiv, selektiv, dynamisch)
			\item Selbstregulation
				\begin{itemize}
					\item Gedankenunterdrückung nur vorübergehend
					\item Überwachung des Verhaltens durch Ist-Soll-Vergleich
					\item Begrenzte Ressourcen (ego-depletion, Übung)
					\item Erleichtert durch obj. Selbstaufmerksamkeit
				\end{itemize}
			\item Objektive Selbstaufmerksamkeit
				\begin{itemize}
					\item Aufmerksamkeit auf eigene Person
					\item Vergleich Verhalten $\leftrightarrow$ Werte und Normen\\
						$\rightarrow$ Anpassung oder Flucht
					\item Positiv: pos. Emot. nach Erfolg / Normkonformität
					\item Negativ: neg. Emot., schädlcihe Flucht
				\end{itemize}
			\item Selbstkonzept
				\begin{itemize}
					\item Organisation in Selbst-Schemata (schnell; chronisch / situativ verfügbar)
					\item Arbeitsselbstkonzept (aktuell verfügbar)
					\item Selbstreferenzeffekt: Infos mit Zusammenhang zu Selbst werden schneller / besser verarbeitet
						$\rightarrow$ Selbst als Vergleichsstandard
					\item Tatsächliches vs. ideales vs. erwünschtes Selbst
				\end{itemize}
			\item Selbstwertgefühl
				\begin{itemize}
					\item Gesamtbewertung als pos. / neg.
					\item Trait vs. self-esteem
					\item explizit vs. implizit
				\end{itemize}
			\item Motivationale Funktionen des Selbst
				\begin{itemize}
					\item Selbstaufwertungsmotiv
						\begin{itemize}
							\item Illusionen (unreal. Optimismus, better-than-average)
							\item Informationsverarbeitung (false consensus, conf. bias, Attributionen, impl. Egoismus)
							\item Abhängig von Kontext, Geschlecht usw.
						\end{itemize}
					\item Selbsteinschätzungsmotiv: möglichst treffendes Selbstbild (v.A. für Stärken)
					\item Selbstbestätigungsmotiv
						\begin{itemize}
							\item Annahmen über Selbst sollen bestätigt werden (konsistenz)
							\item Swann: Vpn wählen pos. / neg. Gesprächspartner je nach Selbstbild
						\end{itemize}
				\end{itemize}
			\item Selbstdarstellungsstrategien
				\begin{itemize}
					\item Self-promotion (Kompetenz), Einschmeichlung (Sympathie), gutes Vorbild (Moral)
					\item BIRG (basking in reflected glory)
					\item Self-handicapping $\Rightarrow$ ext. Attribution
				\end{itemize}
			\item Self-affirmation
				\begin{itemize}
					\item Selbstwert stärken als Puffer für folgende Bedrohung
					\item Puffer gegen self-handicapping (Siegel: Test mit (in)kongruentem Ergebnis $\rightarrow$ ablenkende Musik?)
				\end{itemize}
			\item Quellen der Selbsterkenntnis
				\begin{itemize}
					\item Selbstreflexion / Introspektion
						\begin{itemize}
							\item Relativ selten
							\item Gute Einsicht WAS wir denken aber nicht WARUM $\rightarrow$ Kausaltheorien
						\end{itemize}
					\item 2-Faktoren Theorie (Schachter \& Singer)
						\begin{itemize}
							\item Selbstbeobachtung und Attribution
							\item Wahrnehmung physiolog. Erregung $\Rightarrow$ Interpretation nach Situation
							\item Fehlattribution möglich (Dutton \& Aron: Hängebrücke)
						\end{itemize}
					\item Selbstwahrnehmungstheorie (Bem)
						\begin{itemize}
							\item Interpretation des eigenen Verhaltens wie bei Anderen 
							\item Wesentlich für Selbst-Attribution: Handlungsfreiheit, äußere Rechtfertigung
						\end{itemize}
					\item Andere als Quelle der Selbsterkenntnis
						\begin{itemize}
							\item Vergleiche
								\begin{itemize}
									\item Theorie des soz. Vergleichs (Festinger)\\
										$\rightarrow$ Selbsteinschätzung durch Vergleich mit ähnlichen
									\item Vor allem wenn kein obj. Maßstab vorh.
									\item Selbsterkenntnis vs. Selbstaufwertung vs. Selbstbestätigung als Grund
									\item Aufwärts: motivierend aber Selbstwert
									\item Abwärts: umgekehrt
								\end{itemize}
							\item Soziale Identität
								\begin{itemize}
									\item Soz. Gruppen Bestandteil des Selbstkonzepts
									\item Bedürfnis nach pos. soz. Ident. $\rightarrow$ Eigengruppen aufwerten
									\item Soz. Identität je nach Arbeitskonzept
								\end{itemize}
						\end{itemize}
				\end{itemize}
			\item Effekte von Anreizen
				\begin{itemize}
					\item Verhaltenssteuernd (Gewohnheit, Erfahrung, Selbstwahrnehmung)
					\item Reaktanz
					\item Over-Justification Effect (Greene et al.: Spiel $\rightarrow$ Belohnung $\rightarrow$ danach weniger)
					\item Nur wen Mot. ursprünglich hoch
					\item Abhängig von Aufgabe / Leistung und Ankündigung
				\end{itemize}
			\item Kultur und Selbstkonstruktion
				\begin{itemize}
					\item Independente Selbstsicht (Unabhängigkeit und Einzigartigkeit)
					\item Interdependente Selbstsicht (Verbundenheit und wechsels. Verpflichtung)
					\item Markus et al.: Olympiabericht
					\item Selbstaufwertung v.a. in individual. Kulturen
					\item Kitayama et al.: jap. und US Vpn beschreiben steigen / fallen von Selbstwert
						$\rightarrow$ Auswahl von relevanten Geschichten (US steigernd, jap. kritisch)
				\end{itemize}

				
		\end{itemize}

	\section{Einstellungen}
		\begin{itemize}
			\item Definition
				\begin{itemize}
					\item Psychologische Tendenz einer bestimmten Valenz und Stärke
					\item explizit vs. implizit
					\item Multikomponentenmodell (Zanna \& Rempel)
						\begin{itemize}
							\item Kognitiv (Überzeugungen, Gedanken, Eigenschaften)
							\item Affektive (Gefühle)
							\item Verhaltensbezogene (Verhalten gg. Gegenstand)
						\end{itemize}
					\item Emotionsbasierte vs. kongitionsbasierte Einst.
				\end{itemize}
			\item Einstellungsbildung
				\begin{itemize}
					\item Kognitionsbasiert (Info / Nachdenken)
					\item Emotionsbasiert (Grundeinstellung / Wertesyst., Erfahrungen, Eval. Konditionierung, mere exposure)
					\item Verhaltensbasiert (Rückschlüsse aus Verhalten $\rightarrow$ Selbstwahrnehmungstheorie)
					\item Affektives Priming (subliminal)
					\item Zajonc: mere exposure mit chin. Schriftzeichen
				\end{itemize}
			\item Funktionen
				\begin{itemize}
					\item Einschätzungsfunktion (schnellere Entscheidung, Verhaltensvorhersage)
					\item Utilitaristische Funktion (Kosten-Nutzen)
					\item Soziale Anpassung (Identifikation und Distanzierung)
					\item Selbstwert-Verteidigung
					\item Werteausdrucksfunktion
				\end{itemize}
			\item Einstellungsmessung
				\begin{itemize}
					\item Explizit
						\begin{itemize}
							\item Aufrufbar, verbalisierbar, kontrollierbar
							\item Messung durch Skalen oder Differenziale
							\item Korrelieren mit bewusstem Verhalten
						\end{itemize}
					\item Implizit
						\begin{itemize}
							\item Bauchgefühl, schnell und automatisch
							\item Messung durch eval. Priming oder impl. Assoziationstest
							\item Korrelieren mit unwillkürlichem Verhalten
						\end{itemize}
				\end{itemize}
			\item Einstellung \& Verhalten
				\begin{itemize}
					\item LaPiere: Chinese wird in Hotel aufgenommen, auf Anfrage aber abgelehnt
					\item Geringe Korrelation zwischen Einstellung und Verhalten
					\item Passung (Maß und Verhalten), Art des Verhaltens und Einstellungsstärke / Verfügbarkeit wichtig
				\end{itemize}
			\item Theorie des geplanten Verhaltens (Ajzen)
				\begin{itemize}
					\item Verhaltensrelevante Einstellung
					\item wahrgenommene soz. Norm
					\item wahrgen. Kontrolle
					\item $\rightarrow$ Verhaltensabsicht $\rightarrow$ Verhalten
					\item Emot. Komponente? spontanes Verhalten?
				\end{itemize}
			\item RIM-Modell: Reflexives (überlegt) und impulsives (spontan) System
		\end{itemize}
	
	\section{Einstellungsänderung}
		\begin{itemize}
			\item Persuasion
				\begin{itemize}
					\item Gezielter Überzeugungsversuch
					\item Art der Botschaft und Einstellung sollten passen (emotion/kognition)
					\item Angstappelle nur moderat mit Alternativmöglichkeit
					\item Yale-Ansatz (Hovland et al.)
						\begin{itemize}
							\item WER (Quelle) sagt WAS (Botschaft) zu WEM (Empfänger)
							\item Sender: Exppertise, Vertrauenswürdigkeit, Sypathie
							\item Botschaft: Qualität der Args., wahrgenommene Beeinflussung
							\item Empfänger: Aufmerksamkeit, Motivation, Einstellung
							\item Sleeper-Effekt: Quelle vergessen
						\end{itemize}
					\item 2-Prozess-Theorien: oberfl. / heurustisch vs. gründlich / systematisch
					\item Elaboration-Likelihood Model (Petty \& Cacioppo)
						\begin{itemize}
							\item Information $\rightarrow$ Motivation? Fähigkeit?
							\item Periphere Verarbeitungsroute: heuristisch, oberflächlich
							\item Zentrale Route: sorgfältig / kritisch, inhaltliche Aspekte
							\item Einstellungsänderung instabil vs. stabil
						\end{itemize}
					\item Faktoren: need for cognition / closure, Stimmung (pos $\leftrightarrow$ heur.)
					\item Wahrheitsmotivation vs. Abwehrmotivation vs. Eindrucksmotivation
					\item Widerstand
						\begin{itemize}
							\item Einstellungsimpfung (Gegenargumente in geringen Dosen)
							\item Reaktanz zum Schutz bedrohter Freiheit
						\end{itemize}
				\end{itemize}
			\item Verhaltensänderung
				\begin{itemize}
					\item Kognitive Dissonanz (Festinger)
						\begin{itemize}
							\item Bei zwei unpassenden, selbst-relevanten Kognitionen
							\item Z.B. Einstellung vs. Verhalten
							\item Verhaltens- / Einstellungsänderung je nach Widerstand
								Alternativ Selbstrechtfertigung durch konsonante Infos oder Umdeuten
						\end{itemize}
					\item Einstellungsänderung durch Dissonanz
						\begin{itemize}
							\item Forced / Induced Compliance: freiwillig einstellungskonträres Verhalten zeigen
								\begin{itemize}
									\item Festinger \& Carlsmith: langweilig $\rightarrow$ für Geld Vpn von Spannung 
										überzeugen
									\item Aronson \& Carlsmith: Spielzeug verboten $\rightarrow$ je nach Konsequenz 
										schlechter 
								bewertet
								\end{itemize}
							\item Rechtfertigung von Anstrengung: Anstrengung $\Rightarrow$ Ziel ist toll (Initiationsriten)
							\item Nachentscheidungsdissonanz: nach schwierigen Entscheidung wird gewählt Alternative 
								aufgewertet
							\item Dissonanz vs. Selbstwahrnehmung (Attribution)
						\end{itemize}
					\item Verhaltensänderung durch Dissonanz: Hypocrisy intervention
				\end{itemize}
			\item Änderung impliziter Einstellungen
				\begin{itemize}
					\item Matching Hypothese: explizit durch Persuasion, implizit durch affektive Prozesse
					\item Empirische Gegenbeweise dafür
					\item Reduktion impliziter Biases (Kondition., Persuasion, Konfrontation usw.)
					\item Effektgröße gering und kein Effekt auf Verhalten
				\end{itemize}
			\item Anwendungskontext Werbung
		\end{itemize}
	
	\section{Kommunikation und Sprache}
		\begin{itemize}
			\item Definition
				\begin{itemize}
					\item Kommunikation: Übertragung von Infos, wechselseitig, mehrere Kanäle, (un)absichtlich
					\item Sprache: Laute mit geteilter semantischer Bedeutung
					\item Para-Sprache: Lautstärke, Geschwindigkeit, Seufzen usw.
				\end{itemize}
			\item Non-verbale \& paraverbale Kommunikation
				\begin{itemize}
					\item Gestik (Universalien vs. Embleme)
					\item Blickkontakt (Intimität / Dominanz, status- und kulturabhängig)
					\item Emotionsausdruck vermittelt Gefühle und Verhaltensdisposition
					\item Basisemot. nach Ekman: Ärger, Furcht, Ekel, Überraschung, Freude, Trauer
					\item Täuschung erkennbar an ungenauer Sprache, Gesichtsausdruck, Sprachmelodie, Selbstberührung
				\end{itemize}
			\item Verbale Kommunikation
				\begin{itemize}
					\item Sprache \& Kognition
						\begin{itemize}
							\item Sapir-Whorf-Hypothese: Sprache bestimmt denken
							\item Relativiert: Linguistische Kategorien erleichtern Kognition und Kommunikation
							\item Framing-Effekte
								\begin{itemize}
									\item Vergleichsframing (aufwärts / abwärts)
									\item Positives vs. negatives Framing (Medizin)
									\item Metaphern
								\end{itemize}
							\item Vergleichsfokus
								\begin{itemize}
									\item Tversky: Ähnlichkeit nord Korea / China
									\item Unbekannteres ist bekannterem ähnlicher
									\item Feature Matching: Subjekt im Vordergrund und matching mit Eigenschaften 
										des Objekts
									\item Unrealistischer Optmimismus größer bei selbst $\rightarrow$ andere
									\item Vergleichsrichtung: unbekannt, unwichtig, klein $\rightarrow$ bekannt, 
										wichtig, groß
								\end{itemize}
							\item Sprache und Stereotype
								\begin{itemize}
									\item Stereotype über Sprachvarianten (Herkunft, Status durch Dialekt / 
										Hochsprache)
									\item Aufrechterhaltung von S. durch Sprache\\
										$\rightarrow$ beeinflussen (Framing) und spiegeln sich wieder 
										(Metaphern, Vergleichsrichtung, Verneinung)
									\item Linguistic Category Model (Semin): Deskriptiv, Interpretativ, 
										Zustandsverben, Adjektive
									\item Linguistic Expectancy Bias: konsistent $\rightarrow$ abstrakt 
										$\rightarrow$ internal\\
										inkonsistent $\rightarrow$ konkret $\rightarrow$ external
								\end{itemize}
						\end{itemize}
					\item Sprache im Intergruppenkontext / soz. Identität
						\begin{itemize}
							\item Sprache \& soziale Identität
								\begin{itemize}
									\item Theorie der soz. Ident.: Selbstbild aus Gruppenmitgliedschaft
									\item Sprache Marker für soz. Ident.
									\item Kategorisierung nach verwendeter Sprache
									\item Ethnolinguistische Vitalität: Aufrechterhalten der Sprache in gemischt 
										ethno-linguistischem Kontext
								\end{itemize}
							\item Speech / Communication Accomodation
								\begin{itemize}
									\item Sprachstil je nach Kontext ((un)willkürlich)
									\item Konvergenz (interpersonelle Anpassung) vs. Divergenz (Intergruppenkontext 
										v.a. mit hohem Status)
								\end{itemize}
							\item Linguistic Intergroup Bias
								\begin{itemize}
									\item Eigengruppen: pos. abstr., neg. konkr.
									\item Fremdgruppen: pos. konkr., neg. abstr.
									\item LIB v.a. bei Intergruppenkonflikten
									\item Schoel et al.: Bilder zu Cem vs. Tim
								\end{itemize}
						\end{itemize}
					\item Audience Tuning: Informationsverzerrung gemäß der Perspektive des Publikums\\
						$\rightarrow$ Sender übernimmt teilw. Persp. (shared reality)
					\item Echterhoff et al.: Beschreibung einer Person für Anderen mit pos./neg. Einstellung\\
						$\Rightarrow$ audience tuning und saying is believing
				\end{itemize}
		\end{itemize}


	\section{Zwischenmenschliche Anziehung}
		\begin{itemize}
			\item Bedeuting interpersoneller Beziehungen
				\begin{itemize}
					\item Affiliationsbedürfnis (Gesellschaft zu suchen)
					\item Bedürfnis nach Zugehörigkeit (need to belong; pos. und stabile Beziehungen; angeboren)
					\item Direkter Einfluss auf psych. und phys. Wohlbefinden
					\item Ausschluss löst Stress aus (Williams et al.: Cyberball)
				\end{itemize}
			\item Soziale Unterstützung (SU)
				\begin{itemize}
					\item Emotionale (über Gefühle sprechen)
					\item Informationale (Rat von Expertem)
					\item Instrumentelle (Bitte um Unterstützung)
					\item Einschätzung (soz. Vergleich)
				\end{itemize}
			\item Gründe für Anziehung
				\begin{itemize}
					\item Räumliche Nähe (Festinger, Schachter: Wohnheim; Back: Studieneinführung)
					\item Ähnlichkeit
						\begin{itemize}
							\item Homogamie: Partner haben ähnl. gesellschaftlichen Hintergr.
							\item Wahrgenommene (psychologische) \\Ähnlichkeit entscheidend
							\item Similarity-Attraction effect: ähnlich $\Rightarrow$ sympathisch
							\item Sympathie $\Rightarrow$ Wahrnehmung höherer Ähnlichkeit
						\end{itemize}
					\item Physische Attraktivität
						\begin{itemize}
							\item Halo-Effekt
								\begin{itemize}
									\item Zuschreibung von kulturell positiven Eigenschaften
									\item Stärker in individualistischen Kulturen
									\item Anderson et al.: UVs (attr., Land, Stadt vs. Land, Priming 
										in- / interdependenz) $\rightarrow$ AVs (Eigenschaften, Life Outcome)
								\end{itemize}
							\item Snyder: Telefongespräch freundlicher wenn Partner angebl. attraktiv
							\item Extreme Attraktivität hemmt Interaktion
							\item Symmetrie, umgedrehtes Dreieck, Sanduhr
							\item Abhängig von Kultur und Kontext (Sperrstundeneffekt)
							\item Weniger wichtig als Ähnlichkeit (Koranyi: Aufmerksamkeitsbias für Attr. schwindet bei wechsels. Interesse)
						\end{itemize}
					\item Fehlattribution von Erregung
				\end{itemize}
			\item Arten von Beziehungen
				\begin{itemize}
					\item Austauschorientierte Beziehungen
						\begin{itemize}
							\item Ausgewogenheit, Erfüllung gegenseitiger Bedürfnisse in gleichem Maß
							\item Bei fehlender Erwiderung wird ggf. Beziehung beendet
						\end{itemize}
					\item Gemeinschaftsorientierte Beziehungen
						\begin{itemize}
							\item Gegenseitige Verantwortung und geteilte Identität
							\item Bedürfnisse des Anderen im Vordergrund
						\end{itemize}
				\end{itemize}
			\item Geschlechtsunterschiede
				\begin{itemize}
					\item Parental Investment: Männer Anzahl Nachkommen maximieren(Attrakt. $\rightarrow$ Gesundheit), Frauen 
						Überlebenswahrscheinlichkeit (Status, Ressourcen)
					\item Statusunterschiede: Frauen in höherem Status legen mehr Wert auf Attraktivität (Gangestad)
					\item Korrelative Studien problematisch
				\end{itemize}
			\item Beziehungszufriedenheit
				\begin{itemize}
					\item Bindungsstile: sicherer $\rightarrow$ wenig Eifersucht und Verlassensangst, mehr Vertrauen (im Vergleich 
						zu ängstlich-ambivalent und vermeidend)
					\item Pos. Faktoren: Verständnis, Bereitschaft, geteilte soz. Ident., pos. Attribution, Equity
					\item Neg.: Konfliktvermeidung, Beschwichtigung, destuktive Kommunikation
				\end{itemize}
			\item Dreieckstheorie
				\begin{itemize}
					\item Mögen (Vertrautheit)
					\item Vernarrtheit (Leidenschaft)
					\item Leere Liebe (Festlegung)
					\item Romantisch (M+V), Kumpel (M+L), Albern (V+L), Vollkommen
				\end{itemize}
			\item Investitionsmodell: Zufriedenheit, Investition (+), Alternativen(-) $\rightarrow$ Festlegung $\rightarrow$ Trennung
			\item Beziehungsende
				\begin{itemize}
					\item Seelisch / körperlich ungesünder als (nie) verheiratet
					\item Aktiv leidet weniger als passiv
					\item Einvernehmlichkeit wirkt positiv
				\end{itemize}
		\end{itemize}



	\section{Prosoziales Verhalten}
		\begin{itemize}
			\item Definitionen
				\begin{itemize}
					\item Hilfeverhalten: Situation einer anderen Person verbessern
					\item Altruismus: Primäres Ziel ist, zu helfen
					\item Zivilcourage: Riskantes Einschreiten für gesellschaftliche Norm
				\end{itemize}
			\item Gründe für Hilfeverhalten
				\begin{itemize}
					\item Evolutionär
						\begin{itemize}
							\item Verwandtenselektion
							\item Reziproker Altruismus (Bestrafung von Betrügern, verbesserte Erinnerung)
							\item Folge von genetischen Verhaltensmustern: soz. Normen, Perspektivenübernahme
						\end{itemize}
					\item Austauschtheorien
						\begin{itemize}
							\item Hilfe wenn Nutzen $>$ Kosten
							\item Auch Nutzen der Hilfe / Kosten der Nicht-Hilfe für Empfänger
						\end{itemize}
					\item Empathie
						\begin{itemize}
							\item Empathie-Altruismus Hypothese: Wahrnehmung $\rightarrow$ Perspektivenübernahme? 
								$\rightarrow$ Empathie / Unbehagen $\rightarrow$ Altr. / Egoist.
							\item Toi \& Batson: Hören von Unfall aus selbem / anderen Kurs mit / ohne Persp.übern. 
								$\rightarrow$ Hilfe?
							\item Gegenmodell: negative state relieve model $\rightarrow$ Hilfe immmer egoistisch (Cialdini)
						\end{itemize}
					\item Stimmung
						\begin{itemize}
							\item Pos. erhöht Wahrsch. $\rightarrow$ \\Mood-maintenance Hypothese
							\item Neg. (Schuldgefühle) erhöhen auch: negative state relief Hyp.
							\item Bei neg. aber eingeengter Aufmerksamkeitsfokus
						\end{itemize}
					\item Weitere Faktoren
						\begin{itemize}
							\item Personenfaktoren (Empathiefähigkeit, Normen, Selbstwirksamkeit)
							\item Soz. Normen: Verantwortungszuschreibung, Norm der soz. Verantwortung
							\item Soz. Ident. / Gruppenmitgl. (EG: Empathie, FG: Kosten / Nutzen)
							\item Geschlechtsunterschiede
						\end{itemize}
				\end{itemize}
			\item Gründe für keine Hilfe
				\begin{itemize}
					\item Nicht bemerken, Kosten / Nutzen
					\item Darley \& Batson: Zeitdruck (3) und verf. Norm (Samariter vs. Beruf) $\rightarrow$ Hilfeverhalten
					\item Bystandereffekt: Je mehr helfen könnten, desto weniger hilft der Einzelne
					\item Darley \& Latané: Gespräch $\rightarrow$ ``VPN'' Anfall $\rightarrow$ Hilfe je nach Anzahl Zeugen
					\item Gründe: Verantwortungsdiffusion, Bewertungsangst, Pluralistische Ignoranz (falsche Annahme für Gründe)
					\item Latané \& Rodin: alleine / mit unbek. / mit Freund / mit passiv $\rightarrow$ hören Sturz
					\item 5-Stufen-Modell von Latané \& Darley
						\begin{itemize}
							\item Bemerken
							\item Interpretieren als Notsituation
							\item Persönliche Verantwortung
							\item Handlung für Unterstützung verf.
							\item Umsetzen
						\end{itemize}
				\end{itemize}
			\item Kritische Ansätze
				\begin{itemize}
					\item Spontanes / impulsives Hilfeverhalten?
					\item Langfristige Hilfe?
					\item Negative Effekte für Hilfeempfänger (Macht, Abhängigkeit)
				\end{itemize}
		\end{itemize}



	\section{Aggression}
		\begin{itemize}
			\item Definitionen
				\begin{itemize}
					\item Aggressives Verhalten: Verhalten mit Ziel Schaden / Verletzung ohne Zustimmung zuzufügen
					\item Gewalt: aggr. Verhalten mit Ziel körperl. Schäden
					\item Instrumentelle vs. feindeslige
					\item direkte vs. indirekte
					\item spontan vs. reaktiv (provoziert)
				\end{itemize}
			\item Forschungsfaktoren
				\begin{itemize}
					\item Selten und unerwünscht $\rightarrow$ Beobachtungsdaten und Selbstbericht schwierig
					\item Archivdaten und Fremdbericht
					\item Nur Absicht erfassen und nicht über experiment hinaus Aggr. erhöhen
				\end{itemize}
			\item Personale Einflussfaktoren
				\begin{itemize}
					\item Persönlichkeitseigenschaft: z.B. Selbstbericht (Buss \& Perry)
					\item Feindseliger Attributionsstil (korr. mit Trait und Verhalten)
					\item Geschlechtsunterschiede
						\begin{itemize}
							\item Männer höhere Aggr., phys > verb., spontan > reakt.
							\item Indirekte im Jugendalter bei Mädchen höher
							\item Kulturübergreifend aber abhängig von Sozialisation
						\end{itemize}
				\end{itemize}
			\item Situationale Einflussfaktoren
				\begin{itemize}
					\item Alkohol (v.a. bei geringer Empathie, eingeschr. Info Verarbeitung, Hemmung von Normen)
					\item Hitze-Hypothese (korrelativ)
					\item Mediengewalt
						\begin{itemize}
							\item Erhöht v.a. kurzfristig aggr. Verhalten
							\item Konsum mit 8 $\rightarrow$ Aggression im Alter von 8 / 18 / 30
							\item Soziales Lernen, Habituation, normative Akzeptanz
						\end{itemize}
				\end{itemize}
			\item Biologische Theorien
				\begin{itemize}
					\item Dampfkessel-Modell (Lorenz)
						\begin{itemize}
							\item Druck wird nach entspr. Reiz entladen
							\item Empirisch widerlegt
						\end{itemize}
					\item Verhaltensgenetik
						\begin{itemize}
							\item Viele Unterschiede auf Genetik zurückzuführen aber Umwelt entscheidet über Auswirkung
							\item Genetischer Risikofaktor erhöht Wahrscheinlichkeit bei Gewalterfahrungen
						\end{itemize}
					\item Hormonelle Modelle
						\begin{itemize}
							\item Testosteron korreliert positiv, Cortisol negativ
							\item Reagieren stark auf situative Veränderung
						\end{itemize}
				\end{itemize}
			\item Psychologische Theorien
				\begin{itemize}
					\item Frustrations-Aggressions-Hypothese
						\begin{itemize}
							\item Aggressionsverschiebung
							\item Aggressive Hinweisreize entscheiden ob Aggression oder anderer Frustabbau
							\item Waffeneffekt (Berkowitz): Provozierte Pers. längere Schocks mit Waffe statt Schläger
						\end{itemize}
					\item Kognitiv neo-assoziationistisches Modell
						\begin{itemize}
							\item Aggr. Reiz $\rightarrow$ Neg. Affekt $\rightarrow$ assoz. Reaktion (Fight / Flight) 
								$\rightarrow$ Ärger / Furcht $\rightarrow$ Denken $\rightarrow$ Gefühle $\rightarrow$ 
								Verhalten
						\end{itemize}
					\item Lerntheorien
						\begin{itemize}
							\item Direkte Verstärkung
							\item Lernen am Modell (Bandura: Bobo Doll)
							\item Sozial kognitiv: aggressive Skripts (Schemata)
						\end{itemize}
					\item General Aggression Model
						\begin{itemize}
							\item Individuelle Unterschiede, Situative Variablen
							\item $\rightarrow$ Zustand: Kognitionen, Affekte, Erregung
							\item $\rightarrow$ Automatische Bewertung
							\item $\rightarrow$ Kontrollierte Neubewertung $\rightarrow$ Verhalten
						\end{itemize}
				\end{itemize}
			\item Interventionsmöglichkeiten
				\begin{itemize}
					\item Katharsis empirisch nicht belegt
					\item Bestrafung nur unmittelbar, wahrscheinlich, unanggressiv, mit verfügbaren Alternativen
					\item Unvereinbare Reaktionen (Musik)
					\item Trainings, etablieren sozialer Normen
				\end{itemize}
		\end{itemize}


	\section{Sozialer Einfluss}
		\begin{itemize}
			\item Definition
				\begin{itemize}
					\item Wirkung auf Gedanken, Gefühle oder Verhalten anderer Person(en)
					\item Veränderung von Einstellungen, Überzeugungen usw. durch Konfrontation mit Anderen
				\end{itemize}
			\item Beiläufiger sozialer Einfluss
				\begin{itemize}
					\item Bloße Anwesenheit
						\begin{itemize}
							\item Triplett: Kinder wickeln alleine oder gleichzeitig Faden\\
								$\Rightarrow$ Soziale Erleichterung (schneller mit anderen)
							\item Bei schwierigen Aufgaben soziale Hemmung
							\item Zajonc: Anwesenheit $\rightarrow$ Erregung $\rightarrow$ leichter dominante / 
								schwerer nicht dom. Reaktionen $\rightarrow$ einfach leichter / schwer schwerer
							\item Alternativ: Bewertungsangst, Aufmerksamkeitskonflikt
						\end{itemize}
					\item Einfluss von Normen
						\begin{itemize}
							\item implizit / explizit, betrifft Verh., Werte, Überzeugungen
							\item Erwartung, die für alle gilt
							\item Präskriptiv: Was soll man tun?
							\item Deskriptiv: Was tun andere?
							\item Reduzieren Unsicherheit, erleichtert Koordination
							\item Cialdini: Handtuch wiederverwenden $\rightarrow$ andere (deskr.) > Umwelt (präskr.)
						\end{itemize}
					\item Bildung und Weitergabe von Normen (Sherif)
						\begin{itemize}
							\item Autokinetischer Effekt, 1x allein, 3x Gruppe
							\item allein, dann Gruppe: konvergenz, ansonsten leichte Divergenz
							\item Bleibt zunächst erhalten auch wenn Einflussnehmer nichtmehr in Gruppe
						\end{itemize}
					\item Einfluss soz. Rollen (Stanford Prison)
						\begin{itemize}
							\item Erwartung gegenüber Mitglied einer Gruppe (vgl. Norm)
							\item Einteilung in Wärter / Gefangene mit Uniformen und Nummern (Deindividuation)
							\item Abbruch wegen extremen Verhaltens
							\item Verschwimmen der Realität mit Simulation
							\item Ethische Kritik (Stress, Abbruch nur schwer möglich)
							\item Methodisch: Forscher als Beteiligte, Erwartungseffekte, selektive Datensammlung
							\item Kein Experiment (Standardisierung, Kontrolle)
						\end{itemize}
					\item Folgeforschung
						\begin{itemize}
							\item BBC Prison Study (Reicher \& Haslam)
							\item Theoretischer Hintergrund: social identity theory
							\item Personen verhalten sich gemäß Rolle wenn Teil ihres Selbstkonzepts (social identification)
							\item Systematische Datenerhebung, ethische Richtlinien
							\item Interventionen: Durchlässigkeit, Legitimität, neuer Gefangener
							\item Durchlässigkeit wichtig für Identifikation
							\item Revolte ohne Widerstand, funktioniert nicht, Rückkehr, Stillstand
						\end{itemize}
					\item Konformität / Einfluss von Minderheiten (Asch)
						\begin{itemize}
							\item Linienvergleich, Antwort reihum mit vorletzter Person als Vpn
							\item Andere konsistent falsche Antwort $\rightarrow$ Konformität der Vpn
							\item Häufig mindestens eine falsche, selten alle
							\item Bei abweichenden Antowrten sinkt Konformität
							\item Normativer Einfluss (aber nicht nur: schriftliche Antowrten)
						\end{itemize}
				\end{itemize}
			\item Gründe für soz. Einfluss
				\begin{itemize}
					\item Informationaler soz. Einfluss
						\begin{itemize}
							\item Bedürfnis nach richtiger Einschätzung
							\item Personen als Informationsquelle
							\item Bleibt über Situation erhalten (Sherif, autokinetischer Eff.)
						\end{itemize}
					\item Normativer soz. Einfluss
						\begin{itemize}
							\item Bedürfnis nach Akzeptanz / Harmonie
							\item Bleibt meist nicht erhalten (Asch: Konformität)
						\end{itemize}
				\end{itemize}
			\item Absichtlicher soz. Einfluss
				\begin{itemize}
					\item Compliance / Verkaufsstrategien
						\begin{itemize}
							\item Compliance: freiwilliges Nachgeben bei mildem Einflussversuch
							\item Foot-in-the-door (Konsistenzstreben)
							\item Door-in-the-face (Reziprozitätsnorm)
							\item Low-balling (Commitment ausnutzen)
							\item Reziprozität: Geschenk vor Bitte
							\item Heuristiken, situationale / personale Gegebenheiten nutzen
						\end{itemize}
					\item Gehorsam gg. Autoritäten (Milgram)
						\begin{itemize}
							\item Lehrer (Vpn) bestraft Schüler mit Elektroschocks
							\item Steigende Reaktionen des Schülers
							\item Standardisierte Reaktionen des VL (steigend)
							\item Viele gehorchen bis Ende aber Sichtkontakt / Berührung verringert Wahrsch.
							\item (Un)Gehorsamer weiterer Teilnehmer erhöht / verringert Wahrsch. stark
							\item Methodisch: Interpretation unklar, externe Validität?
							\item Ethisch: extremer Stress für Vpn
						\end{itemize}
					\item Folgeforschung
						\begin{itemize}
							\item Burger: ähnlich mit Abbruch bei 150V und ethische Richtlinien
							\item Haslam et al.: Identifikation mit übergeordnetem Ziel; für symp. Gruppen negative Labels 
								wählen mit standardisierter Aufforderung zum Weitermachen
							\item höheres Gut > Bitte > Befehl
						\end{itemize}
					\item Einfluss Minderheiten
						\begin{itemize}
							\item Konformität vs. Exklusion vs. Minoritäteneinfluss
							\item Moscovici: Farbe von blauen Dias beurteilen
							\item UV: 6 Vpn vs. 4 Vpn + 2 Konf (grün, zufällig / konsistent)
							\item Minderheiteneinfluss wenn konsistent
							\item Metaanalyse (Wood): Mehr. > Minderh.; Minderh. indirekt, verzögert, privat
						\end{itemize}
					\item Konversionstheorie (Moscovici)
						\begin{itemize}
							\item Mehrheit: Konformitätsdruck $\rightarrow$ Vergleichsprozess (Was?)
							\item Minderheit: fällt auf, überrascht $\rightarrow$ Validierungsprozess (Warum?)
						\end{itemize}
					\item Integration mit Persuasionsforschung (Martin \& Hewstone)
						\begin{itemize}
							\item Mot. \& Ress. gering: heuristisch $\rightarrow$ Mehrheit
							\item Hoch: systematische Verarbeitung $\rightarrow$ Qualität der Args. entschiedet
						\end{itemize}
				\end{itemize}
			\item Entscheidungsfindung in Gruppen
				\begin{itemize}
					\item Gruppenpolarisierung
						\begin{itemize}
							\item Gruppenentscheidung extremer als Durchschnitt (v.a. bei ähnl. Tendenzen)
							\item Argumente (inf.), soz. Vergl. und Selbstkategorisierung (norm.)
							\item Norm. bei Urteil, Harmonie, individuelle Meinung öffentlich
							\item Inf. bei intellekt. Aufgaben, richtige Entscheidung wichtig, anonym
						\end{itemize}
					\item Gruppendenken
						\begin{itemize}
							\item Schlechte Entscheidungen in Gruppen mit hoher Kohäsion
							\item Unrealistische Bewertung der Situation durch normativen Einfluss
							\item Qualitative Forschung (u.A. Janis)
							\item Ursachen: hohe Kohäsion, Gruppenisolation, direkte Führung, Lösungsdruck
							\item Symptome: Illusion von Unverwundbarkeit / Einstimmigkeit, Stigmatisierung von Abweichlern, 
								Schwar-Weiß-Denken
							\item Konsequenzen: mangelnde Risikoanalyse / Alternativen / Informationen
							\item Gegenmaßnahmen: advocatus diavoli, externe Experten, Untergruppen, anonym abstimmen, Zeit
						\end{itemize}
				\end{itemize}
		\end{itemize}



	\section{Gruppenpsychologie allgemein}
		\begin{itemize}
			\item Definitionen
				\begin{itemize}
					\item Personen die sich als zusammengehörig wahrnehmen (Interaktion, gem. Schicksal, relevante soziale 
						Kategorie)
					\item Merkmale: Etitativität, Interaktion, Ähnlichkeit, Durchlässigkeit, Größe usw.
					\item 4 Gruppenarten (Lickel et al.)
						\begin{itemize}
							\item Gruppen mit Intimität (Familie $\rightarrow$ Verbundheit, Nähe)
							\item Aufgabenbezogene (Arbeitsgruppe $\rightarrow$ Zielerreichung)
							\item Soz. Kategorien (Geschlecht, eth. Gruppe $\rightarrow$ Identität)
							\item Lockere Verbindung (Publikum)
						\end{itemize}
					\item Gründe für Gruppenanschluss
						\begin{itemize}
							\item Evolutionspsych.: höhere Überlebenschance
							\item Austauschtheorie: Bedürfnisbefriedigung (Kosten / Nutzen)
							\item Soz. Vergleich: Reduktion von Unsicherheit
							\item Soz. Identität: (pos.) Selbstdefinition
						\end{itemize}
					\item Hogg et al.: (Un)Sicherheit aufschreiben $\rightarrow$ Identifikation mit hoch / niedrig entitiver Gruppe
				\end{itemize}
			\item Gruppensozialisation
				\begin{itemize}
					\item Anpassung neuer Mitglieder an Gruppennormen und -praktiken
					\item 5 Stadien (Moreland \& Levine)
						\begin{itemize}
							\item Erkundung $\rightarrow$ Eintritt
							\item Sozialisation $\rightarrow$ Akzeptanz
							\item Aufrechterhaltung $\rightarrow$ Divergenz
							\item Resozialisierung $\rightarrow$ Austritt
							\item Erinnerung
						\end{itemize}
					\item Erkundung: geeignet? (Kompetenz / Passung), (unangenehme) Initiationsriten $\rightarrow$ höhere Sympathie 
						durch Dissonanz
					\item Sozialisation: Gruppennormen, Fähigkeiten, Rolle lernen
					\item Aufrechterhaltung: Festlegung und Rollenaushandlung
					\item Resozialisierung: Gruppendruck auf Mitglied $\rightarrow$ Anpassung
				\end{itemize}
			\item Phasen des Gruppenlebens (Tuckman)
				\begin{itemize}
					\item Forming: Kennenlernen
					\item Storming: Konfliktphase $\rightarrow$ Bildung von Gruppenstruktur
					\item Norming: Normierung / Einigung über Ziele / Normen
					\item Performing: Arbeitsphase
					\item Adjourning: Beendigungsphase
					\item Aufgabenbezogenes vs. sozioemotionales Verhalten im Vordergrund
				\end{itemize}
			\item Strukturmerkmale von Gruppen
				\begin{itemize}
					\item Gruppennormen: präs. / desk., leichtere Interakt.
					\item Kohäsion: aufgabenbezogene / interpersonelle
					\item Status / Rollen: formell vs. informell
					\item Theorie der Erwartungszustände: Statusmerkmale diffus (Alter, Geschlecht) oder spezifisch (Erfahrung)
				\end{itemize}
			\item Geteilte Realität in Gruppen
				\begin{itemize}
					\item Emotionale Ansteckung: Übertragung von Stimmung innerhalb der Gruppe
					\item Transaktives Gedächtnis: Wissen über Gruppenmitglieder verteilt
					\item Morlenad: Radio bauen lernen alleine vs. alleine mit Austausch vs. in Gruppe
					\item Wegner: Erinnerungsaufgaben mit echten Paaren besser / mit unbekannten mit Rollenaufteilung
				\end{itemize}
			\item Gruppen und Umwelt
				\begin{itemize}
					\item Präsenz anderer Gruppe macht Mitgliedschaft salient besonders bei Wettbewerb)
					\item Intergruppenkontexte machen verschiedene Gruppenmerkmale salient
					\item Unterschiedliche Bewertung eines Merkmals je nach Kontext
				\end{itemize}
		\end{itemize}

	\section{Gruppenleistung \& Führung}
		\begin{itemize}
			\item 
		\end{itemize}


	\section{Intergruppale Prozesse}
		\begin{itemize}
			\item
		\end{itemize}


	\section{Kultur}
		\begin{itemize}
			\item
		\end{itemize}

	\section{Angewandte Sozialpsychologie}
		\begin{itemize}
			\item 
		\end{itemize}



\end{document}
