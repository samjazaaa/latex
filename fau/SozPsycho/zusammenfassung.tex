\documentclass[11pt, paper=a4, twocolumn]{scrartcl}

\usepackage[ngerman]{babel}
\usepackage[utf8]{inputenc}

\usepackage[T1]{fontenc}
\usepackage{mathpazo}

\usepackage{geometry}

\geometry{a4paper, top=20mm, left=15mm, right=15mm, bottom=20mm,
headsep=5mm, footskip=12mm}


\pagenumbering{gobble}

\title{\vspace{-1.25cm}Zusammenfassung Sozialpsychologie\vspace{-0.25cm}}
\date{\vspace{-5ex}}

\newcommand*{\Z}{\mathbb{Z}}

\begin{document}
	\maketitle


	\section{Hintergrund}
		\begin{itemize}
			\item Definition
				\begin{itemize}
					\item Denken, Fühlen und Handeln
					\item Beeinflusst durch die tatsächliche, vorgestellte oder implizierte Gegenwart anderer
				\end{itemize}
			\item Personen- vs. Situationsbezogen
			\item Subjektive vs. objektive Wahrnehmung
		\end{itemize}


	\section{Methoden}
		\begin{itemize}
			\item Empirische Wissenschaft
				\begin{itemize}
					\item Verstehen und Vorhersagen $\Rightarrow$ Interventionen
					\item Theorien: komplexe erklärende Aussagen über Realitätsausschnitt
					\item Hypothesen
						\begin{itemize}
							\item Aus Theorien abgeleitet
							\item falsifizierbar
							\item Operationalisierung / Quantifizierung
							\item Hypothesentestende vs. explorative Forschung
						\end{itemize}
				\end{itemize}

			\item Deskriptive Forschung
				\begin{itemize}
					\item Wie ist das Phänomen?
					\item (Un-)Standardisierte Beobachtung
					\item Dokumentenanalyse
					\item Befragungen
					\item Einschränkungen: schwer / nur in bestimmten Kontexten beobachtbar, kein warum
				\end{itemize}

			\item Korrelative Forschung
				\begin{itemize}
					\item Korellationen (pos., neg., keine)
					\item Keine Kausation $\Rightarrow$ deskriptiv
				\end{itemize}

			\item Experimentelle Forschung
				\begin{itemize}
					\item UV manipulieren und AV messen
					\item Standardisierung und Randomisierung
					\item Design
						\begin{itemize}
							\item $\#UV1 * \#UV2 * \dots$
							\item Mediator als Zwischenvariable / Folge der UV
							\item Moderator beeinflusst Stärke des Zusammenhangs
							\item Haupteffekt vs. Interaktionseffekt
						\end{itemize}
					\item Gütekriterien
						\begin{itemize}
							\item Reliabilität (Zuverlässigkeit)
							\item Validität (das richtige gemessen?)
							\item Konfundierung von Vars. (intern)? Generalisierbarkeit (extern)?
						\end{itemize}
					\item Coverstory, Konföderierte, Manipulationscheck
					\item Grenzen: VL/VP-Effekte, Ethik
				\end{itemize}

			\item Qualitative Forschung
				\begin{itemize}
					\item Meist explorativ
					\item Induktiv Theorien entwickeln
					\item Diskursanalyse
					\item Verkuyten (2005): Triangulation; Diskurs $\rightarrow$ Text, dann Text mit (keine) Wahl $\rightarrow$ 
						Unterstützung von Multikult.
				\end{itemize}

			\item Replikationskrise
		\end{itemize}


	\section{Soziale Kognition}
		\begin{itemize}
			\item Grundsätzliches
				\begin{itemize}
					\item Denken im sozialen Kontext
					\item Fokus auf subjektives Erleben und (systematische) Fehler
				\end{itemize}
			\item Prozessstufen der soz. Informationsverarbeitung
				\begin{itemize}
					\item Informationsinput
					\item Wahrnehmung
					\item Aufmerksamkeit (nach Salienz, Relevanz, aktiviertem Wissen)
					\item Kategorisierung (Zuordnung Kategorie / Schema, Anreicherung mit Wissen)
					\item Integration (zusammenfassen und mit gespeichertem Material integrieren)
					\item Urteil, Verhalten
					\item Selektion $\Rightarrow$ Inferenz
					\item Bottom-up (oben) vs. Top-down
				\end{itemize}

			\item Automatische / Kontrollierte Verarbeitung
				\begin{itemize}
					\item Kontrolliert: absichtlich, bewusst, aufwändig $\rightarrow$ gezielt und selten
					\item Automatisch: unbewusst, mühelos $\rightarrow$ ständig und unwillkürlich
					\item Einfluss wenn: Motivation, Ressourcen, Verzerrungen bewusst
					\item Kontinuummodell (Fiske): autom. Kat. $\rightarrow$ Neu/sub $\rightarrow$ Individualisierte Wahrn.
				\end{itemize}

			\item Kognitive Schemata
				\begin{itemize}
					\item Teil der Kategorisierung
					\item Wissen über soziale Welt nach Themenbereichen
					\item Skripte (Restaurant), Selbstbild, soz. Rollen, Stereotype
					\item Funktionen
						\begin{itemize}
							\item Organisation und Sinnverleihung
							\item Mustererkennung
							\item Erinnerungshilfe
						\end{itemize}
					\item Auswirkungen
						\begin{itemize}
							\item Wahrnehmung (Gorilla im Bild passt nicht in Schema)
							\item Erinnerung (schemakonsistente Infos, siehe Cohen: Bibliothekarin vs. Kellnerin)
							\item Urteilsbildung (Zugänglichkeit des Schemas wichtig)\\
								recency \& frequency, situative vs. chronische Verfügbarkeit
						\end{itemize}
					\item Priming
						\begin{itemize}
							\item Beiläufige Voraktivierung eines vorhandenen Schemas
							\item Einfluss auf weitere Verarbeitung
							\item subliminal vs. supraliminal
							\item affektiv / semantisch / behavioral
						\end{itemize}
					\item Donald-Studie von Higgins, Rholes \& Jones 1977
						\begin{itemize}
							\item Wahrnehmungsexp. mit eingebetteten Eigenschaftsbegriffen (pos/neg)
							\item $\Rightarrow$ Beschreibung zu ambiv. Donald beurteilen (pos/neg)
							\item UVs: 2 (pos/neg) $\times$ 2 (passend ja/nein)
							\item AV: Bewertung von Donald
							\item $\Rightarrow$ Anwendbar: entspricht Bewertung; ansonsten nicht
						\end{itemize}
				\end{itemize}

			\item Persohnenwahrnehmung
				\begin{itemize}
					\item Erster Eindruck (snap) $\rightarrow$ Verhaltensbeobachtung (Implikationen) $\rightarrow$ Erwartungen
					\item Snap Judgements
						\begin{itemize}
							\item Physische Erscheinung, (para/non)verbale Merkmale
							\item Schnell mit minimaler Information
							\item Korreliert stark mit deliberativen Urteilen (vertrauenswürdige Gesichter)
							\item Hohe intersubjektive Übereinstimmung (Clips von Vorlesungen vs. Eval.)
						\end{itemize}
					\item Zentrale Dimensionen
						\begin{itemize}
							\item Asch 1946: Eigenschaften + warmherzig vs. kaltherzig $\Rightarrow$ pos./neg.
							\item Communion
								\begin{itemize}
									\item Beziehungen / Sympathie
									\item getting along
									\item prim. in Wahrn. anderer
								\end{itemize}
							\item Agency
								\begin{itemize}
									\item Respekt / Ziele erreichen
									\item getting ahead
									\item prim. für Vorhersage von pos. Selbstwert
								\end{itemize}
						\end{itemize}
					\item Verzerrungen
						\begin{itemize}
							\item Primacy/Recency-Effekte
							\item Framing (Reihenfolge, pos./neg., Vergleichsrichtung)
							\item Pluralistische Ignoranz (kollektive Fehlinterpretation durch geteilte falsche Annahme über 
								Verhaltensgründe)\\
								$\rightarrow$ Kontakt zu anderen ethnischen Gruppen
							\item Selbst erfüllende Prophezeihungen\\
								Pygmalion-Effekt (Rosenthal \& Jacobson): Schüler blühen auf
							\item Perseveranzeffekt (Überzeugungen bleiben auch ohne Grund; Ross: falsches feedback)
							\item Confirmatory Bias (suche bestätigende Infos; Snyder: introvertiert)
						\end{itemize}
					\item Urteilsheuristiken
						\begin{itemize}
							\item  Abkürzung mit geringem kogn. Aufwand für komplexe Entscheidungen
							\item Besser als Zufall aber systemathische Verzerrungen
							\item Beispiel: Experten / Nachahmungsheuristik
							\item Ankerheuristik: relativ zu Rahmen
							\item Verfügbarkeitsheuristik: je mehr Infos desto wahrscheinlicher
							\item Repräsentativitätsheuristik: wahrgenommene Ähnlichkeit
							\item Pseudodiagnostizität: Ereignis als Beleg für Vermutung trotz Alternativerklärung\\
								(self-fulf., conf. bias)
						\end{itemize}
				\end{itemize}

			\item Attribution
				\begin{itemize}
					\item Zuschreibung von Gründen zu beobachtetem Verhalten
					\item Handeln = (Anstrengun $\times$ Fähigkeit) + (Schwierigkeit + Zufall)\\
						$\Rightarrow$ intern vs. extern
					\item Kausalattribution: frei entschieden? was machen andere?
					\item Attributionsarten nach Seligman
						\begin{itemize}
							\item personal / internal vs. situational / external
							\item variabel vs. stabil
							\item global vs. spezifisch
							\item kontrollierbar vs. unkontrollierbar
						\end{itemize}
					\item Wann? $\rightarrow$ unerwartet, negativ, selbst-relevant
					\item Warum? $\rightarrow$ strukturierend, Vorhersagen
					\item Kovariationsprinzip (Kelley 1967)
						\begin{itemize}
							\item Konsensus: was machen andere?
							\item Distinktheit: was bei ähnlichen Stimuli?
							\item Konsistenz: nur einmal oder häufiger?
						\end{itemize}
					\item Optimistischer Attributionsstil
						\begin{itemize}
							\item Erfolg: personal, stabil, global
							\item Misserfolg: situational, variabel, spezifisch
						\end{itemize}
					\item Verzerrungen
						\begin{itemize}
							\item Fundamentaler Attributionsfehler
								\begin{itemize}
									\item Tendenz zu personaler Attribution
									\item Jones \& Harris: (un)freiwillig für Castro argumentiert
								\end{itemize}
							\item Perzeptuelle Salienz
								\begin{itemize}
									\item Taylor \& Fiske: Gespräch zwischen zwei Pers. aus verschiedenen Pos.
									\item Pers. in Blickfeld wirkt wichtiger
								\end{itemize}
							\item Gilbert: Zwei-Schritte-Proz.: automatische vs. aufwändigere Attribution
							\item Kulturell:
						\end{itemize}
				\end{itemize}
		\end{itemize}


	\section{Das Selbst}
		\begin{itemize}
			\item 
		\end{itemize}

	\section{Einstellungen und Einstellungsänderung}
		\begin{itemize}
			\item
		\end{itemize}
	
	\section{Kommunikation und Sprache}
		\begin{itemize}
			\item
		\end{itemize}


	\section{Zwischenmenschliche Anziehung}
		\begin{itemize}
			\item
		\end{itemize}



	\section{Prosoziales Verhalten}
		\begin{itemize}
			\item
		\end{itemize}



	\section{Aggression}
		\begin{itemize}
			\item
		\end{itemize}


	\section{Sozialer Einfluss}
		\begin{itemize}
			\item
		\end{itemize}



	\section{Intragruppale Prozesse}
		\begin{itemize}
			\item
		\end{itemize}


	\section{Intergruppale Prozesse}
		\begin{itemize}
			\item
		\end{itemize}


	\section{Kultur}
		\begin{itemize}
			\item
		\end{itemize}





\end{document}
