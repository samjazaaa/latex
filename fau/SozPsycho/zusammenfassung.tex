\documentclass[11pt, paper=a4, twocolumn]{scrartcl}

\usepackage[ngerman]{babel}
\usepackage[utf8]{inputenc}

\usepackage[T1]{fontenc}
\usepackage{mathpazo}

\usepackage{geometry}

\geometry{a4paper, top=20mm, left=15mm, right=15mm, bottom=20mm,
headsep=5mm, footskip=12mm}


\pagenumbering{gobble}

\title{\vspace{-1.25cm}Zusammenfassung Sozialpsychologie\vspace{-0.25cm}}
\date{\vspace{-5ex}}

\newcommand*{\Z}{\mathbb{Z}}

\begin{document}
	\maketitle


	\section{Hintergrund}
		\begin{itemize}
			\item Definition
				\begin{itemize}
					\item Denken, Fühlen und Handeln
					\item Beeinflusst durch die tatsächliche, vorgestellte oder implizierte Gegenwart anderer
				\end{itemize}
			\item Personen- vs. Situationsbezogen
			\item Subjektive vs. objektive Wahrnehmung
		\end{itemize}


	\section{Methoden}
		\begin{itemize}
			\item Empirische Wissenschaft
				\begin{itemize}
					\item Verstehen und Vorhersagen $\Rightarrow$ Interventionen
					\item Theorien: komplexe erklärende Aussagen über Realitätsausschnitt
					\item Hypothesen
						\begin{itemize}
							\item Aus Theorien abgeleitet
							\item falsifizierbar
							\item Operationalisierung / Quantifizierung
							\item Hypothesentestende vs. explorative Forschung
						\end{itemize}
				\end{itemize}

			\item Deskriptive Forschung
				\begin{itemize}
					\item Wie ist das Phänomen?
					\item (Un-)Standardisierte Beobachtung
					\item Dokumentenanalyse
					\item Befragungen
					\item Einschränkungen: schwer / nur in bestimmten Kontexten beobachtbar, kein warum
				\end{itemize}

			\item Korrelative Forschung
				\begin{itemize}
					\item Korellationen (pos., neg., keine)
					\item Keine Kausation $\Rightarrow$ deskriptiv
				\end{itemize}

			\item Experimentelle Forschung
				\begin{itemize}
					\item UV manipulieren und AV messen
					\item Standardisierung und Randomisierung
					\item Design
						\begin{itemize}
							\item $\#UV1 * \#UV2 * \dots$
							\item Mediator als Zwischenvariable / Folge der UV
							\item Moderator beeinflusst Stärke des Zusammenhangs
							\item Haupteffekt vs. Interaktionseffekt
						\end{itemize}
					\item Gütekriterien
						\begin{itemize}
							\item Reliabilität (Zuverlässigkeit)
							\item Validität (das richtige gemessen?)
							\item Konfundierung von Vars. (intern)? Generalisierbarkeit (extern)?
						\end{itemize}
					\item Coverstory, Konföderierte, Manipulationscheck
					\item Grenzen: VL/VP-Effekte, Ethik
				\end{itemize}

			\item Qualitative Forschung
				\begin{itemize}
					\item Meist explorativ
					\item Induktiv Theorien entwickeln
					\item Diskursanalyse
					\item Verkuyten (2005): Triangulation; Diskurs $\rightarrow$ Text, dann Text mit (keine) Wahl $\rightarrow$ 
						Unterstützung von Multikult.
				\end{itemize}

			\item Replikationskrise
		\end{itemize}


	\section{Soziale Kognition}
		\begin{itemize}
			\item Grundsätzliches
				\begin{itemize}
					\item Denken im sozialen Kontext
					\item Fokus auf subjektives Erleben und (systematische) Fehler
				\end{itemize}
			\item Prozessstufen der soz. Informationsverarbeitung
				\begin{itemize}
					\item 
				\end{itemize}

			\item Automatische / Kontrollierte Verarbeitung

			\item Persohnenwahrnehmung

			\item Attribution
		\end{itemize}


	\section{Das Selbst}
		\begin{itemize}
			\item 
		\end{itemize}

	\section{Einstellungen und Einstellungsänderung}
		\begin{itemize}
			\item
		\end{itemize}
	
	\section{Kommunikation und Sprache}
		\begin{itemize}
			\item
		\end{itemize}


	\section{Zwischenmenschliche Anziehung}
		\begin{itemize}
			\item
		\end{itemize}



	\section{Prosoziales Verhalten}
		\begin{itemize}
			\item
		\end{itemize}



	\section{Aggression}
		\begin{itemize}
			\item
		\end{itemize}


	\section{Sozialer Einfluss}
		\begin{itemize}
			\item
		\end{itemize}



	\section{Intragruppale Prozesse}
		\begin{itemize}
			\item
		\end{itemize}


	\section{Intergruppale Prozesse}
		\begin{itemize}
			\item
		\end{itemize}


	\section{Kultur}
		\begin{itemize}
			\item
		\end{itemize}





\end{document}
