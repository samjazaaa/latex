\documentclass[11pt, paper=a4, twocolumn]{scrartcl}

\usepackage[ngerman]{babel}
\usepackage[utf8]{inputenc}

\usepackage[T1]{fontenc}
\usepackage{mathpazo}

\usepackage{geometry}

\geometry{a4paper, top=20mm, left=15mm, right=15mm, bottom=20mm,
headsep=5mm, footskip=12mm}


\pagenumbering{gobble}

\title{\vspace{-1.25cm}SEH summary\vspace{-0.25cm}}
\date{\vspace{-5ex}}

\newcommand*{\Z}{\mathbb{Z}}

\begin{document}
	\maketitle


	\section{Introduction}
		\begin{itemize}
			\item Dependable systems
				\begin{itemize}
					\item Attributes / Goals (what)
						\begin{itemize}
							\item Confidentiality (no unauthorized disclosure)
							\item Integrity (absence of improper states)
							\item Availability (possibility in nines)
							\item Reliability (mean time to failure)
							\item Safety (consequences for user)
							\item Maintainability
						\end{itemize}
					\item Means (how)
						\begin{itemize}
							\item Fault Prevention
							\item Fault Tolerance
							\item Fault Removal
							\item Fault Forecasting
						\end{itemize}
					\item Threats (against)
						\begin{itemize}
							\item Faults
							\item Errors (state)
							\item Failures (interface behavior)
						\end{itemize}
				\end{itemize}
			\item Risk management
			\item Safety (protect people) vs. Security (CIA)
			\item Authentication (integrity of source, matching identity with entity)
			\item Non-reputability (prevent denial)
			\item Anonymity
		\end{itemize}


	\section{Attack Scenarios in General}
		\begin{itemize}
			\item Types of flaws (security faults)
				\begin{itemize}
					\item intentional malicious (trojan)
					\item intentional non-malicious (debug interface)
					\item inadvertent (software bug)
				\end{itemize}
			\item Attacker classes
				\begin{itemize}
					\item Clever outsider (low resources)
					\item Knowledgeable Insiders (medium resources, big money)
					\item Funded Organizations (high, mafia, also non-technical)
				\end{itemize}
			\item Attack categories
				\begin{itemize}
					\item inv. log.: code injection
					\item inv. phys.: damaging, microprobing
					\item non. log.: phish, crypto, copying
					\item non. phys.: eavesdropping, side-channel
				\end{itemize}
			\item Attacks and Countermeasures on different hw / sw abstraction levels
			\item Attack scenarios
				\begin{itemize}
					\item Malicious software / firmware update
					\item Malicious program installation
					\item Reverse engineering (implementation details, vulnerabilities, IP)
					\item Modifying integrated circuit (read memory by cutting jump signal and bus probing)
					\item Phishing (social)
					\item Forged authenticity (DNS / cert spoof, systems)
					\item Crypto attacks
					\item Eavesdropping (prevented by encryption)
					\item Side-channel attacks (timing, power, fault)
				\end{itemize}
		\end{itemize}

	\section{Attack Scenarios Cryptography}
		\begin{itemize}
			\item Cryptography (designing) vs. Cryptoanalysis (breaking) vs. Cryptology (both)
			\item Block ciphers vs. Stream ciphers
			\item Monoalphabetic substitution ciphers (bad because of probability)
			\item Vigenère (key repeatedly added to plain)
			\item One-Time Pad (same length as plain)
			\item symmetric vs. asymmetric methods
			\item DES
				\begin{itemize}
					\item 56 bit key, 64 bit block
					\item 16 Feistel rounds: expansion, substitution, permutation
					\item Brute forcable (Deep Crack, COPACOBANA: 65 billion key/sec) $\Rightarrow$ max 12 days
					\item Triple-DES (enc,dec,enc) $\Rightarrow$ 168 (eff:112) bit key
				\end{itemize}
			\item AES (Rijndael)
				\begin{itemize}
					\item 128 (10) / 192 (12) / 256 (14) bit keys (rounds)
					\item 128 bit block
					\item 256 byte S-Box
					\item Key Expansion, Initial Round, Rounds, Final Round
					\item Round (work on square state)
						\begin{itemize}
							\item SubByte (S-Box)
							\item ShiftRows (circular, increasing)
							\item MixColumn (polynom)
							\item AddRoundKey
						\end{itemize}
				\end{itemize}
			\item RSA
				\begin{itemize}
					\item factorizing as one way function
					\item Large primes $p$ and $q$
					\item $N_1 = p*q$, $N_2=(p-1)*(q-1)$
					\item $e$ such that $gcd(N_2,e) = 1$ and $0<e<N_2$
					\item $d$ such that $d*e\;mod\;N_2=1$
					\item Enc: $c\equiv m^e\;mod\;N_1$
					\item Dec: $m \equiv c^d\;mod\;N_1$
					\item DES 100 (sw) / 10000 (hw) times faster
					\item Used to encrypt session key for symmetric methods
				\end{itemize}
		\end{itemize}

	\section{Code Injection Attacks}
		\begin{itemize}
			\item Von Neumann concepts
				\begin{itemize}
					\item memory, control unit (control path), arithmetic and logic unit  (data path)
					\item Instruction fetch, decode, read operand, execute, write result
					\item Sequential execution vs. Pipelining
					\item Leon3 as example: 7 stages pipeline, MMU, seperate caches
				\end{itemize}
			\item Code Injection Attacks in general
				\begin{itemize}
					\item Insert routine (mostly embedded into input) and finding its address
					\item Manipulating program flow to execute it (overflows, return into libc, pointer)
				\end{itemize}
			\item Control Flow Graph
				\begin{itemize}
					\item Basic Block: sequence without jumps or branches
					\item Beginnings: first instruction, jump targets, instructions after jump / branch
					\item Control Flow Graph CFG(BB,T), where BB basic blocks and T transitions
				\end{itemize}
			\item Stack-based Buffer Overflows
				\begin{itemize}
					\item Return address first entry after function call, then base pointer
					\item Write (shell)code into buffer and overwrite return pointer
					\item Overwriting the framepointer also can do the job
				\end{itemize}
			\item Heap-based Buffer Overflows
				\begin{itemize}
					\item Unlink macro merges boarded segments
					\item next-Pointer can be overwritten with 'return - 3' and prev with beginning of code
					\item Unlinking will overwrite the return with the beginning of the code
				\end{itemize}
			\item Arc injection - ret2libc / ROP
				\begin{itemize}
					\item Inserting new control flow transpher
					\item Redirect return to libc function call (system)
					\item Returning to gadgets with returns (followed by their args)
					\item Gadgets are found in libraries and through miss-alignment
				\end{itemize}
			\item Pointer Subterfuge
				\begin{itemize}
					\item Function pointer clobbering: modify function pointer of virtual function
					\item Data pointer manipulation: arbitrary memory writes
					\item Exception handler hijacking: overwrite list of registered exception handlers
					\item Virtual pointer smashing: modifying virtual pointer to point to own virtual function table
				\end{itemize}
			\item Hardware Code Injection
				\begin{itemize}
					\item Partial bitfiles in FPGAs / Trojan cores in ASICs
					\item Adding a malicious core to gather data / do evil shit
				\end{itemize}
			\item Countermeasures
				\begin{itemize}
					\item Additional Return Stack
						\begin{itemize}
							\item Additional hardware return stack (SRAS)
							\item Problems: multi-tasking, interrupts, setjmp / longjmp
							\item Solutions: additional instructions, SmashGuard searches for longjmp returns
						\end{itemize}
					\item Address Obfuscation
						\begin{itemize}
							\item Randomize stack and heap addresses and location of shared libs / static data
							\item XORing function pointers with randomly assigned key
						\end{itemize}
					\item Software Encryption
						\begin{itemize}
							\item Instructions are stored encrypted
							\item During fetch they get decoded (each process has own key)
						\end{itemize}
					\item Safe Languages
						\begin{itemize}
							\item Cyclone: static and dynamic checks, avoidsoverflow and format string
							\item CCured (source-source translator): static + dynamic, need library wrappers
							\item Static analyzers: Splint, PREfix, ITS4 (heuristics and annotations)
							\item Dynamic Code analyzers: Rational Purify, STOBO, FIST (test pot. malicious inputs)
						\end{itemize}
					\item Anomaly Detection
						\begin{itemize}
							\item Compare behavior to profile (defined vs. learned)
							\item Approach: monitor syscalls
						\end{itemize}
					\item Compiler Support
						\begin{itemize}
							\item StackGuard: canary word between vars and return address
							\item ProPolice: also protects registers (SSP)
							\item Stack Shield: redundand return address in data segment
							\item Return Address Defender: copies of return pointers stored in data segment, surrounded by mines
							\item PointGuard: pointers are stored encrypted
							\item XFI: static analysis and inline guards through binary rewriting, dual stacks
							\item Out-of-bound pointers: redirected to out of bounds object
						\end{itemize}
					\item Library Support
						\begin{itemize}
							\item Use of safer functions (strlcpy, SafeStr lib, FormatGuard lib)
							\item Problem modification of code
							\item libsafe and libverify: transparent dynamic link libs
							\item libsafe intercepts vulnerable calls and replaces them
							\item libverify uses binary rewriting to verify stack pointer (also canaries)
							\item ContraPolice: modifies alloc and free of heap with canaries
						\end{itemize}
					\item OS Support
						\begin{itemize}
							\item Marking stack non-executable (NX flag)
							\item Kernel extension PaX and Exec Shield
							\item $W\oplus X$ protection
							\item Drawbacks: functional languages, ROP
						\end{itemize}
					\item Control Flow Checking
						\begin{itemize}
							\item Checking sequences of program counter values
							\item Software Implemented Hardware Fault Tolerance (SIHFT)
							\item Control blocks between basic blocks preventing wrong jumps
							\item Watchdog Processor: program derived from cf-graph or function-graph
							\item Uses signatures (e.g. start address) for each node
							\item Embedded (ESM) vs. autonomous (ASM) Signature Monitoring
							\item ESM: signatures stored between basic blocks and transfered while execution
							\item ASM: signatures stored in watchdog, cfg mapped to control program
						\end{itemize}
					\item ASM approach
						\begin{itemize}
							\item Needed: current pc, next pc, correct jump targets + sources
							\item direct jumps: source + target, indirect: only source, interrupts: only target
							\item Control Flow Method (CF): basic blocks as nodes
							\item CF Instruction Method (CFI): cf instructions as nodes
							\item Components: Memory, CF-Checker, Extensions (modular concept)
							\item Small area overhead, fast error detection, no performance impact
						\end{itemize}
				\end{itemize}

		\end{itemize}

	\section{Invasive physical attacks}
		\begin{itemize}
			\item Compromised security goals
				\begin{itemize}
					\item Availability: damaging, cutting wires
					\item Confidentiality: recover sensitive data / implementation details
					\item Integrity: Manipulation of circuits / data
				\end{itemize}
			\item Attacks need special equipment and time
			\item Depacking
				\begin{itemize}
					\item Opening chip package or removing plastic card
					\item Heating and use of chemicals
				\end{itemize}
			\item Layer Reconstruction
				\begin{itemize}
					\item Creating map of the chip
					\item Using optical microscope to get mosaics of chip surface
					\item Stripping of metal layers with acid
					\item Standard architectures / cells help identifying components
					\item Reading out ROM contents
					\item Constructing a netlist for understanding and finding weaknesses
					\item Layout can be used to clone product
				\end{itemize}
			\item Manual Micro-probing
				\begin{itemize}
					\item Probe shaft thickness about $10\mu m$, tip $< 0.1 \mu m$
					\item Recording values / overwriting signals
					\item Removing the passivation layer with a laser cutter
				\end{itemize}
			\item Advanced Beam Technologies
				\begin{itemize}
					\item Focused Ion Beam
						\begin{itemize}
							\item Vacuum chamber and particle gun (gallium)
							\item Scanning structures down to 5nm
							\item Detecting secondary particles (ions and electrons)
							\item Removing material with higher currents
							\item Gas-assisted etch: up to 12 times deeper
							\item Navigate using laser interferometer stages
							\item Microprobing deeper signal lines by drilling holes and creating pads
						\end{itemize}
					\item Electron Beam Tester
						\begin{itemize}
							\item Scanning Electron Microscope with voltage-contrast function
							\item Energy of secondary electrons corresponds to electric field
							\item Observing signals if clock is below 100 kHz
						\end{itemize}
					\item Infrared Laser
						\begin{itemize}
							\item Scanning with infrared laser from rear (silicon transparent)
							\item Photon current corresponds to internal states of transistors
						\end{itemize}
				\end{itemize}
			\item Key and Memory Reading
				\begin{itemize}
					\item Reading out ROM contents (layer reconstr., EBT)
					\item Reading RAM content by bus probing (micro porbe, EBT)
					\item Replaying memory transactions by overwriting signals
					\item Checking checksums leaks information
					\item Generating linear memory trace by cutting jump / branch signals
					\item Key Retrieval using Test Modes: Re-enable test routines by bridging fuses (probe)
					\item KR using ROM Overwriting
						\begin{itemize}
							\item Setting bits with FIB or laser cutter
							\item Alternation of instructions to disclosure the key
							\item Reduce rounds or alter S-Boxes
						\end{itemize}
					\item KR using EEPROM overwriting
						\begin{itemize}
							\item (Un)setting easier than reading
							\item Setting key bits via micro-probing and checking output
						\end{itemize}
					\item KR using Gate Destruction
						\begin{itemize}
							\item Harm gates in registers to produce constant bits
							\item DES: rounds through shifting
							\item If one bit is stuck, key bits can be recovered
						\end{itemize}
					\item KR using Memory Remanence
						\begin{itemize}
							\item DRAM and SRAM keep content after power down for short time
							\item Increased by cooling
						\end{itemize}
					\item KR by Probing Single Bus Bits
						\begin{itemize}
							\item Reading address bit by micro-probing or EBT
							\item Recoverable: RSA exponents, DES key bits with different ciphers
						\end{itemize}
				\end{itemize}
			\item Countermeasures
				\begin{itemize}
					\item Robust Low-frequency Sensor
						\begin{itemize}
							\item Bus observation by EBT requires low frequency
							\item Sensor activates reset on low frequencies
							\item Problems: sensor can be destroyed, clock under control of attacker
							\item Solution: only lower reset after sensor self test, circuit works only with reset
						\end{itemize}
					\item Destruction of Test Circuitry
						\begin{itemize}
							\item Fuses can be reconnected by attacker
							\item Better place test circuitry between chips so they get removed entirely
						\end{itemize}
					\item Restricted Program Counter
						\begin{itemize}
							\item Watchdog resets PC if no jump for long time
							\item Problem: area, can be disabled with FIB
							\item Replace PC with segment register S and offset counter O
							\item If O overflows reset; switching S through jumps / branches
						\end{itemize}
					\item Top-Layer Sensor Meshes
						\begin{itemize}
							\item Additional metalization layer that is monitored for interrupts
							\item Destruction triggers countermeasures
							\item Makes preparation for layer reconstruction and microprobing harder
							\item FIB still possible and can create a pad for microprobing
						\end{itemize}
					\item IP Watermarking in layout / higher levels
				\end{itemize}

		\end{itemize}

	\section{Non invasive logical attacks}
		\begin{itemize}
			\item Gathering information without altering the system
			\item Authenticity Forging
				\begin{itemize}
					\item Phishing
						\begin{itemize}
							\item Uses social engineering to gain trust
							\item Fake Webistes, Emails, IMs, SMS
						\end{itemize}
					\item DNS spoofing
					\item Certificate Spoofing
				\end{itemize}
			\item Exploiting cryptographic weaknesses
				\begin{itemize}
					\item Too small keys can be cracked by brute force (e.g. DES)
					\item Wrong cryptographic design (WEP)
					\item Stream cipher simulates one-time pad by generating a keystream
					\item WEP: 40 bit (easy) vs. 104 bit (no bruteforce) keysize
					\item Wireless: Frames get lost and PRNG restarted for each frame
					\item Small key size because of 24 bits IV (also reused)
				\end{itemize}
			\item Copying
				\begin{itemize}
					\item Copying of sensitive data, e.g. Intellectual Property (IP)
					\item Confidentiality compromised
					\item Copying of Programs / IP Cores is easy
					\item Register Transfer Level (hdl), Logic Level (netlist), Ciruit Level (bitfile)
				\end{itemize}
			\item Overview of countermeasures
				\begin{itemize}
					\item Social protection (law, copyrigth)
					\item Reactive protection (no prevention but detection)
					\item Active protection (physical / crypto, highest deterrent degree)
					\item Social protection sometimes hard for smaller companies
				\end{itemize}
			\item Encryption (active)
				\begin{itemize}
					\item Instructions stored encrypted
					\item Decrypt during fetch (HW) or while loading into memory
					\item Goals: prevent unauthorized use or modification
					\item Encrypted HDL / netlists: design tools decrypt before usage
						\begin{itemize}
							\item Problem: different EDA tools must enrypt result again
							\item Protection of EDA tools against read-out attacks
							\item No industrial standard for encrypted IP core
							\item Keyhandling
								\begin{itemize}
									\item Symmetric (AES): tools must know different vendor keys; if one key is 
										cracked, all connected cores lose protection
									\item Asymmetric: keys enrypted with EDAs public key; seperate versions for each 
										tool, much time
									\item Hybrid: cores encrypted symmetric, key encrypted with EDA pub key
								\end{itemize}
							\item Disadvantages
								\begin{itemize}
									\item Fixed constraints, restricted choice of tools
									\item slower simulation, no re-use of cores
								\end{itemize}
						\end{itemize}
					\item Encryption of FPGA file: bitstream decrypted inside the FPGA
						\begin{itemize}
							\item Bitfile stored in non-volatile memory (PROM) and tranferred encrypted
							\item FPGA decrypts with a hard core before configuring
							\item Key storage
								\begin{itemize}
									\item Volatile: low power SRAM with external battery (more comlicated attack, 
										but more space and cost)
									\item Non-Volatile: flash / EEPROM, no ext. device but needs latest CMOS with 
										increased costs
								\end{itemize}
							\item Key Management
								\begin{itemize}
									\item Embedded Key: FPGA loads enc. bitfile from PROM and decrypts it at power on
									\item Encrypted Partial Bitfiles: decryption done by a statically loaded soft 
										core
								\end{itemize}
						\end{itemize}
				\end{itemize}
			\item Unique Identification of the Execution Site (active)
				\begin{itemize}
					\item Software / IP Core checks ID of board
					\item Only executes or loads if ID is correct
					\item Identification through USB Dongle, MAC or TPM
					\item TPM: endorsement key in persistent memory can bind software to host
					\item FPGA identification through external device, unique embedded key or physical unclonable function (PUF)
					\item CPLD can store board key and crypto for challenge-response auth.
					\item Unique keys (device DNA) in FPGAs
					\item PUFs
						\begin{itemize}
							\item Returns unique value, extracted by physical properties
							\item Silicon PUFs use manufacturing related delay variations
							\item Categories
								\begin{itemize} 
									\item Ring oscillator: edge detection and counter
									\item Arbiter: challenge vector is road through multiplexers, clock registers 
										data bit 
									\item RAM: during power up SRAM cells create pattern which can be used
								\end{itemize}
						\end{itemize}
				\end{itemize}
			\item Watermarking (reactive)
				\begin{itemize}
					\item Multimedia watermarking alters data slightly
					\item IP watermarking must no alter user data (functionality)
					\item Fingerprint is a watermark that is individual for each copy
					\item Attacks
						\begin{itemize}
							\item Ambiguity attack: multiple watermarks in design
							\item Removal attack
							\item Copy attack: copying work without permission
							\item Key copy attack: exchange watermark (signature)
						\end{itemize}
					\item Should be functional correct, easily verifiable, difficult to remove
					\item Additive vs. constraint-based
					\item Can be used for HDL, netlists, bitfiles
					\item Additive
						\begin{itemize}
							\item Not embedded into functionality but can be masked
							\item HDL: difficult because its human readable, stored in unused space or sent during testmodes
							\item Bitfile: encoded into unused LUTs, location can be used as fingerprint, number of LUTs in 
								2x2 tile or stored in unused multiplexers
							\item Can be removed by searching for unusual connections
						\end{itemize}
					\item Constraint-based
						\begin{itemize}
							\item Optimization constraints characterize set of feasible solutions
							\item Additional constraints for the syntheses tool as watermark
							\item HDL: exploit scan chain, for DSP alter the result slightly
							\item Netlists: restrict number of used inputs of LUTs, rewiring through redundant nets
							\item Bitfile: placement (odd / even rows), routing, timing (split paths)
							\item Layouts: number of vias and bends, size of repeaters
						\end{itemize}
					\item Constraints only work as is (not on another level, embedded or re-routed) and can be ambiguous
				\end{itemize}
			\item Watermark verification
				\begin{itemize}
					\item Temperature watermark (up to 15 minutes to verify)
					\item Netlist identification
						\begin{itemize}
							\item Extraction of LUT content and checking if its a subset
							\item Problem: further optimization might change LUT contents
							\item Solution: Prevent optimization by e.g. converting LUTs into shift registers
						\end{itemize}
					\item Power Watermarking (netlist level)
						\begin{itemize}
							\item Measuring power consumption of the FPGA
							\item Watermark ist produced by additional consumer (e.g. shift register)
							\item Two shift registers: one for signature, (large) one as consumer
							\item If curr. signature bit is 1, the large one gets shifted
							\item Different Encoding Methods: basic, enhanced robustness, BPSK, correlation
							\item Transfer rate up to 500 kbit/s
							\item Signatures can also be multiplexed
						\end{itemize}
				\end{itemize}

		\end{itemize}

	\section{Non invasive physical attacks}
		\begin{itemize}
			\item Overview
				\begin{itemize}
					\item Eavesdropping: gathering sensitive information
					\item Solution: (proper) Encryption
					\item Side-Channel Attacks: attacker has control over power and clock lines
					\item Active (inducing faults) vs. passive (only observe)
				\end{itemize}
			\item Timing Side-Channel Attacks
				\begin{itemize}
					\item E.g. strcmp execution time depends on number of matching chars
					\item Measuring the longest execution time reveals the right digits
					\item Software Data-Dependencies: no conditional branches that depend on secret data
					\item DDs in Hardware: hardware (functional units or caches) can cause DDs
					\item DDs of Functional Units
						\begin{itemize}
							\item Mostly in multiplication and division
							\item Some use early termination depending on size of source
							\item Div terminates earlier if dividend and divisor are similar size
							\item Conditional branches take different amount of cycles
							\item Load/Store take two for the first access then one for each additional
						\end{itemize}
					\item DDs caused by the cache
						\begin{itemize}
							\item Two consecutive lookups take less time if the same address is accessed
							\item This can be used to gather secret information
						\end{itemize}
				\end{itemize}
			\item Timing Attack on simple AES
				\begin{itemize}
					\item SubBytes usually implemented using lookup tables $\Rightarrow$ timing attack
					\item MixColumns performs additional xor if MSB was set before shift $\Rightarrow$ timing attack
					\item Attack Setting
						\begin{itemize}
							\item AES encrypts $D$ random data blocks
							\item Attacker knows plaintexts and is able to observe overall timing for each encrypt
							\item Each byte: xor with key, s-box, multiplied inside MixColumn
							\item Depending on MSB of s-box output the encryption takes 1 cycle longer or not
							\item Generate s-box bytes for every data-byte xor key-byte combination
							\item For each key candidate calculate avg. 1 and 0 time and select the one with biggest diff.
						\end{itemize}
				\end{itemize}
			\item Power Side-Channel Attacks
				\begin{itemize}
					\item CMOS consists of PMOS (low) and NMOS (high)
					\item Gates consist of pull-up (PMOS) and pull-down (NMOS)
					\item Leakage current is data-dependent
					\item Short circuit current: when input switches both networks are conductive for a short time
					\item Charging current: When output switches capacitance needs to be charged, leading to a higher data-dependent 
						current
					\item Setup needs a clock generator, measurement circuit and osciloscope
					\item Measurement throught small resistor, current probe or electromagnetic probes
					\item First step is to find interesting parts by visual inspection of the power trace
					\item Attacker learns operation-dependent power consumption and thus sequence of instructions
					\item Gross changes in power profile: cache hit/miss, memory access, peripherals (e.g. coprocessors)
					\item Detection of repeated patterns and variations
					\item Differential Power analysis on AES
						\begin{itemize}
							\item Measuring power consumption based on AES output
							\item Each 1000 trace where MSB of byte 1 of cypher is 1 / 0
							\item Calculate mean and difference between these values
							\item Very effective against block ciphers like AES
							\item Power consumption depends on intermediate values of crypto algs.
							\item For every time instance and every key hyp. calculate difference between high and low means
							\item Result is matrix of size \#hyps * length of trace
							\item Only at the right time with right hyp. greater power differences are observable 
								(max of matrix)
						\end{itemize}
					\item Countermeasures
						\begin{itemize}
							\item Switch key frequently (protocol level countermeasure)
							\item Make power consumption independent of intermediate values (random or equal for each cycle)
						\end{itemize}
					\item Hiding
						\begin{itemize}
							\item Same calculations but reduction of data-dependent power consumption
							\item Lowering signal-to-noise ratio (adding noise) increases number of needed measurements
							\item Random delays add propability p of executing a command at a certain time; measurements 
								increase with square of p
							\item Choice of instructions, noise engines, dual-rail pre-charge
						\end{itemize}
					\item Masking
						\begin{itemize}
							\item Randomize intermediate values so the attacker cant exploit the leakage
							\item Typically values are xored with random mask
							\item Masked software / hardware implementation, masked logic styles
						\end{itemize}
				\end{itemize}
			\item Fault Analysis
				\begin{itemize}
					\item Inducing faults through temperature, frequency, supply voltage, ...
					\item Spike / Glitch Attacks: disturb power supply or I/O lines
					\item Light Attacks: switching transistors with light; local with lasers or global
					\item Countermeasure approaches: sensor-based (detect induction) vs. error-detection 
					\item If attacker can precisely set bits he can set every key bit to 0 and check new result
					\item Fault Attacks on AES
						\begin{itemize}
							\item Changing input of AddRoundKey with different fault models
								\begin{enumerate}
									\item attacker can toggle bits $\Rightarrow$ no info
									\item set certain bits to certain values $\Rightarrow$ yes
									\item set bits to unknown values $\Rightarrow$ no info
								\end{enumerate}
							\item Same for input of ShiftRows
							\item Attacking input of SubByte with fault model 1 is possible
								\begin{itemize}
									\item Able to toggle LSB of s-box input
									\item Generate pairs of correct and faulty encryptions for each plaintext
									\item Check for all possible keybytes for which the s-box input only differs 
										in the lsb
								\end{itemize}
						\end{itemize}
				\end{itemize}
		\end{itemize}

	\section{Automotive case study}
		\begin{itemize}
			\item Current systems mainly focus on safety but not security
			\item Constraints: weight, power consumption, cost, reliability
			\item Confidentiality
				\begin{itemize}
					\item Disclosure of secret info (keys) / data
					\item Access to system routines, information about the state of the system
				\end{itemize}
			\item Integrity
				\begin{itemize}
					\item Unauthorized Updates
					\item Addition of deficient components
					\item Manipulation of data and parameters
				\end{itemize}
			\item Availability
				\begin{itemize}
					\item Damage or deactivation of vehicle / routines
					\item Disabling or overloading buses
				\end{itemize}
			\item Envolved entities
				\begin{itemize}
					\item Vehicle Owner: tuning or improving guarantee claims
					\item Vehicle manufacturer: reverse engineering of competing products
					\item Component manufacturer: spying other components or offer cloned products
					\item Software provider: Trojans / Viruses
					\item Workshop staff: inserting unsupported modules (tuning), obfuscate faults
					\item Constitutional State: Provide Logs to identify guilty parties
					\item Unknown third party: disclosure of sensitive data, stealing car, assault on occupants
				\end{itemize}
			\item Gateways
				\begin{itemize}
					\item Physical: On Board Diagnostic, USB / Memory Card, wiring (mirrors)
					\item Non-physical: wireless interface, software update, appstores
				\end{itemize}
			\item Attacks on Communication Networks
				\begin{itemize}
					\item Physical: Disconnecting / eavesdropping of cables
					\item Protocol Attacks: inserting / modifying / logging of bus messages, DOS
					\item Network Topology Attack: creating new gateways by modifying ECUs that are connected to multiple buses, 
						bypassing security functions
				\end{itemize}
			\item Protection against Code-Injection
				\begin{itemize}
					\item Additional return stack in hw / sw
					\item Address obfuscation and software encryption
					\item Secure programming languages
					\item Static / dynamic code analyzer
					\item Anomaly detection
					\item Compiler / Library / OS Extensions
				\end{itemize}
			\item Security Zones
				\begin{itemize}
					\item Subdividing network in different zones with different trustworthiness
					\item A Bus can only be assigned to a single zone
					\item Encrypted channels and tunnels across bus boundaries
					\item Highest rust zone uses encryption, authentication and secure gateways
				\end{itemize}
			\item Physical protection
				\begin{itemize}
					\item Make trustworthy networks hard to reach
					\item Switch of bus segment if manipulation is detected
					\item Implementation as System-on-Chip
					\item Destruction of ECUs when opened
				\end{itemize}
			\item ECU authentication through (double) challenge-response
			\item Authorization rules for each ECU (ignoring non authorized messages)
			\item Intrusion detection
				\begin{itemize}
					\item Network protocol: additional nodes analyze traffic
					\item Physical: measurement instruments
					\item Deny access to sensitive data, disconnect bus segments, issue warning
				\end{itemize}
			\item Conservation of attacks evidence through Event Data Recorders (EDR)
		\end{itemize}


\end{document}
