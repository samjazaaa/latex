\documentclass[11pt, paper=a4, twocolumn]{scrartcl}

\usepackage[ngerman]{babel}
\usepackage[utf8]{inputenc}

\usepackage[T1]{fontenc}
\usepackage{mathpazo}

\usepackage{geometry}

\usepackage{mathabx}

\geometry{a4paper, top=20mm, left=15mm, right=15mm, bottom=20mm,
headsep=5mm, footskip=12mm}


\pagenumbering{gobble}

\title{\vspace{-1.25cm}Zusammenfassung Klinische Psychologie\vspace{-0.25cm}}
\date{\vspace{-5ex}}

\newcommand*{\Z}{\mathbb{Z}}

\begin{document}
	\maketitle


	\section{Schizophrenie}
		\begin{itemize}
			\item 
		\end{itemize}

	\section{Somatoforme Störungen}
		\begin{itemize}
			\item 
		\end{itemize}

	\section{Rahmenbedingungen von Psychotherapie}
		\begin{itemize}
			\item Definitionen
				\begin{itemize}
					\item Wissenschaftlich anerkannte therapeutische Verfahren
					\item Feststellung, Heilung oder Linderung von Störungen mit Krankheitswert
					\item Interaktionaler Prozess mit psychologischen Mitteln
				\end{itemize}
			\item Rechtliche Rahmenbedingungen
				\begin{itemize}
					\item Psychotherapeutengesetz: Approbation nötig, regelt Voraussetzungen
					\item Gesetzliche Leistungserstattung nur mit Kassenzulassung
				\end{itemize}
			\item Ausbildung zum Psychologischem Psychotherapeuten (PPT)
				\begin{itemize}
					\item 3 / 5 Jahre
					\item Praktische Tätigkeit: Psychiatriejahr (Diagnostik und Behandlung) + Behandlungsstunden
					\item Theoretische Fortbildung: Vertiefung und praktische Übung
					\item Selbsterfahrung: eigenes Verhalten, Denken, Fühlen überprüfen und pos. verändern
					\item Supervision: Identifizieren und Üben von Optimierungsmöglichkeiten
				\end{itemize}
			\item Antragspflicht: notwendig, zweckmäßig und wirtschaftlich
			\item Gutachterverfahren für LZT
				\begin{itemize}
					\item Liegt Krankheit vor?
					\item Psychotherapie indiziert?
					\item Geplante Therapie zweckmäßig, wirtschaftlich und hinreichend gute Prognose?
					\item Rahmenbedingungen eingehalten?
				\end{itemize}
			\item Gliederung: Beschwerden, Vorgeschichte, psychische / somatische Symptomatik, Störungsmodell, ICD-10, Ziele, Therapieplan
			\item Gutachter gibt Empfehlung, Kasse entscheidet
			\item Stundenkontingente\\
				\begin{tabular}{c|cccc}
					& kurz & lang & max & $\diameter$ (Schulz) \\
					\hline Verhaltensth & 24 & 60 & 80 & 38.6 \\
					Tiefenpsych. & 24 & 60 & 100 & 53.7 \\
					Psychoanalyse & - & 160 & 300 & 106.5
				\end{tabular}
			\item Therapeutische Versorgung
				\begin{itemize}
					\item Ambulant (Praxis, Ambulanzen, Tageskliniken)
					\item Stationär (Psychosomatik, Rehabilitation, Psychiatrie)
					\item 32\% psychische Störungen, davon 63\% nicht versorgt, 18\% der Versorgten bei PPT
					\item Bei Depressionen ca. 50\%
					\item Starker Ost-West Unterschied
					\item Versorgangsgrad ca. 46\%
					\item AU-Tage steigen konstant (11-69\%), Frauen höher als Männer
				\end{itemize}
			\item Ansatz für Approbation nach Studiumsabschluss in Diskussion
			\item Rechtliche Rahmenbedingungen
				\begin{itemize}
					\item Ausbildungs- und Prüfungsverordnung
					\item Berufsordnung des jeweiligen Bundeslands
					\item Landespsychotherapeutenkammer (verfasst Berufsordnung)
					\item Bundespsychotherapeutenkammer
					\item Kassenärztliche Vereinigung
					\item Gemeinsamer Bundesausschuss (erarbeitet PT-Richtlinien)
					\item Schweigepflicht und Datenschutz
				\end{itemize}
		\end{itemize}

	\section{Therapiemotivation}
		\begin{itemize}
			\item Therapiemotivation vs. Veränderungsmotivation
			\item Misserfolg $\leftrightarrow$ Motivationsbeeinträchtigung
			\item Duales Therapiemodell nach Schulte und Eifert
				\begin{itemize}
					\item Methodenstrang
						\begin{itemize}
							\item Identifizieren von Bedingungen
							\item Auswahl von Strategien und Techniken
							\item Umsetzung
						\end{itemize}
					\item Motivationsstrang
						\begin{itemize}
							\item Prüfen von Therapie- / Veränderungsmotivation
							\item Identifizieren von motivationalen Problemen
							\item Einsatz von Methoden zur Motivationsstärkung
						\end{itemize}
				\end{itemize}
			\item Motivationsstärke = ((Nutzen - Kosten) * Durchführbarkeit) + Entschluss der Ausführung
			\item Präskriptives Entschlussmodell
				\begin{itemize}
					\item Option gut genug? Ja $\rightarrow$ Dominanzbildung, Planung, Handlung
					\item Nein $\rightarrow$ suche nach neuer Opt. und Antizipation aversiver Konsequenzen
					\item Nein + emotionale Selbsteffizienz und Einsicht in Ausschließlichkeit, Opfernotwendigkeit 
						$\rightarrow$ Option beste verfügbare?
					\item Ja $\rightarrow$ Dissonanzreduktion (Rechtfertigung durch Langzeitpräferenzen, kognitive Repräsentation) 
						$\rightarrow$ Handlung
				\end{itemize}
			\item Vor- und Nachteilsanalysen
				\begin{itemize}
					\item Tabellenartige Auflistung für spezifische Situation
					\item Kurzfristig: Vorteile (jemand der zuhört) vs. Nachteile (Aussetzen von Ängsten)
					\item Langfristig: Vorteile (Überwindung von Ängsten) vs. Nachteile (weniger Schonung / Ausreden)
				\end{itemize}
			\item Stadien der Veränderungsbereitschaft
				\begin{itemize}
					\item Precontemplation (Problem nicht erkannt)
					\item Contemplation (Veränderungsabsicht)
					\item Determination / Preparation (Schritte planen)
					\item Action (Veränderung vollziehen)
					\item Maintenance (Aufrechterhalten und Rückfall vermeiden)
					\item Relapse (Rückfall bewältigen)
				\end{itemize}
			\item Motivierende Gesprächsführung
				\begin{itemize}
					\item Annahmen: Autonmieverletzung führt zu Reaktanz, ambivalent motiviert, pos. Motive stärken, neg. 
						wertschätzen
					\item 2 Phasen: Aufbau von Veränderungsbereitschaft $\rightarrow$ Stärkung der Selbstverpflichtung
					\item Prinzipien
						\begin{itemize}
							\item Empathie / Wertschätzung: dysfunktionales Verhalten für legitime Ziele
							\item Herausarbeiten von Diskrepanzen: Konsequenzen vs. Ziele
							\item Geschmeidiger Umgang mit Widerstand
							\item Stärkung der Selbsteffizienz: Zuversicht auf Beeinflussung
						\end{itemize}
					\item Interventionen
						\begin{itemize}
							\item Offene Fragen
							\item Aktives und empathisches Zuhören (Paraphrasieren, Zusammenfassen, usw.)
							\item Würdigung (Anerkennen und wertschätzen ALLER Leistungen)
							\item Informationsvermittlung
							\item Non-Direktivität / geleitetes Entdecken
							\item Förderung von change talk (Verstärken von Aussagen mit Veränderungsabsicht)
							\item Konstruktiver Umgang mit Widerstand (überzogenes Widerspiegeln)
							\item Förderung von Handlungsorientierung (konkrete Ziele, commitment fördern)
						\end{itemize}
				\end{itemize}
			\item EPOS (Elaboration positiver Perspektiven)
				\begin{itemize}
					\item Zwei Sitzungen
					\item Vorstellen eines Tags in 5 Jahren, wenn alles nach seinem Willen gelaufen ist
					\item Offene Fragen, mehrere Sinneskanäle
					\item Audioaufnahme für Patient
					\item Zweite Sitzung: imaginativ-holistisch $\rightarrow$ analytisch-zielgerichtet
					\item Zentrale Bilder und dahinterliegende Wünsche herausarbeiten
					\item Wünsche langfristige Ziele?
					\item $\rightarrow$ mittelfristige, kurzfristige Ziele
				\end{itemize}
		\end{itemize}

	\section{KVT-Intro \& Entspannung}
		\begin{itemize}
			\item KVT
				\begin{itemize}
					\item Empirisch belegte Methoden und Theorien
					\item Gegenwarts-, Handlungs-, Problem- und Zielorientiertheit
				\end{itemize}
			\item PMR
				\begin{itemize}
					\item Psychophysiologischer Entspannungsprozess durch Kontrasterleben
					\item Kurzform vs. Langform
					\item Einsatzmöglichkeiten
						\begin{itemize}
							\item Regelmäßige Entspannung (anhaltende Reduktion der Anspannung)
							\item Angewandte Entspannung (Stressbewältigung)
							\item Systematische Desensibilisierung (stufenweise Konfrontation in sensu, reziproke Inhibition)
						\end{itemize}
					\item Ablauf: Einführung, Verkürzung, ohne Anspannung, Selbstinstruktion, verschiedene Kontexte, schnell, 
						Anwendung
					\item Effektivität
						\begin{itemize}
							\item Durch Studien belegt: Anspannungsreduktion, chronische Schmerzzustände, Hypertonie, 
								Angststörung, Schlafstörung, psychosomatische Beschwerden
							\item System. Desens. nachgewiesen aber nur wenn massive Konfrontation nicht möglich
							\item Effektstärken $d=0.58$
						\end{itemize}
				\end{itemize}
			\item Biofeedback
				\begin{itemize}
					\item Rückmeldung physiologischer Parameter durch Geräte
					\item Patient soll Signale beeinflussen $\rightarrow$ auch psych. Prozesse
					\item Evtl. Vermittlung bestimmter Technik
					\item Lernen Beschwerden direkt (Blutdruck) / indirekt (generelle Entspannung) zu beeinflussen
					\item Beeinflussbar: Muskelspannung, Hautleitfähigkeit, Arteriendurchmesser der Schläfenartherie, usw.
					\item Psychophysiologische Überaktivierung als Störungskorrelat
						\begin{itemize}
							\item Erhöhter Grundtonus und Reaktivität
							\item Verringerte Habituation bei Wiederholung
							\item Verzögerte Erholungsphasen
							\item Erhöhte Dishabituation / Reaktivität
						\end{itemize}
					\item Ablauf
						\begin{itemize}
							\item Identifikation von Zielparametern
							\item Eingangsdiagnostik, Mutlikanal-Messung und Einflussübung
							\item Trainingssitzungen (Baseline, Training, Generalisierung)
							\item Kombination mit anderen Interventionen
							\item Abschluss
						\end{itemize}
					\item Metaanalyse bei Migräne belegt stabilen Effekt von ca. $d=0.56$
					\item Spannungskopfschmerzen auch per Metastudie belegt $d=.82$
				\end{itemize}
			\item Imaginative Entspannung
				\begin{itemize}
					\item Verschiedene Verfahren mit unterschiedlichen Zielen
					\item Z.B. Fantasiereise zur Anregung von individuellen Assoziationen und Interpretationen
					\item Problematisch bei gering ausgeprägtem Vorstellungsvermögen
					\item Keine ausreichenden empirischen Belege
				\end{itemize}
			\item Hypnose
				\begin{itemize}
					\item Trancezustand: minimaler Widerstand / maximiale Akzeptanz
					\item Entspannungselemente aber auch weitere Zielsetzungen (Neubewertung, Veränderung von Reaktionsmustern)
					\item Einengung der Wahrnehmung durch Augenfixation oder Suggestion
					\item Hypnotisierbarkeit als Persönlichkeitsmerkmal
					\item Empirische Fundierung
						\begin{itemize}
							\item Gut bei chronischen Schmerzen
							\item Positiv bei Rauchstopp, Schlafstörungen, Angststörungen
							\item Wenig bei Abhängigkeitserkrankungen und Bluthochdruck
						\end{itemize}
				\end{itemize}
			\item Autogenes Training
				\begin{itemize}
					\item Anwender leitet Entspannungszustand selbst ein
					\item Durchführung
						\begin{itemize}
							\item 6 Übungen (Schwere-, Wärme-, Herz-, Atmungs-, Sonnengeflechts-, Stirnkühleübung)
							\item Formeln für relevante Prozesse zur Beeinflussung (Arme schwer usw.)
						\end{itemize}
					\item Empirische Fundierung
						\begin{itemize}
							\item Weniger eindeutig als PMR
							\item Mittlere Effekte für u.a. Kopfschmerzen, Hypertonie, Schlafstörungen
						\end{itemize}
				\end{itemize}
			\item Indikation
				\begin{itemize}
					\item Angst-, Schlaf- und somatoforme Störungen
					\item Geringe Wahrscheinlichkeit von Nebenwirkungen
				\end{itemize}
			\item Kontraindikationen: Hypotonie (niedriger Blutdruck), Herz- oder Atembeschwerden, akute Psychosen
			\item Probleme
				\begin{itemize}
					\item Sexuelle Missbrauchsopfer (Panikattacken durch Setting)
					\item Panikattacken generell (Vermeidung durch Entspannung)
					\item Angst vor Kontrollverlust
				\end{itemize}
		\end{itemize}

	\section{Konfrontationsverfahren}
		\begin{itemize}
			\item Einteilung
				\begin{itemize}
					\item Graduiert: System. Desensibilisierung (sensu), Habituationstraining (vivo)
					\item Massiert: Implosion (sensu), Flooding (vivo)
				\end{itemize}
			\item Systematische Desensibilisierung
				\begin{itemize}
					\item Prinzip der reziproken Hemmung
					\item Anwendung
						\begin{itemize}
							\item Hierarchisierung der Angstsituationen
							\item Lernen von Entspannung
							\item Vorstellung und Entspannung
							\item Wenn angstfrei Aufstieg
						\end{itemize}
					\item Wirkmechanismus hauptsächlich nur Außeinandersetzung mit Angst, nicht Hierarchie
				\end{itemize}
			\item Durchführung
				\begin{itemize}
					\item Diagnostik, Verhaltens- und Bindungsanalyse
					\item Kognitive Vorbereitung: Störungsmodell $\rightarrow$ therapeutisches Vorgehen
					\item Bewusste und freiwillige Entscheidung mit mental präsentiertem Grund
					\item Intensiv-Expo begleitet
					\item Expo allein
					\item Generalisierungstraining
				\end{itemize}
			\item Begründungsmodelle
				\begin{itemize}
					\item VT-Vermeidungs-Rational: Aufrechterhalten von Angst durch Vermeidung, da keine Habituation
					\item Kogn. Rational I: Falsche Bewertung des Stimulus kann durch Vermeidung nicht geprüft werden
					\item Kogn. Rational II: Angst durch Bewertung $\rightarrow$ bewertungsfrei wahrnehmen
				\end{itemize}
			\item Erwartete Angstverläufe als Vorbereitung
			\item Regeln für Exposition
				\begin{itemize}
					\item Zulassen der Ängste
					\item Verzicht auf Vermeidungsstrategien
					\item Angst beobachten ohne zu bewerten
					\item Achten auf kleine Veränderungen
					\item Unterscheidung Realität $\leftrightarrow$ Phantasie
					\item In Situation bleiben bis Angst abfällt
				\end{itemize}
			\item Therapeutenverhalten
				\begin{itemize}
					\item Angst beobachten und auf Skala angeben lassen (auch körperliche Symptome)
					\item Nach Vermeidungsstrategien fragen
					\item Fragen wie man Angst steigern kann und umsetzen
					\item Hinweise auf Angstabfall beschreiben lassen und versuchen trotzdem zu steigern
				\end{itemize}
			\item Weitere Regeln
				\begin{itemize}
					\item Genügend Zeit, kontrollierbare Situation
					\item Erst bei deutlichem Angstabfall beenden
					\item Konstruktive Nachbewertung
					\item Problem: unangenehme Übung als Beleg für Angst
					\item Expositionsrational: Weg aus Angst durch Angst
					\item Kognitives Rational: Bewertung kann sich nur ändern wenn Chance zu sehen dass Erwartungen nicht eintreffen
					\item Aufstellen und Bewerten von Hypothesen
				\end{itemize}
			\item Massiert vs. graduiert: ähnliche Erfolge aber massiert stabiler
			\item Individualisierte vs. standardisierte Pläne: standard besser, v.a. bei unerfahrenen Therapeuten
			\item Studien zur Panikbehandlung
				\begin{itemize}
					\item Barlow et al.: CBT > CBT + PME > PME
					\item Clark et al.: CBT mit Expo > Antidepressiva > graduierte Expo
					\item Metaanalyse: Konfrontation in vivo > kogn. behav. > kogn. > GT
					\item Rief et al.: Intensiv Expo. effektiver als kürzere
				\end{itemize}
			\item Exposition bei Zwangsstörungen
				\begin{itemize}
					\item Problem: Zwangshandlungen nach Expo als Ausgleich
					\item Lösung: Mehr und längere Expos mit Reaktionsverhinderung
					\item Oft nur bis 50\% Angst
				\end{itemize}
			\item Expo bei PTSD
				\begin{itemize}
					\item Problem: Trauma nicht direkt konfrontierbar
					\item Lösung: In sensu und Trauma-assoziierte Stimuli in vivo
					\item Prolonged Exposure: Imagination, Bericht, Tonbandaufzeichnung, in vivo Expo
					\item Stress-Impfungstraining: Entspannung, Rollenspiele
					\item Beste Ergebnisse bei PE
				\end{itemize}
			\item Expo bei Essstörungen
				\begin{itemize}
					\item Expo mit Körper (Spiegel), Nahrungsmittel $\rightarrow$ Prüfung irrationaler Kognitionen
				\end{itemize}
			\item Indikationen für graduiertes Vorgehen
				\begin{itemize}
					\item Körperliche Komplikationen
					\item Fehlende Motivation
				\end{itemize}
		\end{itemize}

	\section{Operante Verfahren und Modelllernen}
		\begin{itemize}
			\item Operante Verfahren
				\begin{itemize}
					\item Arten operanter Konditionierung
						\begin{itemize}
							\item Positiv: Belohnung (Erhöhung), Bestrafung (Verringerung)
							\item Negativ: Flucht (beenden) / Vermeidung (nicht erfolgen), Time Out / Löschung
						\end{itemize}
					\item Verstärkerarten
						\begin{itemize}
							\item Positive vs. negative Verstärker
							\item Primäre Verstärker: Erfüllen Grundbedürfnisse (abhängig von Sättigung)
							\item Sekundäre (konditionierte) Verstärker: assoziiert mit prim. Verstärker (z.B. Lob, 
								Aufmerksamkeit)
							\item Generalisierte Verstärker: Tauschwert (Token)
							\item Soziale Verstärkung: Gratifikation / Sanktion
							\item Selbstverstärkung: z.B. explizites Selbstlob
						\end{itemize}
					\item Therapeutisches Vorgehen
						\begin{itemize}
							\item Shaping: Aufbau des Zielverhaltens
							\item Chaining: ähnlich zu Shaping aber Verstärkung des Abschlusses, nicht Beginn
							\item Fading: Verstärkung des Zielverhaltens unter schrittweiser Ausblendung der Hilfsstimuli
							\item Prompting: Unterstützung des Beginnns der Verhaltensänderung durch (non)verbale 
								Hilfestellung
							\item Intermittierende Verstärkung: (variables) Festigen bereits gezeigten Verhaltens
							\item Direkte Bestrafung: selten langfristig, negative Auswirkung auf Beziehung
							\item Indirekte Bestrafung: Verstärkerentzug bei dysfunkt. Verhalten
							\item Löschung: Verhindern pos. Konsequenzen von neg. Verhalten
							\item Time-Out
							\item Verhaltens- oder Kontingenzverträge: Verhalten $\rightarrow$ Konsequenz
						\end{itemize}
					\item Stimuluskontrolle durch Hinzufügen / Entfernen von Hinweisreizen
					\item Indikation
						\begin{itemize}
							\item Verstärkungspläne für Depression
							\item Bei Abhängigkeiten, Kindern, Anorexie, Schlafstörungen, Demenz
							\item Bei eingeschränkten kognitiven Fähigkeiten
						\end{itemize}
				\end{itemize}
			\item Modelllernen (vgl. Bandura)
				\begin{itemize}
					\item Therapeutisches Vorgehen
						\begin{itemize}
							\item Modell führt Verhalten aus und wird verstärkt (für Patient sichtbar)
							\item Patient führt Verhalten selbst aus und wird ebenfalls verstärkt
							\item Ähnlichkeit von Modell und Patient wichtig
						\end{itemize}
					\item Effektivität
						\begin{itemize}
							\item Mastery-Modell: Modell problemlos $\rightarrow$ Scham bei Patient
							\item Coping-Modell: Modell überwindet anfängliche Probleme
							\item Transfer in Alltag durch Entkopplung und steigende Einsicht des Patienten
							\item Als Mittel durch empirische Untersuchungen belegt aber keine Belege für eigenständige 
								Methode
						\end{itemize}
				\end{itemize}
			\item Rollenspiele
				\begin{itemize}
					\item Operante Techniken und Modelllernen
					\item Diagnostisches vs. therapeutisches Rollenspiel
					\item Problembeschreibung, Festlegen der Situation, Durchführung, Auswertung, Erneutes Spielen und Auswerten, 
						Transfer
				\end{itemize}
		\end{itemize}

	\section{Psychodynamische Ansätze}
		\begin{itemize}
			\item Geschichte und Grundlagen
				\begin{itemize}
					\item Oberbegriff für Therapieformen die aus Psychoanalyse (Freud) hervorgegangen
					\item Konzept des Unbewussten und Anwendung auf Entwicklung, Pathologie, Therapie und Sozial- / Kulturtheorie
					\item Triebtheorie (anfänglich rekonstruktive Entwicklungstheorie)
				\end{itemize}
			\item Ein-Personen-Psychologie 
				\begin{itemize}
					\item Topographisches Modell
						\begin{itemize}
							\item Unbewusstes (Lustprinzip, Primärvorgänge)
							\item Vorbewusstes
							\item Bewusstes (Realitätsprinzip, Sekundärvorgänge)
						\end{itemize}
					\item 3 Instanzen-Modell
						\begin{itemize}
							\item Erleben, Handeln und Denken von unbewussten dynamischen Kräften beeinflusst
							\item Es: triebhaft, instinktiv, Energiespeicher
							\item Über-Ich: elterliche / gesellschaftliche Werte, Ge- / Verbote, Moral
							\item Ich: Ausgleich und Vermittlung
							\item Triebe: angeboren, Drang, Objektfixierung, somatische Quelle, Triebökonomie
						\end{itemize}
					\item Konflikttypen
						\begin{itemize}
							\item Zwischen versch. Triebobjekten
							\item Verschiedene Wünsche an einem Objekt
							\item Triebabfuhr vs. Unterdrückung
						\end{itemize}
					\item Entwicklungsspezifische Konflikte (Phasen)
						\begin{itemize}
							\item Orale Phase: Mutter, Süchte, Depressionen, Schizophrenie
							\item Anale Phase: Abgabe / Zurückhaltung, Mutter, Zwänge, Perversionen, Persönlichkeitsstörungen
							\item Phallische Phase: Libidinöse Triebbefriedigung, anderes Geschlecht, Neurosen, Ängste, 
								Phobien, Hysterie
						\end{itemize}
					\item Ich-Psychologie
						\begin{itemize}
							\item Ich-Funktionen autonom von Es
							\item Ich - Umwelt: Intention, Motorik, Denken usw.
							\item Ich-Funktionen zur Konfliktbewältigung: Reife vs. Unreife als Abwehrmechan.
						\end{itemize}
				\end{itemize}
			\item Zwei-Personen-Psychologie
				\begin{itemize}
					\item Frühe zwischenmenschliche Beziehungserfahrungen (von Trieben motiviert)
					\item Selbstpsychologie
						\begin{itemize}
							\item Entwicklung der Struktur des Selbst (reflektierter Teil des Ich)
							\item Ziel: narzisstische Homöostase (stabiler Selbstwert durch mütterliche Empathie)
						\end{itemize}
					\item Objektbeziehungstheorie
						\begin{itemize}
							\item Beziehungen als Zentrum seelischen Erlebens
							\item Psychische Strukturen als Ergebnis von internalisierten Objektbeziehungen
							\item Jede reale Begegnung wird verinnerlicht
						\end{itemize}
				\end{itemize}
			\item Intersubjektive Wende
				\begin{itemize}
					\item Selbst als Resultat von wechselseitiger intersubjektiver Bezogenheit
					\item Entsteht im Hier und Jetzt
				\end{itemize}
			\item Gemeinsame Annahmen psychodyn. Verfahren
				\begin{itemize}
					\item Psychischer Determinismus (Beeinflussung von Wahrnehmung durch unbewusste Bedürfnisse)
					\item Einzigartigkeit (gleiches Symptom kann untersch. Gründe haben)
					\item Gesundheit und Pathologie als Kontinuum
					\item Abwehrmechanismen (zur Fernhaltung unangenehmer Gefühle von Bewusstsein)
					\item Innerpsychischer Konflikt (Symptombildung bei mangelnden Abwehrmechan.)
					\item Symptombildung als Lösungsversuch
				\end{itemize}
			\item Neurose und Neurosenstruktur
				\begin{itemize}
					\item William Cullen: funktionelle Erkrankung ohne organische Läsion
					\item Laplanche \& Pontalis: psychogene Affektion, Symptome Ausruck von Konflikt, Wurzeln in Kindheitsgeschichte
					\item Hoffmann \& Hochapfel: überwiegend umweltbedingt, psych., körperl. oder persönl. Störung, Symptome 
						Verarbeitungsversuche infantiler Konflikte
				\end{itemize}
			\item Genetisch-konstitutionelle Aspekte
				\begin{itemize}
					\item Neurotische Störungen überwiegend psychoreaktiv
					\item Symptomwahl genetisch mitbedingt
					\item Einflusstendenz: Zwang > Phobie > Angst > neurot. Depression > Konversionsstörung
				\end{itemize}
			\item Psychoanalytische Krankheitsmodelle
				\begin{itemize}
					\item Konfliktmodell
						\begin{itemize}
							\item Konflikt: Mind. 2 widerstrebende, unvereinbare Tendenzen
							\item Aktueller innerer Konflikt aktiviert ungelösten infantilen Konflikt
							\item Außerdem Angst + Bewältigugnsversuche dieser
							\item Versuche durch infant. Konfl. eingeschränkt
							\item Dadurch massive Angst und neurotische Symptombildung als Abwehrmechan.
							\item Symptom als Kompromiss zw. Wunsch \& Verbot
							\item $\Rightarrow$ Angstminderung und Entlastung
						\end{itemize}
					\item Strukturmodell
						\begin{itemize}
							\item Bezug auf Entwicklungsstufen
							\item Annahme: Strukturdefizit (gestörtes / unterentwickeltes Ich)
							\item Ursache: misslunge Internalisierungserfahrungen als Kind
							\item Interaktionales Austragen von Konflikten
							\item Symptomatik bestimmt durch Eingeschränkte Regulationsfähig. und unterpers. Beziehungen
							\item Abbruch einer Beziehung $\rightarrow$ Regulative Funktion entfällt $\rightarrow$ Angst 
								$\rightarrow$ Symptome $\rightarrow$ Regrisseve Lösungsversuche (Dissoziation) oder 
								Verschiebung / Kompensation (Süchte, Essstörungen)
						\end{itemize}
					\item Traumamodell
						\begin{itemize}
							\item Traumatische Erfahrungen $\rightarrow$ Ausgeliefertsein $\rightarrow$ Sicherheitsverlust 
								$\rightarrow$ Entwicklungsstörungen
							\item Traumatische Erfahrung bleibt abgespalten und unintegriert
							\item Symptome: PTSD
							\item Traumatisierung überfordert Verarbeitungsfähigkeit des Ich $\Rightarrow$ Dissoziation
						\end{itemize}
				\end{itemize}
			\item Diagnostik
				\begin{itemize}
					\item Bevorzugung weniger strukturierter Vorgehensweisen
					\item Patient kann Gesprächsverlauf unbewusst gestalten und Therapeut kann subj. Realität erfassen
					\item Operationalisierte Psychodynamische Diagnostik
						\begin{itemize}
							\item Psychodynamische Ergänzung zu deskriptiven Manualen (IDC-10)
							\item OPD-2 auch zur Therapieplanung, Ressourcenfassung und Veränderungsmessung
						\end{itemize}
					\item 5 Achsen der OPD
						\begin{itemize}
							\item Krankheitserleben und Behandlungsvoraussetzung
							\item Beziehung
							\item Konflikt
								\begin{itemize}
									\item Individuation vs. Abhängigkeit
									\item Unterwerfung vs. Kontrolle
									\item Autarkie vs. Versorgung
									\item Selbstwertkonflikt
									\item Schuldkonflikt
									\item Ödipal-sexuelle Konflikte
									\item Identitätskonflikt
								\end{itemize}
							\item Struktur
							\item Psych. und psychosom. Störungen nach ICD-10
						\end{itemize}
				\end{itemize}
			\item Psychodynamische Therapieverfahren
				\begin{itemize}
					\item Gemeinsame Bestrebungen
						\begin{itemize}
							\item Zusammenhang zw. Erfahrungen und Erleben herstellen
							\item Förderung der Einsicht und Selbstakzeptanz
							\item Aufdeckung der Hintergründe, nicht Symptom selbst
						\end{itemize}
					\item Psychoanalyse
						\begin{itemize}
							\item Frequenz, Dauer und Sitzungsanzahl unbegrenzt
							\item Ziel: Veränderung der Persönlichkeitsstruktur
							\item Basis: intensive emotionale Beziehung zu Analytiker
							\item Aktivierung, Wiedererleben und Korrektur von Konflikten (Übertragung)
							\item Fokus: subjektives Realitätserleben
							\item Liegeposition $\Rightarrow$ Abschirmen von Einflüssen und Verringerung der Selbst- \& 
								Fremdkontrolle
							\item Techniken: Freie Assoziation, Klären, Konfrontieren, Deuten, Übertragung, Gegenübertragung
						\end{itemize}
					\item Analytische Psychotherapie
						\begin{itemize}
							\item Psychoanalyse mit Rahmenbedingungen des Versorgungssystems
							\item Mehr Fokus auf Symptome und Störungen
							\item Strukturelle Änderung aber weniger tiefgreifend
							\item 1-3 Jahre, 80-240 (300) Sitzungen
						\end{itemize}
					\item Tiefenpsychologisch fundierte Psychotherapie
						\begin{itemize}
							\item Fokus: Aktuelle Lebensbelastungen und soziale Beziehungen
							\item Häufigste dyn. Therapieform
							\item 25 bis 100 Sitzungen im Sitzen
							\item Therapeut aktiver (emotional unterstützend, fokussiert)
						\end{itemize}
					\item Psychodynamische Kurzzeittherapie
						\begin{itemize}
							\item Klar definiertes Problem un kurzem festen Zeitraum behandeln
							\item Interpretationen v.a. auf Gegenwart bezogen
							\item 8-25 mal im Sitzen
						\end{itemize}
				\end{itemize}
			\item Empirische Absicherung
				\begin{itemize}
					\item Vorgänge zu komplex um gemessen zu werden
					\item Jedoch zunehmende malualisierung und Annäherung an Wirksamkeitsprüfung
					\item Einzelstudien, jedoch keine emoirische Forschung zu psychodyn. Langzeittherapie
					\item Zwei Studien, jedoch stark kritisierte methodische Qualität und Schlussfolgerungen
					\item Metaanalyse zu Schizophrenie spricht gegen psychodyn. PT im stationären Bereich
					\item Kurzzeittherapie am besten untersucht: große Effekte u.a. bei Depression
				\end{itemize}
		\end{itemize}

	\section{Training emotionaler Kompetenzen}
		\begin{itemize}
			\item Modell des konstruktiven Umgangs mit Gefühlen
				\begin{itemize}
					\item Bewusstes Wahrnehmen $\rightarrow$ Erkennen \& Benennen $\rightarrow$ Analyse der Ursachen
					\item Emotionale Selbstunterstützung unterstütz Analyse und andere Folgeprozesse
					\item a) Finden von Veränderungspunkten $\rightarrow$ Zielgerichtete Modifikation
					\item b) Konstruktive Hoffnungslosigkeit$\rightarrow$ Akzeptanz \& Toleranz
					\item Kompetenzerwerb (a) / Resilienzbildung (b) $\Rightarrow$ Konfrontationsbereitschaft
				\end{itemize}
			\item Gezielte Diagnostik gesundheitsschädlicher Emotionen und der Emotionsregulationskompetenzen
			\item Training emotionaler Kompetenzen (TEK)
				\begin{itemize}
					\item Theorie: Information $\rightarrow$ Orientierung $\rightarrow$ Motivation
					\item Intensives Training: 3 * 15 sec + 1 * 20 min pro Tag
					\item TEK-Sequence
						\begin{enumerate}
							\item Muskelentspannung
							\item Atementspannung
							\item Nicht-bewertende Wahrnehmung
							\item Akzeptieren und Tolerieren
							\item Effektive Selbstunterstützung
							\item Analysieren
							\item Regulieren
						\end{enumerate}
				\end{itemize}
			\item Entstehung von Stressreaktionen und konkreten Gefühlen
				\begin{itemize}
					\item Amygdala als Angst- und Stresszentrum (Stresshormone, Neurotransmitter, peripher. NS) 
					\item Stressreaktion ist unspezifische Aktivierung zur Mobilisierung und für effektiveren Schutz
					\item 2 Schritte der Emotionsentstehung
						\begin{enumerate}
							\item Schnelle Aktivierung der Amygdala $\rightarrow$ Stressreaktion (unspezifisch)
							\item Langsame Analys in höheren kortikalen Reagionen $\rightarrow$ Gefühl
						\end{enumerate}
					\item Schwächer $\rightarrow$ Angst $\rightarrow$ Fliehen und Vermeiden
					\item Stärker $\rightarrow$ Ärger $\rightarrow$ effektive Selbstdurchsetzung
				\end{itemize}
			\item Funktionen von Gefühlen
				\begin{itemize}
					\item Wichtige Informationen über bedrohte Ziele
					\item Hilfe bei der Durchführung von Handlungen
				\end{itemize}
			\item Kurzfristiger Stress unbedenklich, langfristig jedoch Gesundheitsschädlich
			\item Selbstregulation durch Cortisol aus Nebennierenrinde (ausgelöst durch Hypophyse)
			\item Ursachen anhaltenden Stresserlebens
				\begin{itemize}
					\item Wechselseitige Erregung versch. Areale
					\item Amygdala $\leftrightarrow$ Muskelanspannung / unruhiger, flacher Atem
				\end{itemize}
			\item Therapeutische Durchbrechung des Kreislaufs durch Erwerb von Basiskompetenzen
			\item Therapie: Regelmäßiges Training
			\item Hilfestellungen: Materalien, CDs, Übungsvorschläge per SMS / Mail, Übungskalender
			\item KVT + TEK effektiver als ohne
		\end{itemize}

	\section{Training sozialer Kompetenzen}
		\begin{itemize}
			\item Definitionen
				\begin{itemize}
					\item Fertigkeiten zur Erreichung von zwischenmenschlichen Zielen
					\item Durchsetzen eigener Interessen, Sympathie, Beziehungen, Konfliktumgang, Hilfe, Kritik
					\item Selbstsicherheit: Ansprüche stellen und verwirklichen (haben, äußern, durchsetzen)
				\end{itemize}
			\item Diagnostik
				\begin{itemize}
					\item Prüfung auf Kompetenzprobleme und Beziehung zu Behandlungszielen
					\item Interpers. Competence Questionnaire, Strukturiertes Interview zu operationalisierter Fertigkeitsdiagnostik
					\item U-Fragebogen
						\begin{enumerate}
							\item Angst vor Misserfolg / Kritik
							\item Kontaktangst
							\item Fordern können
							\item Nein sagen können
							\item Schuldgefühle
							\item Anständigkeit
						\end{enumerate}
					\item Rollenspiele
						\begin{itemize}
							\item Problem analysieren und Zielverhalten festlegen
							\item Von Model vormachen lassen
							\item In geschützter Umgebung üben
							\item Rückmeldung (konkret, positiv, Veränderungsmöglichkeiten)
							\item Schwierigkeitssteigerung
							\item Transfer
						\end{itemize}
				\end{itemize}
			\item Assertiveness Training Programm
				\begin{itemize}
					\item Stark vorstrukturiert
					\item 9 Schwierigkeiten bzgl. eigenem Sozialverhalten, Partnervariablen, Reaktion, Ort
					\item Hierarchie Bereiche
						\begin{itemize}
							\item Forderungen stellen
							\item Nein sagen, Kritik äußern
							\item Kontakte herstellen und aufrechterhalten
							\item Angst vor Fehlern überwinden
						\end{itemize}
				\end{itemize}
			\item Gruppentraining sozialer Kompetenzen
				\begin{itemize}
					\item Soz. Kompetenz als Kompromiss zw. soz. Anpassung und indiv. Bedürfnissen
					\item Standardisiert, flexibel, multimodal, kognitive Aspekte, allgemeine Problemlösefähig.
					\item Teilprozesse von Kompetenzproblemen
						\begin{itemize}
							\item Situationale Überforderung
							\item Ungünstige kogn. Verarbeitung / emotionale Prozesse / Verhaltenskonsequenzen
							\item Motorische Defizite
						\end{itemize}
					\item Modell: Situation löst aus:
						\begin{itemize}
							\item Neg. Selbstverbalisation $\rightarrow$ Angst, Unsicherheit $\rightarrow$ Vermeidung, 
								Flucht
							\item Pos. Selbstverbal. $\rightarrow$ Zuversicht $\rightarrow$ in Situation gehen
						\end{itemize}
					\item Verhaltensweise $\rightarrow$ Gewohnheit $\rightarrow$ Persönlichkeit
					\item Aufbau
						\begin{itemize}
							\item Einführungsveranstaltung
							\item PMR zur Angstkontrolle
							\item Beispiele für pos. / neg. Selbstverbal.
							\item Diskrimination soz. sicher / unsicher / aggressiv
							\item Instruktion selbstsicher. Verh. (RECHT)
							\item Pos. Selbstverbal. finden
							\item Gefühle äußern (BEZIEHUNG)
							\item Sympathien gewinnen (SYMPATHIE)
							\item Differenzierter Einsatz je nach Ziel

						\end{itemize}
					\item R-Typ-Situationen (Recht)
						\begin{itemize}
							\item Durchsetzen von Forderungen
							\item Durch Konventionen legitimiert
							\item Klar, präzise und verhaltensnah formulieren
						\end{itemize}
					\item B-Typ-Situationen (Beziehung)
						\begin{itemize}
							\item Einigung als Ziel
							\item Argumentation eher über eig. Gefühle / Bedürfnisse statt Normen
							\item Interesse anderen zu verstehen
							\item Emotionen ansprechen und in Wunsch münden lassen
						\end{itemize}
					\item S-Typ-situationen (Sympathie)
						\begin{itemize}
							\item Gegenüber hat Legitimation oder gute Beziehung im Fokus
							\item Flexibles Reagieren
							\item Gemeinsame Interessen herausstellen und Wunsch nach mehr Kontakt formulieren
						\end{itemize}

				\end{itemize}
			\item Soz. Kompetenztraining als integrativer Bestandteil
				\begin{itemize}
					\item Beginn (Diagnostik usw.) $\rightarrow$ Symptomorientierte Th. $\rightarrow$ soz. Komp.train.
					\item Training als Baustein
				\end{itemize}
			\item Soziale Kompetenz bei Kindern
				\begin{itemize}
					\item Lösungssuche, Beruhigen, Perspektivenübernahme, kooperativ, Rücksicht, Sorgsamkeit
					\item Gruppentraining für aggr. Jugendliche
				\end{itemize}
			\item Wissenschaftliche Bewertung
				\begin{itemize}
					\item Grawe: sehr wirksam bei soz. unsicheren Personen, überlegen zu GT und psychodyn.
					\item Kritisch bei sehr starken soz. Ausgangsdefiziten, Persönlichkeitsstörungen, (Agora)Phobie
				\end{itemize}
		\end{itemize}

	\section{Problemlösetrainings}
		\begin{itemize}
			\item Problem: Barriere verhindert Übergang von Ist- in Soll-Zustand
			\item Klinische Relevanz
				\begin{itemize}
					\item Unfähigkeit ein Problem alleine zu Lösen häufig Therapiegrund
					\item Generelle Heuristik für Umgang mit Problemen langfristig sinnvoller
				\end{itemize}
			\item Das allgemeine Problemlösemodell
				\begin{enumerate}
					\item Konstruktive Einstellung zu Problem aktivieren
					\item Problem sorgfältig beschreiben (Ist, Soll, Barriere) und analysieren (Gründe)
					\item Ziel setzen (relevant, realistisch, konkret, $\neq$ Sollzustand)
					\item Unkritisch viele Ideen zusammentragen
					\item Ideen bewerten (gewichtete Summe von Vor- / Nachteilen)
					\item Ideen zu Plan zusammenstellen
					\item Umsetzung
					\item Erfolgsprüfung ($\rightarrow$ Verstärkung / Ursachenanalyse)
				\end{enumerate}
			\item Diagnostik
				\begin{itemize}
					\item Exploration im Gespräch
					\item Fragebögen
					\item Testverfahren und / oder Interviews wie OFD (operationalisierte Fertigkeitsdiagnostik)
				\end{itemize}
			\item Selbstmanagement Therapie
				\begin{itemize}
					\item Am allg. Problemlösemodell ausgerichtet
					\item Anhand exemplarischer Probleme lernen
					\item Modell und dahinterstehende Struktur vermitteln
				\end{itemize}
			\item PL-Trainings
				\begin{itemize}
					\item Im Gruppensetting 
					\item Explizit und exklusiv auf Problemlösekompetenz fokussiert
					\item Einleitungsphase mit begründung der Relevanz und Vorstellung des Modells
					\item Gemeinsames entwickeln von Lösungen für konkr. Probleme der Teilnehmer
				\end{itemize}
			\item Empirische Absicherung
				\begin{itemize}
					\item Als eigenständige Therapiemaßnahme, zur Stressbewältigung oder im Rahmen von komplexeren KVT-Programmen
					\item Durch Vielzahl von Studien belegt
					\item Besonders bei Depression überlegen
					\item Effektiver wenn sämtliche Schritte thematisiert wurden
				\end{itemize}
		\end{itemize}

	\section{Paar-, Familientherapie und systemische Ansätze}
		\begin{itemize}
			\item 
		\end{itemize}

	\section{KVT II: Clark, Salovskis, Beck, Young}
		\begin{itemize}
			\item 
		\end{itemize}

	\section{KVT I: Ellis, Beck \& Meichenbaum}
		\begin{itemize}
			\item 
		\end{itemize}

	\section{Gesprächstherapie}
		\begin{itemize}
			\item 
		\end{itemize}

	\section{Interpersonelle Therapie \& CBASP}
		\begin{itemize}
			\item 
		\end{itemize}

	\section{Integrative Ansätze}
		\begin{itemize}
			\item 
		\end{itemize}

	\section{Psychotherapieforschung}
		\begin{itemize}
			\item 
		\end{itemize}



\end{document}
