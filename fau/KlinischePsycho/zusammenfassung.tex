\documentclass[11pt, paper=a4, twocolumn]{scrartcl}

\usepackage[ngerman]{babel}
\usepackage[utf8]{inputenc}

\usepackage[T1]{fontenc}
\usepackage{mathpazo}

\usepackage{geometry}

\usepackage{mathabx}

\geometry{a4paper, top=20mm, left=15mm, right=15mm, bottom=20mm,
headsep=5mm, footskip=12mm}


\pagenumbering{gobble}

\title{\vspace{-1.25cm}Zusammenfassung Klinische Psychologie\vspace{-0.25cm}}
\date{\vspace{-5ex}}

\newcommand*{\Z}{\mathbb{Z}}

\begin{document}
	\maketitle


	\section{Schizophrenie}
		\begin{itemize}
			\item 
		\end{itemize}

	\section{Somatoforme Störungen}
		\begin{itemize}
			\item 
		\end{itemize}

	\section{Rahmenbedingungen von Psychotherapie}
		\begin{itemize}
			\item Definitionen
				\begin{itemize}
					\item Wissenschaftlich anerkannte therapeutische Verfahren
					\item Feststellung, Heilung oder Linderung von Störungen mit Krankheitswert
					\item Interaktionaler Prozess mit psychologischen Mitteln
				\end{itemize}
			\item Rechtliche Rahmenbedingungen
				\begin{itemize}
					\item Psychotherapeutengesetz: Approbation nötig, regelt Voraussetzungen
					\item Gesetzliche Leistungserstattung nur mit Kassenzulassung
				\end{itemize}
			\item Ausbildung zum Psychologischem Psychotherapeuten (PPT)
				\begin{itemize}
					\item 3 / 5 Jahre
					\item Praktische Tätigkeit: Psychiatriejahr (Diagnostik und Behandlung) + Behandlungsstunden
					\item Theoretische Fortbildung: Vertiefung und praktische Übung
					\item Selbsterfahrung: eigenes Verhalten, Denken, Fühlen überprüfen und pos. verändern
					\item Supervision: Identifizieren und Üben von Optimierungsmöglichkeiten
				\end{itemize}
			\item Antragspflicht: notwendig, zweckmäßig und wirtschaftlich
			\item Gutachterverfahren für LZT
				\begin{itemize}
					\item Liegt Krankheit vor?
					\item Psychotherapie indiziert?
					\item Geplante Therapie zweckmäßig, wirtschaftlich und hinreichend gute Prognose?
					\item Rahmenbedingungen eingehalten?
				\end{itemize}
			\item Gliederung: Beschwerden, Vorgeschichte, psychische / somatische Symptomatik, Störungsmodell, ICD-10, Ziele, Therapieplan
			\item Gutachter gibt Empfehlung, Kasse entscheidet
			\item Stundenkontingente\\
				\begin{tabular}{c|cccc}
					& kurz & lang & max & $\diameter$ (Schulz) \\
					\hline Verhaltensth & 24 & 60 & 80 & 38.6 \\
					Tiefenpsych. & 24 & 60 & 100 & 53.7 \\
					Psychoanalyse & - & 160 & 300 & 106.5
				\end{tabular}
			\item Therapeutische Versorgung
				\begin{itemize}
					\item Ambulant (Praxis, Ambulanzen, Tageskliniken)
					\item Stationär (Psychosomatik, Rehabilitation, Psychiatrie)
					\item 32\% psychische Störungen, davon 63\% nicht versorgt, 18\% der Versorgten bei PPT
					\item Bei Depressionen ca. 50\%
					\item Starker Ost-West Unterschied
					\item Versorgangsgrad ca. 46\%
					\item AU-Tage steigen konstant (11-69\%), Frauen höher als Männer
				\end{itemize}
			\item Ansatz für Approbation nach Studiumsabschluss in Diskussion
			\item Rechtliche Rahmenbedingungen
				\begin{itemize}
					\item Ausbildungs- und Prüfungsverordnung
					\item Berufsordnung des jeweiligen Bundeslands
					\item Landespsychotherapeutenkammer (verfasst Berufsordnung)
					\item Bundespsychotherapeutenkammer
					\item Kassenärztliche Vereinigung
					\item Gemeinsamer Bundesausschuss (erarbeitet PT-Richtlinien)
					\item Schweigepflicht und Datenschutz
				\end{itemize}
		\end{itemize}

	\section{Therapiemotivation}
		\begin{itemize}
			\item Therapiemotivation vs. Veränderungsmotivation
			\item Misserfolg $\leftrightarrow$ Motivationsbeeinträchtigung
			\item Duales Therapiemodell nach Schulte und Eifert
				\begin{itemize}
					\item Methodenstrang
						\begin{itemize}
							\item Identifizieren von Bedingungen
							\item Auswahl von Strategien und Techniken
							\item Umsetzung
						\end{itemize}
					\item Motivationsstrang
						\begin{itemize}
							\item Prüfen von Therapie- / Veränderungsmotivation
							\item Identifizieren von motivationalen Problemen
							\item Einsatz von Methoden zur Motivationsstärkung
						\end{itemize}
				\end{itemize}
			\item Motivationsstärke = ((Nutzen - Kosten) * Durchführbarkeit) + Entschluss der Ausführung
			\item Präskriptives Entschlussmodell
				\begin{itemize}
					\item Option gut genug? Ja $\rightarrow$ Dominanzbildung, Planung, Handlung
					\item Nein $\rightarrow$ suche nach neuer Opt. und Antizipation aversiver Konsequenzen
					\item Nein + emotionale Selbsteffizienz und Einsicht in Ausschließlichkeit, Opfernotwendigkeit 
						$\rightarrow$ Option beste verfügbare?
					\item Ja $\rightarrow$ Dissonanzreduktion (Rechtfertigung durch Langzeitpräferenzen, kognitive Repräsentation) 
						$\rightarrow$ Handlung
				\end{itemize}
			\item Vor- und Nachteilsanalysen
				\begin{itemize}
					\item Tabellenartige Auflistung für spezifische Situation
					\item Kurzfristig: Vorteile (jemand der zuhört) vs. Nachteile (Aussetzen von Ängsten)
					\item Langfristig: Vorteile (Überwindung von Ängsten) vs. Nachteile (weniger Schonung / Ausreden)
				\end{itemize}
			\item Stadien der Veränderungsbereitschaft
				\begin{itemize}
					\item Precontemplation (Problem nicht erkannt)
					\item Contemplation (Veränderungsabsicht)
					\item Determination / Preparation (Schritte planen)
					\item Action (Veränderung vollziehen)
					\item Maintenance (Aufrechterhalten und Rückfall vermeiden)
					\item Relapse (Rückfall bewältigen)
				\end{itemize}
			\item Motivierende Gesprächsführung
				\begin{itemize}
					\item Annahmen: Autonmieverletzung führt zu Reaktanz, ambivalent motiviert, pos. Motive stärken, neg. 
						wertschätzen
					\item 2 Phasen: Aufbau von Veränderungsbereitschaft $\rightarrow$ Stärkung der Selbstverpflichtung
					\item Prinzipien
						\begin{itemize}
							\item Empathie / Wertschätzung: dysfunktionales Verhalten für legitime Ziele
							\item Herausarbeiten von Diskrepanzen: Konsequenzen vs. Ziele
							\item Geschmeidiger Umgang mit Widerstand
							\item Stärkung der Selbsteffizienz: Zuversicht auf Beeinflussung
						\end{itemize}
					\item Interventionen
						\begin{itemize}
							\item Offene Fragen
							\item Aktives und empathisches Zuhören (Paraphrasieren, Zusammenfassen, usw.)
							\item Würdigung (Anerkennen und wertschätzen ALLER Leistungen)
							\item Informationsvermittlung
							\item Non-Direktivität / geleitetes Entdecken
							\item Förderung von change talk (Verstärken von Aussagen mit Veränderungsabsicht)
							\item Konstruktiver Umgang mit Widerstand (überzogenes Widerspiegeln)
							\item Förderung von Handlungsorientierung (konkrete Ziele, commitment fördern)
						\end{itemize}
				\end{itemize}
			\item EPOS (Elaboration positiver Perspektiven)
				\begin{itemize}
					\item Zwei Sitzungen
					\item Vorstellen eines Tags in 5 Jahren, wenn alles nach seinem Willen gelaufen ist
					\item Offene Fragen, mehrere Sinneskanäle
					\item Audioaufnahme für Patient
					\item Zweite Sitzung: imaginativ-holistisch $\rightarrow$ analytisch-zielgerichtet
					\item Zentrale Bilder und dahinterliegende Wünsche herausarbeiten
					\item Wünsche langfristige Ziele?
					\item $\rightarrow$ mittelfristige, kurzfristige Ziele
				\end{itemize}
		\end{itemize}

	\section{KVT-Intro \& Entspannung}
		\begin{itemize}
			\item KVT
				\begin{itemize}
					\item Empirisch belegte Methoden und Theorien
					\item Gegenwarts-, Handlungs-, Problem- und Zielorientiertheit
				\end{itemize}
			\item PMR
				\begin{itemize}
					\item Psychophysiologischer Entspannungsprozess durch Kontrasterleben
					\item Kurzform vs. Langform
					\item Einsatzmöglichkeiten
						\begin{itemize}
							\item Regelmäßige Entspannung (anhaltende Reduktion der Anspannung)
							\item Angewandte Entspannung (Stressbewältigung)
							\item Systematische Desensibilisierung (stufenweise Konfrontation in sensu, reziproke Inhibition)
						\end{itemize}
					\item Ablauf: Einführung, Verkürzung, ohne Anspannung, Selbstinstruktion, verschiedene Kontexte, schnell, 
						Anwendung
					\item Effektivität
						\begin{itemize}
							\item Durch Studien belegt: Anspannungsreduktion, chronische Schmerzzustände, Hypertonie, 
								Angststörung, Schlafstörung, psychosomatische Beschwerden
							\item System. Desens. nachgewiesen aber nur wenn massive Konfrontation nicht möglich
							\item Effektstärken $d=0.58$
						\end{itemize}
				\end{itemize}
			\item Biofeedback
				\begin{itemize}
					\item Rückmeldung physiologischer Parameter durch Geräte
					\item Patient soll Signale beeinflussen $\rightarrow$ auch psych. Prozesse
					\item Evtl. Vermittlung bestimmter Technik
					\item Lernen Beschwerden direkt (Blutdruck) / indirekt (generelle Entspannung) zu beeinflussen
					\item Beeinflussbar: Muskelspannung, Hautleitfähigkeit, Arteriendurchmesser der Schläfenartherie, usw.
					\item Psychophysiologische Überaktivierung als Störungskorrelat
						\begin{itemize}
							\item Erhöhter Grundtonus und Reaktivität
							\item Verringerte Habituation bei Wiederholung
							\item Verzögerte Erholungsphasen
							\item Erhöhte Dishabituation / Reaktivität
						\end{itemize}
					\item Ablauf
						\begin{itemize}
							\item Identifikation von Zielparametern
							\item Eingangsdiagnostik, Mutlikanal-Messung und Einflussübung
							\item Trainingssitzungen (Baseline, Training, Generalisierung)
							\item Kombination mit anderen Interventionen
							\item Abschluss
						\end{itemize}
					\item Metaanalyse bei Migräne belegt stabilen Effekt von ca. $d=0.56$
					\item Spannungskopfschmerzen auch per Metastudie belegt $d=.82$
				\end{itemize}
			\item Imaginative Entspannung
				\begin{itemize}
					\item Verschiedene Verfahren mit unterschiedlichen Zielen
					\item Z.B. Fantasiereise zur Anregung von individuellen Assoziationen und Interpretationen
					\item Problematisch bei gering ausgeprägtem Vorstellungsvermögen
					\item Keine ausreichenden empirischen Belege
				\end{itemize}
			\item Hypnose
				\begin{itemize}
					\item Trancezustand: minimaler Widerstand / maximiale Akzeptanz
					\item Entspannungselemente aber auch weitere Zielsetzungen (Neubewertung, Veränderung von Reaktionsmustern)
					\item Einengung der Wahrnehmung durch Augenfixation oder Suggestion
					\item Hypnotisierbarkeit als Persönlichkeitsmerkmal
					\item Empirische Fundierung
						\begin{itemize}
							\item Gut bei chronischen Schmerzen
							\item Positiv bei Rauchstopp, Schlafstörungen, Angststörungen
							\item Wenig bei Abhängigkeitserkrankungen und Bluthochdruck
						\end{itemize}
				\end{itemize}
			\item Autogenes Training
				\begin{itemize}
					\item Anwender leitet Entspannungszustand selbst ein
					\item Durchführung
						\begin{itemize}
							\item 6 Übungen (Schwere-, Wärme-, Herz-, Atmungs-, Sonnengeflechts-, Stirnkühleübung)
							\item Formeln für relevante Prozesse zur Beeinflussung (Arme schwer usw.)
						\end{itemize}
					\item Empirische Fundierung
						\begin{itemize}
							\item Weniger eindeutig als PMR
							\item Mittlere Effekte für u.a. Kopfschmerzen, Hypertonie, Schlafstörungen
						\end{itemize}
				\end{itemize}
			\item Indikation
				\begin{itemize}
					\item Angst-, Schlaf- und somatoforme Störungen
					\item Geringe Wahrscheinlichkeit von Nebenwirkungen
				\end{itemize}
			\item Kontraindikationen: Hypotonie (niedriger Blutdruck), Herz- oder Atembeschwerden, akute Psychosen
			\item Probleme
				\begin{itemize}
					\item Sexuelle Missbrauchsopfer (Panikattacken durch Setting)
					\item Panikattacken generell (Vermeidung durch Entspannung)
					\item Angst vor Kontrollverlust
				\end{itemize}
		\end{itemize}

	\section{Konfrontationsverfahren}
		\begin{itemize}
			\item Einteilung
				\begin{itemize}
					\item Graduiert: System. Desensibilisierung (sensu), Habituationstraining (vivo)
					\item Massiert: Implosion (sensu), Flooding (vivo)
				\end{itemize}
			\item Systematische Desensibilisierung
				\begin{itemize}
					\item Prinzip der reziproken Hemmung
					\item Anwendung
						\begin{itemize}
							\item Hierarchisierung der Angstsituationen
							\item Lernen von Entspannung
							\item Vorstellung und Entspannung
							\item Wenn angstfrei Aufstieg
						\end{itemize}
					\item Wirkmechanismus hauptsächlich nur Außeinandersetzung mit Angst, nicht Hierarchie
				\end{itemize}
			\item Durchführung
				\begin{itemize}
					\item Diagnostik, Verhaltens- und Bindungsanalyse
					\item Kognitive Vorbereitung: Störungsmodell $\rightarrow$ therapeutisches Vorgehen
					\item Bewusste und freiwillige Entscheidung mit mental präsentiertem Grund
					\item Intensiv-Expo begleitet
					\item Expo allein
					\item Generalisierungstraining
				\end{itemize}
			\item Begründungsmodelle
				\begin{itemize}
					\item VT-Vermeidungs-Rational: Aufrechterhalten von Angst durch Vermeidung, da keine Habituation
					\item Kogn. Rational I: Falsche Bewertung des Stimulus kann durch Vermeidung nicht geprüft werden
					\item Kogn. Rational II: Angst durch Bewertung $\rightarrow$ bewertungsfrei wahrnehmen
				\end{itemize}
			\item Erwartete Angstverläufe als Vorbereitung
			\item Regeln für Exposition
				\begin{itemize}
					\item Zulassen der Ängste
					\item Verzicht auf Vermeidungsstrategien
					\item Angst beobachten ohne zu bewerten
					\item Achten auf kleine Veränderungen
					\item Unterscheidung Realität $\leftrightarrow$ Phantasie
					\item In Situation bleiben bis Angst abfällt
				\end{itemize}
			\item Therapeutenverhalten
				\begin{itemize}
					\item Angst beobachten und auf Skala angeben lassen (auch körperliche Symptome)
					\item Nach Vermeidungsstrategien fragen
					\item Fragen wie man Angst steigern kann und umsetzen
					\item Hinweise auf Angstabfall beschreiben lassen und versuchen trotzdem zu steigern
				\end{itemize}
			\item Weitere Regeln
				\begin{itemize}
					\item Genügend Zeit, kontrollierbare Situation
					\item Erst bei deutlichem Angstabfall beenden
					\item Konstruktive Nachbewertung
					\item Problem: unangenehme Übung als Beleg für Angst
					\item Expositionsrational: Weg aus Angst durch Angst
					\item Kognitives Rational: Bewertung kann sich nur ändern wenn Chance zu sehen dass Erwartungen nicht eintreffen
					\item Aufstellen und Bewerten von Hypothesen
				\end{itemize}
			\item Massiert vs. graduiert: ähnliche Erfolge aber massiert stabiler
			\item Individualisierte vs. standardisierte Pläne: standard besser, v.a. bei unerfahrenen Therapeuten
			\item Studien zur Panikbehandlung
				\begin{itemize}
					\item Barlow et al.: CBT > CBT + PME > PME
					\item Clark et al.: CBT mit Expo > Antidepressiva > graduierte Expo
					\item Metaanalyse: Konfrontation in vivo > kogn. behav. > kogn. > GT
					\item Rief et al.: Intensiv Expo. effektiver als kürzere
				\end{itemize}
			\item Exposition bei Zwangsstörungen
				\begin{itemize}
					\item Problem: Zwangshandlungen nach Expo als Ausgleich
					\item Lösung: Mehr und längere Expos mit Reaktionsverhinderung
					\item Oft nur bis 50\% Angst
				\end{itemize}
			\item Expo bei PTSD
				\begin{itemize}
					\item Problem: Trauma nicht direkt konfrontierbar
					\item Lösung: In sensu und Trauma-assoziierte Stimuli in vivo
					\item Prolonged Exposure: Imagination, Bericht, Tonbandaufzeichnung, in vivo Expo
					\item Stress-Impfungstraining: Entspannung, Rollenspiele
					\item Beste Ergebnisse bei PE
				\end{itemize}
			\item Expo bei Essstörungen
				\begin{itemize}
					\item Expo mit Körper (Spiegel), Nahrungsmittel $\rightarrow$ Prüfung irrationaler Kognitionen
				\end{itemize}
			\item Indikationen für graduiertes Vorgehen
				\begin{itemize}
					\item Körperliche Komplikationen
					\item Fehlende Motivation
				\end{itemize}
		\end{itemize}

	\section{Operante Verfahren und Modelllernen}
		\begin{itemize}
			\item Operante Verfahren
				\begin{itemize}
					\item Arten operanter Konditionierung
						\begin{itemize}
							\item Positiv: Belohnung (Erhöhung), Bestrafung (Verringerung)
							\item Negativ: Flucht (beenden) / Vermeidung (nicht erfolgen), Time Out / Löschung
						\end{itemize}
					\item Verstärkerarten
						\begin{itemize}
							\item Positive vs. negative Verstärker
							\item Primäre Verstärker: Erfüllen Grundbedürfnisse (abhängig von Sättigung)
							\item Sekundäre (konditionierte) Verstärker: assoziiert mit prim. Verstärker (z.B. Lob, 
								Aufmerksamkeit)
							\item Generalisierte Verstärker: Tauschwert (Token)
							\item Soziale Verstärkung: Gratifikation / Sanktion
							\item Selbstverstärkung: z.B. explizites Selbstlob
						\end{itemize}
					\item Therapeutisches Vorgehen
						\begin{itemize}
							\item Shaping: Aufbau des Zielverhaltens
							\item Chaining: ähnlich zu Shaping aber Verstärkung des Abschlusses, nicht Beginn
							\item Fading: Verstärkung des Zielverhaltens unter schrittweiser Ausblendung der Hilfsstimuli
							\item Prompting: Unterstützung des Beginnns der Verhaltensänderung durch (non)verbale 
								Hilfestellung
							\item Intermittierende Verstärkung: (variables) Festigen bereits gezeigten Verhaltens
							\item Direkte Bestrafung: selten langfristig, negative Auswirkung auf Beziehung
							\item Indirekte Bestrafung: Verstärkerentzug bei dysfunkt. Verhalten
							\item Löschung: Verhindern pos. Konsequenzen von neg. Verhalten
							\item Time-Out
							\item Verhaltens- oder Kontingenzverträge: Verhalten $\rightarrow$ Konsequenz
						\end{itemize}
					\item Stimuluskontrolle durch Hinzufügen / Entfernen von Hinweisreizen
					\item Indikation
						\begin{itemize}
							\item Verstärkungspläne für Depression
							\item Bei Abhängigkeiten, Kindern, Anorexie, Schlafstörungen, Demenz
							\item Bei eingeschränkten kognitiven Fähigkeiten
						\end{itemize}
				\end{itemize}
			\item Modelllernen (vgl. Bandura)
				\begin{itemize}
					\item Therapeutisches Vorgehen
						\begin{itemize}
							\item Modell führt Verhalten aus und wird verstärkt (für Patient sichtbar)
							\item Patient führt Verhalten selbst aus und wird ebenfalls verstärkt
							\item Ähnlichkeit von Modell und Patient wichtig
						\end{itemize}
					\item Effektivität
						\begin{itemize}
							\item Mastery-Modell: Modell problemlos $\rightarrow$ Scham bei Patient
							\item Coping-Modell: Modell überwindet anfängliche Probleme
							\item Transfer in Alltag durch Entkopplung und steigende Einsicht des Patienten
							\item Als Mittel durch empirische Untersuchungen belegt aber keine Belege für eigenständige 
								Methode
						\end{itemize}
				\end{itemize}
			\item Rollenspiele
				\begin{itemize}
					\item Operante Techniken und Modelllernen
					\item Diagnostisches vs. therapeutisches Rollenspiel
					\item Problembeschreibung, Festlegen der Situation, Durchführung, Auswertung, Erneutes Spielen und Auswerten, 
						Transfer
				\end{itemize}
		\end{itemize}

	\section{Psychodynamische Ansätze}
		\begin{itemize}
			\item Geschichte und Grundlagen
				\begin{itemize}
					\item Oberbegriff für Therapieformen die aus Psychoanalyse (Freud) hervorgegangen
					\item Konzept des Unbewussten und Anwendung auf Entwicklung, Pathologie, Therapie und Sozial- / Kulturtheorie
					\item Triebtheorie (anfänglich rekonstruktive Entwicklungstheorie)
				\end{itemize}
			\item Ein-Personen-Psychologie 
				\begin{itemize}
					\item Topographisches Modell
						\begin{itemize}
							\item Unbewusstes (Lustprinzip, Primärvorgänge)
							\item Vorbewusstes
							\item Bewusstes (Realitätsprinzip, Sekundärvorgänge)
						\end{itemize}
					\item 3 Instanzen-Modell
						\begin{itemize}
							\item Erleben, Handeln und Denken von unbewussten dynamischen Kräften beeinflusst
							\item Es: triebhaft, instinktiv, Energiespeicher
							\item Über-Ich: elterliche / gesellschaftliche Werte, Ge- / Verbote, Moral
							\item Ich: Ausgleich und Vermittlung
							\item Triebe: angeboren, Drang, Objektfixierung, somatische Quelle, Triebökonomie
						\end{itemize}
					\item Konflikttypen
						\begin{itemize}
							\item Zwischen versch. Triebobjekten
							\item Verschiedene Wünsche an einem Objekt
							\item Triebabfuhr vs. Unterdrückung
						\end{itemize}
					\item Entwicklungsspezifische Konflikte (Phasen)
						\begin{itemize}
							\item Orale Phase: Mutter, Süchte, Depressionen, Schizophrenie
							\item Anale Phase: Abgabe / Zurückhaltung, Mutter, Zwänge, Perversionen, Persönlichkeitsstörungen
							\item Phallische Phase: Libidinöse Triebbefriedigung, anderes Geschlecht, Neurosen, Ängste, 
								Phobien, Hysterie
						\end{itemize}
					\item Ich-Psychologie
						\begin{itemize}
							\item Ich-Funktionen autonom von Es
							\item Ich - Umwelt: Intention, Motorik, Denken usw.
							\item Ich-Funktionen zur Konfliktbewältigung: Reife vs. Unreife als Abwehrmechan.
						\end{itemize}
				\end{itemize}
			\item Zwei-Personen-Psychologie
				\begin{itemize}
					\item Frühe zwischenmenschliche Beziehungserfahrungen (von Trieben motiviert)
					\item Selbstpsychologie
						\begin{itemize}
							\item Entwicklung der Struktur des Selbst (reflektierter Teil des Ich)
							\item Ziel: narzisstische Homöostase (stabiler Selbstwert durch mütterliche Empathie)
						\end{itemize}
					\item Objektbeziehungstheorie
						\begin{itemize}
							\item Beziehungen als Zentrum seelischen Erlebens
							\item Psychische Strukturen als Ergebnis von internalisierten Objektbeziehungen
							\item Jede reale Begegnung wird verinnerlicht
						\end{itemize}
				\end{itemize}
			\item Intersubjektive Wende
				\begin{itemize}
					\item Selbst als Resultat von wechselseitiger intersubjektiver Bezogenheit
					\item Entsteht im Hier und Jetzt
				\end{itemize}
			\item Gemeinsame Annahmen psychodyn. Verfahren
				\begin{itemize}
					\item Psychischer Determinismus (Beeinflussung von Wahrnehmung durch unbewusste Bedürfnisse)
					\item Einzigartigkeit (gleiches Symptom kann untersch. Gründe haben)
					\item Gesundheit und Pathologie als Kontinuum
					\item Abwehrmechanismen (zur Fernhaltung unangenehmer Gefühle von Bewusstsein)
					\item Innerpsychischer Konflikt (Symptombildung bei mangelnden Abwehrmechan.)
					\item Symptombildung als Lösungsversuch
				\end{itemize}
			\item Neurose und Neurosenstruktur
				\begin{itemize}
					\item William Cullen: funktionelle Erkrankung ohne organische Läsion
					\item Laplanche \& Pontalis: psychogene Affektion, Symptome Ausruck von Konflikt, Wurzeln in Kindheitsgeschichte
					\item Hoffmann \& Hochapfel: überwiegend umweltbedingt, psych., körperl. oder persönl. Störung, Symptome 
						Verarbeitungsversuche infantiler Konflikte
				\end{itemize}
			\item Genetisch-konstitutionelle Aspekte
				\begin{itemize}
					\item Neurotische Störungen überwiegend psychoreaktiv
					\item Symptomwahl genetisch mitbedingt
					\item Einflusstendenz: Zwang > Phobie > Angst > neurot. Depression > Konversionsstörung
				\end{itemize}
			\item Psychoanalytische Krankheitsmodelle
				\begin{itemize}
					\item Konfliktmodell
						\begin{itemize}
							\item Konflikt: Mind. 2 widerstrebende, unvereinbare Tendenzen
							\item Aktueller innerer Konflikt aktiviert ungelösten infantilen Konflikt
							\item Außerdem Angst + Bewältigugnsversuche dieser
							\item Versuche durch infant. Konfl. eingeschränkt
							\item Dadurch massive Angst und neurotische Symptombildung als Abwehrmechan.
							\item Symptom als Kompromiss zw. Wunsch \& Verbot
							\item $\Rightarrow$ Angstminderung und Entlastung
						\end{itemize}
					\item Strukturmodell
						\begin{itemize}
							\item Bezug auf Entwicklungsstufen
							\item Annahme: Strukturdefizit (gestörtes / unterentwickeltes Ich)
							\item Ursache: misslunge Internalisierungserfahrungen als Kind
							\item Interaktionales Austragen von Konflikten
							\item Symptomatik bestimmt durch Eingeschränkte Regulationsfähig. und unterpers. Beziehungen
							\item Abbruch einer Beziehung $\rightarrow$ Regulative Funktion entfällt $\rightarrow$ Angst 
								$\rightarrow$ Symptome $\rightarrow$ Regrisseve Lösungsversuche (Dissoziation) oder 
								Verschiebung / Kompensation (Süchte, Essstörungen)
						\end{itemize}
					\item Traumamodell
						\begin{itemize}
							\item Traumatische Erfahrungen $\rightarrow$ Ausgeliefertsein $\rightarrow$ Sicherheitsverlust 
								$\rightarrow$ Entwicklungsstörungen
							\item Traumatische Erfahrung bleibt abgespalten und unintegriert
							\item Symptome: PTSD
							\item Traumatisierung überfordert Verarbeitungsfähigkeit des Ich $\Rightarrow$ Dissoziation
						\end{itemize}
				\end{itemize}
			\item Diagnostik
				\begin{itemize}
					\item Bevorzugung weniger strukturierter Vorgehensweisen
					\item Patient kann Gesprächsverlauf unbewusst gestalten und Therapeut kann subj. Realität erfassen
					\item Operationalisierte Psychodynamische Diagnostik
						\begin{itemize}
							\item Psychodynamische Ergänzung zu deskriptiven Manualen (IDC-10)
							\item OPD-2 auch zur Therapieplanung, Ressourcenfassung und Veränderungsmessung
						\end{itemize}
					\item 5 Achsen der OPD
						\begin{itemize}
							\item Krankheitserleben und Behandlungsvoraussetzung
							\item Beziehung
							\item Konflikt
								\begin{itemize}
									\item Individuation vs. Abhängigkeit
									\item Unterwerfung vs. Kontrolle
									\item Autarkie vs. Versorgung
									\item Selbstwertkonflikt
									\item Schuldkonflikt
									\item Ödipal-sexuelle Konflikte
									\item Identitätskonflikt
								\end{itemize}
							\item Struktur
							\item Psych. und psychosom. Störungen nach ICD-10
						\end{itemize}
				\end{itemize}
			\item Psychodynamische Therapieverfahren
				\begin{itemize}
					\item Gemeinsame Bestrebungen
						\begin{itemize}
							\item Zusammenhang zw. Erfahrungen und Erleben herstellen
							\item Förderung der Einsicht und Selbstakzeptanz
							\item Aufdeckung der Hintergründe, nicht Symptom selbst
						\end{itemize}
					\item Psychoanalyse
						\begin{itemize}
							\item Frequenz, Dauer und Sitzungsanzahl unbegrenzt
							\item Ziel: Veränderung der Persönlichkeitsstruktur
							\item Basis: intensive emotionale Beziehung zu Analytiker
							\item Aktivierung, Wiedererleben und Korrektur von Konflikten (Übertragung)
							\item Fokus: subjektives Realitätserleben
							\item Liegeposition $\Rightarrow$ Abschirmen von Einflüssen und Verringerung der Selbst- \& 
								Fremdkontrolle
							\item Techniken: Freie Assoziation, Klären, Konfrontieren, Deuten, Übertragung, Gegenübertragung
						\end{itemize}
					\item Analytische Psychotherapie
						\begin{itemize}
							\item Psychoanalyse mit Rahmenbedingungen des Versorgungssystems
							\item Mehr Fokus auf Symptome und Störungen
							\item Strukturelle Änderung aber weniger tiefgreifend
							\item 1-3 Jahre, 80-240 (300) Sitzungen
						\end{itemize}
					\item Tiefenpsychologisch fundierte Psychotherapie
						\begin{itemize}
							\item Fokus: Aktuelle Lebensbelastungen und soziale Beziehungen
							\item Häufigste dyn. Therapieform
							\item 25 bis 100 Sitzungen im Sitzen
							\item Therapeut aktiver (emotional unterstützend, fokussiert)
						\end{itemize}
					\item Psychodynamische Kurzzeittherapie
						\begin{itemize}
							\item Klar definiertes Problem un kurzem festen Zeitraum behandeln
							\item Interpretationen v.a. auf Gegenwart bezogen
							\item 8-25 mal im Sitzen
						\end{itemize}
				\end{itemize}
			\item Empirische Absicherung
				\begin{itemize}
					\item Vorgänge zu komplex um gemessen zu werden
					\item Jedoch zunehmende malualisierung und Annäherung an Wirksamkeitsprüfung
					\item Einzelstudien, jedoch keine emoirische Forschung zu psychodyn. Langzeittherapie
					\item Zwei Studien, jedoch stark kritisierte methodische Qualität und Schlussfolgerungen
					\item Metaanalyse zu Schizophrenie spricht gegen psychodyn. PT im stationären Bereich
					\item Kurzzeittherapie am besten untersucht: große Effekte u.a. bei Depression
				\end{itemize}
		\end{itemize}

	\section{Training emotionaler Kompetenzen}
		\begin{itemize}
			\item Modell des konstruktiven Umgangs mit Gefühlen
				\begin{itemize}
					\item Bewusstes Wahrnehmen $\rightarrow$ Erkennen \& Benennen $\rightarrow$ Analyse der Ursachen
					\item Emotionale Selbstunterstützung unterstütz Analyse und andere Folgeprozesse
					\item a) Finden von Veränderungspunkten $\rightarrow$ Zielgerichtete Modifikation
					\item b) Konstruktive Hoffnungslosigkeit$\rightarrow$ Akzeptanz \& Toleranz
					\item Kompetenzerwerb (a) / Resilienzbildung (b) $\Rightarrow$ Konfrontationsbereitschaft
				\end{itemize}
			\item Gezielte Diagnostik gesundheitsschädlicher Emotionen und der Emotionsregulationskompetenzen
			\item Training emotionaler Kompetenzen (TEK)
				\begin{itemize}
					\item Theorie: Information $\rightarrow$ Orientierung $\rightarrow$ Motivation
					\item Intensives Training: 3 * 15 sec + 1 * 20 min pro Tag
					\item TEK-Sequence
						\begin{enumerate}
							\item Muskelentspannung
							\item Atementspannung
							\item Nicht-bewertende Wahrnehmung
							\item Akzeptieren und Tolerieren
							\item Effektive Selbstunterstützung
							\item Analysieren
							\item Regulieren
						\end{enumerate}
				\end{itemize}
			\item Entstehung von Stressreaktionen und konkreten Gefühlen
				\begin{itemize}
					\item Amygdala als Angst- und Stresszentrum (Stresshormone, Neurotransmitter, peripher. NS) 
					\item Stressreaktion ist unspezifische Aktivierung zur Mobilisierung und für effektiveren Schutz
					\item 2 Schritte der Emotionsentstehung
						\begin{enumerate}
							\item Schnelle Aktivierung der Amygdala $\rightarrow$ Stressreaktion (unspezifisch)
							\item Langsame Analys in höheren kortikalen Reagionen $\rightarrow$ Gefühl
						\end{enumerate}
					\item Schwächer $\rightarrow$ Angst $\rightarrow$ Fliehen und Vermeiden
					\item Stärker $\rightarrow$ Ärger $\rightarrow$ effektive Selbstdurchsetzung
				\end{itemize}
			\item Funktionen von Gefühlen
				\begin{itemize}
					\item Wichtige Informationen über bedrohte Ziele
					\item Hilfe bei der Durchführung von Handlungen
				\end{itemize}
			\item Kurzfristiger Stress unbedenklich, langfristig jedoch Gesundheitsschädlich
			\item Selbstregulation durch Cortisol aus Nebennierenrinde (ausgelöst durch Hypophyse)
			\item Ursachen anhaltenden Stresserlebens
				\begin{itemize}
					\item Wechselseitige Erregung versch. Areale
					\item Amygdala $\leftrightarrow$ Muskelanspannung / unruhiger, flacher Atem
				\end{itemize}
			\item Therapeutische Durchbrechung des Kreislaufs durch Erwerb von Basiskompetenzen
			\item Therapie: Regelmäßiges Training
			\item Hilfestellungen: Materalien, CDs, Übungsvorschläge per SMS / Mail, Übungskalender
			\item KVT + TEK effektiver als ohne
		\end{itemize}

	\section{Training sozialer Kompetenzen}
		\begin{itemize}
			\item Definitionen
				\begin{itemize}
					\item Fertigkeiten zur Erreichung von zwischenmenschlichen Zielen
					\item Durchsetzen eigener Interessen, Sympathie, Beziehungen, Konfliktumgang, Hilfe, Kritik
					\item Selbstsicherheit: Ansprüche stellen und verwirklichen (haben, äußern, durchsetzen)
				\end{itemize}
			\item Diagnostik
				\begin{itemize}
					\item Prüfung auf Kompetenzprobleme und Beziehung zu Behandlungszielen
					\item Interpers. Competence Questionnaire, Strukturiertes Interview zu operationalisierter Fertigkeitsdiagnostik
					\item U-Fragebogen
						\begin{enumerate}
							\item Angst vor Misserfolg / Kritik
							\item Kontaktangst
							\item Fordern können
							\item Nein sagen können
							\item Schuldgefühle
							\item Anständigkeit
						\end{enumerate}
					\item Rollenspiele
						\begin{itemize}
							\item Problem analysieren und Zielverhalten festlegen
							\item Von Model vormachen lassen
							\item In geschützter Umgebung üben
							\item Rückmeldung (konkret, positiv, Veränderungsmöglichkeiten)
							\item Schwierigkeitssteigerung
							\item Transfer
						\end{itemize}
				\end{itemize}
			\item Assertiveness Training Programm
				\begin{itemize}
					\item Stark vorstrukturiert
					\item 9 Schwierigkeiten bzgl. eigenem Sozialverhalten, Partnervariablen, Reaktion, Ort
					\item Hierarchie Bereiche
						\begin{itemize}
							\item Forderungen stellen
							\item Nein sagen, Kritik äußern
							\item Kontakte herstellen und aufrechterhalten
							\item Angst vor Fehlern überwinden
						\end{itemize}
				\end{itemize}
			\item Gruppentraining sozialer Kompetenzen
				\begin{itemize}
					\item Soz. Kompetenz als Kompromiss zw. soz. Anpassung und indiv. Bedürfnissen
					\item Standardisiert, flexibel, multimodal, kognitive Aspekte, allgemeine Problemlösefähig.
					\item Teilprozesse von Kompetenzproblemen
						\begin{itemize}
							\item Situationale Überforderung
							\item Ungünstige kogn. Verarbeitung / emotionale Prozesse / Verhaltenskonsequenzen
							\item Motorische Defizite
						\end{itemize}
					\item Modell: Situation löst aus:
						\begin{itemize}
							\item Neg. Selbstverbalisation $\rightarrow$ Angst, Unsicherheit $\rightarrow$ Vermeidung, 
								Flucht
							\item Pos. Selbstverbal. $\rightarrow$ Zuversicht $\rightarrow$ in Situation gehen
						\end{itemize}
					\item Verhaltensweise $\rightarrow$ Gewohnheit $\rightarrow$ Persönlichkeit
					\item Aufbau
						\begin{itemize}
							\item Einführungsveranstaltung
							\item PMR zur Angstkontrolle
							\item Beispiele für pos. / neg. Selbstverbal.
							\item Diskrimination soz. sicher / unsicher / aggressiv
							\item Instruktion selbstsicher. Verh. (RECHT)
							\item Pos. Selbstverbal. finden
							\item Gefühle äußern (BEZIEHUNG)
							\item Sympathien gewinnen (SYMPATHIE)
							\item Differenzierter Einsatz je nach Ziel

						\end{itemize}
					\item R-Typ-Situationen (Recht)
						\begin{itemize}
							\item Durchsetzen von Forderungen
							\item Durch Konventionen legitimiert
							\item Klar, präzise und verhaltensnah formulieren
						\end{itemize}
					\item B-Typ-Situationen (Beziehung)
						\begin{itemize}
							\item Einigung als Ziel
							\item Argumentation eher über eig. Gefühle / Bedürfnisse statt Normen
							\item Interesse anderen zu verstehen
							\item Emotionen ansprechen und in Wunsch münden lassen
						\end{itemize}
					\item S-Typ-situationen (Sympathie)
						\begin{itemize}
							\item Gegenüber hat Legitimation oder gute Beziehung im Fokus
							\item Flexibles Reagieren
							\item Gemeinsame Interessen herausstellen und Wunsch nach mehr Kontakt formulieren
						\end{itemize}

				\end{itemize}
			\item Soz. Kompetenztraining als integrativer Bestandteil
				\begin{itemize}
					\item Beginn (Diagnostik usw.) $\rightarrow$ Symptomorientierte Th. $\rightarrow$ soz. Komp.train.
					\item Training als Baustein
				\end{itemize}
			\item Soziale Kompetenz bei Kindern
				\begin{itemize}
					\item Lösungssuche, Beruhigen, Perspektivenübernahme, kooperativ, Rücksicht, Sorgsamkeit
					\item Gruppentraining für aggr. Jugendliche
				\end{itemize}
			\item Wissenschaftliche Bewertung
				\begin{itemize}
					\item Grawe: sehr wirksam bei soz. unsicheren Personen, überlegen zu GT und psychodyn.
					\item Kritisch bei sehr starken soz. Ausgangsdefiziten, Persönlichkeitsstörungen, (Agora)Phobie
				\end{itemize}
		\end{itemize}

	\section{Problemlösetrainings}
		\begin{itemize}
			\item Problem: Barriere verhindert Übergang von Ist- in Soll-Zustand
			\item Klinische Relevanz
				\begin{itemize}
					\item Unfähigkeit ein Problem alleine zu Lösen häufig Therapiegrund
					\item Generelle Heuristik für Umgang mit Problemen langfristig sinnvoller
				\end{itemize}
			\item Das allgemeine Problemlösemodell
				\begin{enumerate}
					\item Konstruktive Einstellung zu Problem aktivieren
					\item Problem sorgfältig beschreiben (Ist, Soll, Barriere) und analysieren (Gründe)
					\item Ziel setzen (relevant, realistisch, konkret, $\neq$ Sollzustand)
					\item Unkritisch viele Ideen zusammentragen
					\item Ideen bewerten (gewichtete Summe von Vor- / Nachteilen)
					\item Ideen zu Plan zusammenstellen
					\item Umsetzung
					\item Erfolgsprüfung ($\rightarrow$ Verstärkung / Ursachenanalyse)
				\end{enumerate}
			\item Diagnostik
				\begin{itemize}
					\item Exploration im Gespräch
					\item Fragebögen
					\item Testverfahren und / oder Interviews wie OFD (operationalisierte Fertigkeitsdiagnostik)
				\end{itemize}
			\item Selbstmanagement Therapie
				\begin{itemize}
					\item Am allg. Problemlösemodell ausgerichtet
					\item Anhand exemplarischer Probleme lernen
					\item Modell und dahinterstehende Struktur vermitteln
				\end{itemize}
			\item PL-Trainings
				\begin{itemize}
					\item Im Gruppensetting 
					\item Explizit und exklusiv auf Problemlösekompetenz fokussiert
					\item Einleitungsphase mit begründung der Relevanz und Vorstellung des Modells
					\item Gemeinsames entwickeln von Lösungen für konkr. Probleme der Teilnehmer
				\end{itemize}
			\item Empirische Absicherung
				\begin{itemize}
					\item Als eigenständige Therapiemaßnahme, zur Stressbewältigung oder im Rahmen von komplexeren KVT-Programmen
					\item Durch Vielzahl von Studien belegt
					\item Besonders bei Depression überlegen
					\item Effektiver wenn sämtliche Schritte thematisiert wurden
				\end{itemize}
		\end{itemize}

	\section{Paar-, Familientherapie und systemische Ansätze}
		\begin{itemize}
			\item Familientherapeutische Perspektive
				\begin{itemize}
					\item Nicht Patient / Indexperson krank, sondern Interaktion mit Bezugssystem fehlerhaft
					\item Analyse von Allianzen und Koalitionen
					\item Veränderung von Rahmenbedingungen zur Entwicklung hilfreicher Kommunikations- / 
						Verhaltensformen
					\item Patient als Symptomträger (Lösungsversuch)
					\item Mobile-Modell: Veränderung an mehreren Stellen
				\end{itemize}
			\item Mailänder Modell
				\begin{itemize}
					\item Familien von Schizophrenen, kurz, sehr effektiv?
					\item Familienspiel aus Gleichgewicht bringen und Regelen verändern
					\item Zirkuläres Fragen: Wie geht es B wenn C das tut?
				\end{itemize}
			\item Wunderfrage: was würde man selbst und andere anders machen wenn Problem weg wäre?
			\item Bedeutung von Symptomen
				\begin{itemize}
					\item Ineffektive Lösung eines Problems
					\item Schutzfunktion
					\item Verschafft Macht
					\item Symbolischer Hinweis auf andere Familienprobleme
				\end{itemize}
			\item Subsysteme (strukturelle FT)
				\begin{itemize}
					\item Ehesystem $\rightarrow$ Interessen, Unterstützung $\rightarrow$ Komm. Training
					\item Elternsystem $\rightarrow$ Versorgung, Vorbild $\rightarrow$ Elterntraining
					\item Geschwistersystem $\rightarrow$ Lernprozesse, Bündnisse $\rightarrow$ als Gruppe anspr.
				\end{itemize}
			\item Kriterien von Systemen
				\begin{itemize}
					\item Besetzung und Funktionalität der Subsysteme
					\item Grenzen (durchlässigkeit)
					\item Beziehungen (Triaden)
					\item Hierarchie (funktional / dysfunktional)
					\item Entwicklungsstand
				\end{itemize}
			\item Therapeutische Haltung
				\begin{itemize}
					\item Respektvolle Allparteilichkeit
					\item Respektlosigkeit gegenüber pathogenen Ideen
					\item Ressourcenorientiert (Kompromissfähigkeit, Zugehörigkeitsgefühl usw.)
				\end{itemize}
			\item Familiendiagnostik nach Bodenmann
				\begin{itemize}
					\item Kommunikation, Problemlösung, Machtverteilung
					\item Struktur, familiäre Kohäsion, Flexibilität / Adaptivität
					\item Familiengeheimnisse, familiäre Tradierungen (Stammbaum)
				\end{itemize}
			\item Systemische Methoden / Behandlungstechniken
				\begin{itemize}
					\item Zirkuläre Fragen
					\item Familienskulptur (Beziehungen und Verhalten symbolisch darstellen)
					\item Reframing (Umdeuten von Situationen / Verhalten)
					\item Kommunikationstraining
					\item Joining (Einbinden des Therapeuten in Strukturen)
					\item Genogramm (Symbole liefern Fakten und weiche Informationen)
				\end{itemize}
			\item Familienskulptur
				\begin{itemize}
					\item Gemäß emotionaler Bindung / Stellung in Familie
					\item Definition, Charakterisierung
					\item Weitere Bearbeitung: typische Aussagen, emot. Befinden
				\end{itemize}
			\item Paartherapie für Partnerschaftsprobleme
				\begin{itemize}
					\item Paardiagnostik nach Bodenmann
						\begin{itemize}
							\item Partnerschaftszufriedenheit / -qualität
							\item Dyadische(s) Kommunikation / Coping
							\item Macht- / Rollenverteilung
							\item Bindung / Liebesstil
							\item Sexualität
							\item Vertrauen in Partner
							\item Erwartungen in PS
							\item Problembereiche
						\end{itemize}
					\item Ansätze
						\begin{itemize}
							\item Verhaltenstherapeutische Ehetherapie
							\item Integrative Verhaltenstherapie bei Paaren (Kognitions- / Verhaltensänderung, Akzeptanz)
							\item Bewältigungsorientierter Therapieansatz (Bodenmann)
							\item Emotionally-focused Couple Therapy (EFT)
						\end{itemize}
					\item Bewältigungsorienterter Ansatz nach Bodenmann
						\begin{itemize}
							\item Psychoedukative Intervention (Vrständnis von Stress)
							\item Verbesserung des individuellen / dyadischen Copings
							\item Kommunikation
							\item Akzeptanzarbeit
						\end{itemize}
					\item Merkmale unzufriedener Paare nach Bodenmann
						\begin{itemize}
							\item Negativität, lange und häufige Konflikte
							\item Häufig Missverständnisse $\rightarrow$ schnelle Eskalation
							\item Geringe Selbstöffnung
						\end{itemize}
					\item Verhaltenstherapie mit Paaren (Bodenmann)
						\begin{itemize}
							\item Einüben neuer Kommunikations- und Problemlösestrategien
							\item Sensibilisierung für Problem 
							\item Wissensvermittlung
							\item Veränderungsmotivation
							\item Training im Alltag
							\item Maßnahmen zur Aufrechterhaltung
						\end{itemize}
					\item Verhaltenstherapeutische Techniken der Paartherapie
						\begin{itemize}
							\item Komm.-, Konfl.- und Problemlösetrainings
							\item Erklärungsmodelle
							\item Hausaufgaben
							\item Gelenkte Dialoge
							\item Rollentausch
							\item Kontingenzverträge und Verstärkung
							\item Bilanzierung
							\item Reziprozitätstraining
						\end{itemize}
					\item Verhaltenstherapeuth. Studien bestätigen Wirksamkeit
				\end{itemize}
			\item Paar- und familientherapeutische Interventionen
				\begin{itemize}
					\item Depressive Störungen
						\begin{itemize}
							\item 50\% der Suizidversuche durch Partnerschaft bedingt
							\item .40 Korrelation Partnerschaftsqualität $\leftrightarrow$ Depression
						\end{itemize}
					\item Substanzgebundene Störungen
						\begin{itemize}
							\item Partner als Auslöser, Bekräftigung oder Motivation zur Verhaltensänderung
							\item Weniger Partnerschaftsprobleme verringert Rückfallrisiko
						\end{itemize}
					\item Schizophrenie
						\begin{itemize}
							\item Reduktion von Expressed Emotion
							\item Stressreduktion für Angehörige und Verbesserung der Interaktion
						\end{itemize}
				\end{itemize}
			\item Prävention für Partnerschaftsprobleme: Ein partnerschaftliches Lernprogramm
				\begin{itemize}
					\item Zufriedenheit durch Gesprächstrainings und -regeln
					\item Verhaltensanalysen
					\item Maßnahmen zur Steigerung pos. Rezipr. / Reduzierung neg. Interaktion
					\item Konfliktlösung
					\item Veränderung dysfunktionaler Kognitionen
					\item Gegenseitige Wahrnehmungsübungen
					\item Verwöhntage
				\end{itemize}
			\item Gesprächsregeln des EPL
				\begin{itemize}
					\item Sprecher: Ich, konkr. Situation / Verhalten ansprechen, themabezogen, Selbstöffnung
					\item Zuhörer: aufnehmend, Zusammenfassen, offene Fragen, Verstärkung, Emotionen rückmelden
				\end{itemize}
			\item Risikofaktoren für psych. Störungen bei Kindern
				\begin{itemize}
					\item Kindliche Faktoren (Gene / Temperament)
					\item Soziale Faktoren (Arbeitslosigkeit, Wohnverhältnisse, finanz. / soz. Unterstützung)
					\item Elterliche Faktoren (Zuwendung, inkonsist. Erziehungsverhalten, harte Bestrafung)
					\item Familiäre Faktoren (Depression, Konflikte)
				\end{itemize}
			\item Erziehungsfertigkeiten
				\begin{itemize}
					\item Positive Beziehung aufbauen (Zeit, Reden, Zuneigung)
					\item Neues Verhalten vermitteln (Modelllernen, beiläufig, Punktekarten)
					\item Wünschenswertes Verhalten fördern (Loben, Aufmerksamkeit)
					\item Umgang mit Problemverhalten (Regeln, Ignorieren, ruhige Anweisungen, Konsequenzen, Auszeit)
				\end{itemize}
			\item Arten von Familientherapie
				\begin{itemize}
					\item Eltern-Kind-Interaktionstherapie (PCIT)
					\item Multisystemische Therapie
					\item Family-based treatment for anorexia nervosa
					\item Positive Parenting Program (Triple P, Prävention)
				\end{itemize}
			\item Kontroversen und Gesetzgebung (Systemische Therapie)
				\begin{itemize} 
					\item Wirksamkeit wissenschaftlich bestätigt
					\item Einstufung als Richtlinienverfahren weiterhin unklar
					\item Nicht Teil des Leistungskatalogs der GKV (Fokus auf Störung, nicht Interaktion)
				\end{itemize}
		\end{itemize}

	\section{KVT II: Clark, Salovskis, Beck, Young}
		\begin{itemize}
			\item Clark: A cognitive approach to panic
				\begin{itemize}
					\item Äußere Reize $\rightarrow$ körperliche Empfindungen, physiolog. Veränderungen, Wahrnehmung, Gedanken, 
						Angst
					\item Gedanken: Was könnte passieren? $\rightarrow$ bisher aber noch nie passiert
				\end{itemize}
			\item Verhaltensexperimente zur Unterstützung (Hypervent., Panikbegriffe)
			\item Clark \& Wells: KVT-Modell der sozialen Phobie
				\begin{itemize}
					\item Soziale Situation (+ Vorerfahrungen) $\rightarrow$ Erwartungen
					\item $\rightarrow$ wahrgenommene Gefahr
					\item $\rightarrow$ Selbstfokussierung der Aufmerksamkeit, Angstsymptome, Sicherheitsverhalten
				\end{itemize}
			\item Kuchendiagramm in der KVT (Gründe vergleichen)
			\item Verhaltensexperiment für soz. Phob.: erhobene Hände
			\item Kognitionen bei generalisierter Angststörung (GAD) nach Breitholz
				\begin{itemize}
					\item Aufzeichnungen bei Auftreten von Ängsten
					\item Bei: soz. Konflikten, Zweifel an Kompetenz / Akzeptanz, Sorge und andere / Kleinigkeiten
				\end{itemize}
			\item Kognitionen bei GAD nach Wells
				\begin{itemize}
					\item Typ-I-Sorgen: externale Ereignisse 
					\item Typ-II-Sorgen: bezogen auf Sorgen / Gedankengänge
				\end{itemize}
			\item Kognitiver Ansatz bei GAD
				\begin{itemize}
					\item Vor- und Nachteile von Sorgen Einschätzen / Entpathologisieren
					\item Effekte von Gedankenunterdrückung veranschaulichen
					\item Worrying verschieben
					\item Thema durchdenken (vgl. Expo.)
				\end{itemize}
			\item Kognitiver Ansatz bei Hypochondrie und somatoformen Störungen
				\begin{itemize}
					\item Katastrophisierende Bewertung
					\item Intoleranz körperlicher Beschwerden
					\item Körperliche Schwäche
					\item Körperliche Missempfindungen
				\end{itemize}
			\item Pyramidentechnik bei Hypochondrie (Wahrscheinlichkeit)
			\item Kognitives Modell der Sucht nach Beck et al.
				\begin{itemize}
					\item Dysfunktionale Kognitionen $\rightarrow$ Sucht
					\item Z.B.: Entzug furhtbar, Geminschaft nur bei Drogenkonsum usw.
				\end{itemize}
			\item Besonderheiten bei Sucht
				\begin{itemize}
					\item Fehlende Problemeinsicht, Abschiebung von Verantwortung, unkontrollierbar
					\item $\Rightarrow$ Problemeinsicht schaffen und Einflussmöglichkeiten erkennen lassen
				\end{itemize}
			\item Kognitive Techniken bei Sucht
				\begin{itemize}
					\item Vierfeldertafel Vorteile / Nachteile
					\item Leben ohne Sucht vorstellen
					\item Rückfallbewältigung in sensu durchspielen
				\end{itemize}
			\item Der Rückfallprozess als Kern von AA (Marlatt)
				\begin{itemize}
					\item Hochrisikosituation
					\item Fehlen von Coping-Strategien (alternative Strategien lernen)
					\item Verminderte Selbstwirksamkeit (Entspannung)
					\item Anfänglicher Gebrauch (Rückfallübung)
					\item Abstinenzverletzungseffekt (kogn. Umstrukturierung)
					\item Rückfall
				\end{itemize}
			\item Mindfulness-Based Stress Reduction (Yoga, Atemmeditation und Achtsamkeitsübungen)
			\item Mindfulness-Based Cognitive Therapy
				\begin{itemize}
					\item Fokus auf Prozesse statt Inhalt (decentering)
					\item Einordnung von Gedanken statt Veränderung
					\item Konkrete Coping-Handlung
						\begin{itemize}
							\item Intentionale und permanente Fokussierung auf Wahrnehmung
							\item Gedanken / Gefühl als spontane Aktivität des Gehirns
							\item Mentale Notiz
							\item Liebevolles Zurückführen der Aufmerksamkeit auf Wahrnehmung
						\end{itemize}
					\item Empirisch bewährt für Depressionsrückfall
				\end{itemize}
			\item Mindfulness-Based Relapse Prevention: SOBER (Stop, Observe, Breath, Expend, React)
			\item Dialectic Behavior Therapy (DBT)
				\begin{itemize}
					\item Achtsamkeit als Core-Skill
					\item Was-Skills: Wahrnehmen, Beschreiben, Teilhaben
					\item Wie-Skills: bewertungsfrei, fokussiert, effektiv ($\neq$ richtig)
				\end{itemize}
			\item Acceptance \& Commitment Therapy (ACT)
				\begin{itemize}
					\item Acceptance: Aversives akzeptieren können
					\item Committment: Statt Schmerzvermeidung auf persönliche Werte ausrichten
					\item Kognitive Flexibilität (Hexaflex)
						\begin{itemize}
							\item Contact with present moment
							\item Values
							\item Committed Action
							\item Self-as-Context
							\item Defusion
							\item Acceptance
						\end{itemize}
				\end{itemize}
			\item Schematherapie nach Jeffrey Young
				\begin{itemize}
					\item KVT störungsspezifisch aber nicht ausreichend (Achse-II-Problematiken, diffuse Probleme)
					\item Oft spezifische Annahmen der KVT (Arbeitsbeziehung, motiviert, Wahrnehmung) nicht erfüllt
					\item Relevant bei A2: ther. Beziehung, Bearbeitung von Emot., Ursachen in Kindheit, zentrale Themen (Schema)
					\item Integrativer Ansatz (kognitiv, Verhalten, Bindungstheorie, Objektbeziehungstheorie)
					\item Für langanhaltende emot. Schwierigkeiten
					\item Entstehung von Schemata
						\begin{itemize}
							\item Angeborenes Temperament + frühe Umgebung des Kindes
							\item (nicht)erfüllung der Grundbedürfnisse $\Rightarrow$ (mal)adaptive Schemata
						\end{itemize}
					\item Schema-Domänen
						\begin{itemize}
							\item Abgetrenntheit und Ablehnung
							\item Beeinträchtigung von Autonomie und Leistung
							\item Beeinträchtigung im Umgang mit Begrenzungen
							\item Fremdbezogenheit
							\item Übertriebene Wachsamkeit und Gehemmtheit
						\end{itemize}
					\item Modi (aktuelle Funktionszustände)
						\begin{itemize}
							\item Kindliche Modi (starke Emot., pos. / neg.)
							\item Dysfunktionale Eltern-Modi (strafend, fordernd)
							\item Maladaptive Bewältigungsmodi (Erdulden, Vermeiden, Überkompensieren)
							\item Gesunder Erwachsener (Vorbild)
						\end{itemize}
					\item Phase I
						\begin{itemize}
							\item Zentrale Schemata erkennen
							\item Informationen über Schemata
							\item Verbindung zu Problemen
							\item Ursprünge der Schemata
							\item Gefühle um Schemata
							\item Dysfunktionale Bewältigungsstile identifizieren
						\end{itemize}
					\item Phase II
						\begin{itemize}
							\item Kognitiv (Neustrukturierung bzgl. Schemata)
							\item Erfahrungsorientiert (Wut / Trauer über Erfahrungen)
							\item Therapeutische Beziehung (pos. Neuerfahrungen)
							\item Verhaltensmuster ändern
						\end{itemize}
				\end{itemize}
			\item Effektivität von KVT und Entwicklungen
				\begin{itemize}
					\item Wissenschaftlich am besten untersucht
					\item V.a. bei Depression, Angst- / Panikstörung, soz. Phobien, PTSD
					\item Überlegenheit störungsspezifisch
				\end{itemize}
		\end{itemize}

	\section{KVT I: Ellis, Beck \& Meichenbaum}
		\begin{itemize}
			\item Kognitiver Ansatz: Gedanken, Wahrnehmung, Einstellungen $\rightarrow$ kognitive Umstrukturierung
			\item Kognitives Modell: Ereignis $\rightarrow$ Bewertung $\rightarrow$ Gefühl $\rightarrow$ Gedanken / Verhalten
			\item Epiktet: Nicht die Dinge, sondern die Meinungen darüber beunruhigen uns
			\item Die rational-emotive Therapie nach Ellis
				\begin{itemize}
					\item Kategorien irrationaler Annahmen nach Ellis
						\begin{itemize}
							\item Absolute Forderungen (musturbations)
							\item Globale negative Selbst- und Fremdbewertungen
							\item Katastrophendenken
							\item Niedrige Frustrationnstoleranz (I can't standitis)
						\end{itemize}
					\item A-B-C-Modell nach Ellis
						\begin{itemize}
							\item Acting Event (extern)
							\item Belief (Bewertung)
							\item Konsequenz (emotional und auf Verhaltensebene)
							\item (Sekundäres ABC: Symptomstress)
							\item Disputation irrationaler Annahmen
							\item Effekte nach Veränderung kognitiver Sichtweise
						\end{itemize}
					\item Rolle des Therapeuten
						\begin{itemize}
							\item aktiv, direktiv, rationales Modell
							\item Empathie für Person, Skepsis für Einstellung
							\item Pragmatisch
						\end{itemize}
					\item Therapeutischer Prozess
						\begin{itemize}
							\item Vermittlung der Grundlagen
							\item Assessment des Belief-Systems
							\item Disputation irrationales Annahmen
							\item Durcharbeiten zentraler Themen
							\item Vermittlung von Selbsthilfestrategien
						\end{itemize}
					\item Arten des Disputs
						\begin{itemize}
							\item Logisch: logische Widersprüche
							\item Empirisch: erfahrbarer Widerspruch
							\item Hedonistisch: negative Konsequenzen einer Bewertung aufzeigen
						\end{itemize}
					\item Techniken
						\begin{itemize}
							\item Sokratischer Dialog
							\item Vorstellungstechniken
							\item Humor / SelbstöffnungRisikoübungen
							\item Hausaufgaben
						\end{itemize}
					\item Wirksamkeit
						\begin{itemize}
							\item Gut abgesichert bei Angststörungen, soz. Unsicherheit, Depression
						\end{itemize}
				\end{itemize}
			\item Kognitive Therapie nach Beck
				\begin{itemize}
					\item Kogn. Triade de Depression: neg. Bewertung von Selbst, Umwelt, Zukunft
					\item Kognitive Fehler
						\begin{itemize}
							\item Willkürliche Schlussfolgerungen
							\item Selektive Abstraktion
							\item Übergeneralisierung
							\item Personalisierung
							\item Dichotomes Denken (alles oder nichts)
							\item $\rightarrow$ bedingt durch negative Schemata (Grundannahmen)
						\end{itemize}
					\item Kognitives Modell
						\begin{itemize}
							\item Schemata (core beliefs)
							\item Intermediate Beliefs (Annahmen, wenn-dann)
							\item Automatic thoughts (situationsspezifisch, reflexhaft)
						\end{itemize}
					\item Therapeutisches Vorgehen
						\begin{itemize}
							\item Einführung in kogn. Modell durch Entdecken
							\item Zweifel an absoluten dysfunktionalen Grundannahmen (z.B. Faktenchecks)
							\item Erarbeiten alternativer, rationaler, hilfreicher Gedanken / Annahmen
						\end{itemize}
					\item Methoden zur Exploration von Kognitionen
						\begin{itemize}
							\item Analogsituation nachspielen
							\item Als-ob-Methode
							\item In Zeitlupe vorwärts / rückwärts
							\item Spontanveränderungen in Therapie nutzen
							\item Spalten-Techniken
						\end{itemize}
					\item Spaltentechniken
						\begin{itemize}
							\item Diagnostik: Situation, Gedanken (\%), Gefühl, Verhalten, Körper
							\item Intervention: Situation, Gedanken (\%), Gefühl, Alternative Sicht (\%), Gefühl
						\end{itemize}
				\end{itemize}
			\item Offene vs. verdeckte Umstrukturierung
				\begin{itemize}
					\item Offene Umstrukturierung
						\begin{itemize}
							\item Fehlinterpretationen + Überzeugung herausarbeiten
							\item Pro und Contra Argumente sammeln + Überzeugungsratings
							\item Verhaltensaufgabe ableiten
							\item Üben
							\item Probleme: Keine auslösenden Gedanken identifizierbar, gleiches Verhalten, Schuldgefühle
						\end{itemize}
					\item Verdeckte Umstrukturierung
						\begin{itemize}
							\item Gedanken und Folgen werden markiert
							\item Frage nach Lösungsmöglichkeit und lenken des kognitiven Fokus
							\item Eingehen auf Vorschläge des Patienten
							\item Alternative Gedanken mit alternativen Bildern anreichern
						\end{itemize}
				\end{itemize}
			\item Wirksamkeit kogn. Therapie
				\begin{itemize}
					\item Bei Depressionen ähnlich wie Antidepressiva aber geringere Rückfallwahrsch.
					\item Gut belegt bei Majr Depression, gen. Angststörung, Panikstörung
				\end{itemize}
			\item KVT nach Meichenbaum
				\begin{itemize}
					\item Betonung von Selbstverbalisierung
					\item Handlungskontrolle durch internalisierte externe verbale Signale der Eltern
					\item Selbstinstruktionstraining
						\begin{itemize}
							\item Modellvorgabe unter lautem, kommentierten Sprechen
							\item Externale Anleitung
							\item Offene Selbstanleitung (laut vorsprechen)
							\item Ausblendung der Selbstanleitung (flüsternd)
							\item Verdeckte Selbstinstruktion
						\end{itemize}
					\item Stressimpfungstraining
						\begin{itemize}
							\item Informationsphase (Modell, Stressreaktionen)
							\item Übung: Vorbereitung, Konfrontation, Umgang mit Gefühl, Selbstverstärkung
							\item Anwendungsphase
						\end{itemize}
					\item Wirksamkeit: für weites Spektrum aber insbesondere Prüfungsängste, evtl. PTSD
				\end{itemize}
			\item Multimodale Verhaltenstherapie nach Lazarus
				\begin{itemize}
					\item BASIC-ID zur Problemanalyse und Veränderung
					\item Behavior
					\item Affect
					\item Sensations (Empfindungen)
					\item Imagination
					\item Cognitions
					\item Interpersonal
					\item Disposition (biologisch)
				\end{itemize}
		\end{itemize}

	\section{Gesprächstherapie}
		\begin{itemize}
			\item Geschichtliche Entwicklung
				\begin{itemize}
					\item Humanistischer Gegenentwurf zu PA und VT
					\item Fokus auf motivationale Klärung
				\end{itemize}
			\item Menschenbild und Störungskonzept nach Rogers
				\begin{itemize}
					\item Humanistisches Menschenbild (gut, sozial, Einklang)
					\item Aktualisierungstendenz: Selbstentfaltung, Erfahrungen wahrnehmen, bewerten und reflektieren
					\item Selbstaktualisierungstendenz: Selbst(wert) weiterentwickeln und neue Erfahrungen integrieren
					\item Bedürfnis nach unbed. pos. Wertschätzung
					\item Wenn Wertschätzung an Bedingung geknüpft werden best. Erfahrungen nicht mehr symbolisiert
					\item $\rightarrow$ neg. Selbstbild erschwert Selbstaktualisierung
					\item Störung als Inkongruenz von Erfahrungen und bewusster Repräsentationen
					\item Nicht integrierbare Erfahrungen werden verzerrt, verdrängt oder verleugnet
				\end{itemize}
			\item Fully functioning person
				\begin{itemize}
					\item Offen für Erfahrungen, keine Verzerrungen
					\item Bedingungslose pos. Selbsteinschätzung
					\item Fehlentscheidungen leicht korrigierbar
					\item Befriedigende soz. Interaktionen
				\end{itemize}
			\item Therapeutische Grundhaltung: Authentizität, Wertschätzung, Empathie
			\item Therapeutische Strategie
				\begin{itemize}
					\item VEE: Verbalisierung emotionaler Erlebnisinhalte (Paraphrasierung)
					\item Schaffung eines emotionalen Klimas der Selbstentfaltung (angstfreier Raum, Wertschätzung)
				\end{itemize}
			\item Umsetzung der Grundvariablen
				\begin{itemize}
					\item Bedingungsfreie Anerkennung: Interesse, Bestätigen, Solidarisieren
					\item Kongruenz: Konfrontieren, Beziehungsklären, sich selbst einbringen
					\item Empathie: Emotionen verbalisieren, konkretisierendes Verstehen, Herausarbeiten persönlicher Bedeutung
				\end{itemize}
			\item Klientenvariablen für Therapieerfolg
				\begin{itemize}
					\item Selbstexploration
					\item Experiencing
				\end{itemize}
			\item Therapiestudien vorhanden aber wenige mit definierten Achse I Störungen
			\item Weiterentwicklung der GT nach Sachse
				\begin{itemize}
					\item Zielorientierte GT
					\item Klärungsorientierte PT
					\item Patient durch dysfunktionale Schemata gesteuert
					\item Diese zugänglich machen und klären
					\item Differentielle Gesprächstherapie
				\end{itemize}
			\item Prozess- / Erlebnisorientierte Psychotherapie nach Greenberg
				\begin{itemize}
					\item Verbindung von GT (Empathie) mit Gestalttherapie (aktiv-direktiv)
					\item Betonung von Emotionen: Zugang zu emotionalen Schemata
					\item Informationsvermittlung, Handlungsvorschlag, Zwei-Stuhl-Technik
					\item Emotionale Prozesse
						\begin{itemize}
							\item Primär adaptive emotionale Reaktionen (ungelernt, direkt)
							\item Maladaptive emotionale Reaktionen (gelernt, direkt)
							\item Sekundäre reaktive emotionale Reaktionen (adaptive Reaktion verschleiert prim. Emot.)
							\item Instrumentelle sekundäre Reaktionen (Emot. unabh. von Zustand zeigen)
						\end{itemize}
				\end{itemize}
			\item Emotionsfokussierte Therapie (EFT) nach Greenberg / Elliott
				\begin{itemize}
					\item Entwickelt aus Proz. / Erl.-orientierten Therapie und klinischen Erfahhrungen
					\item Erfahrungen als emotionale Schemata gespeichert (wichtig für Bewertung)
					\item Emotionen durch Emotionen verändern
					\item Prozesshafte Grundprinzipien: Wahrnehmung, Regulation, Reflexion
				\end{itemize}
			\item Fokussing nach Gendlin
				\begin{itemize}
					\item Inneren Freiraum schaffen (Entspannung)
					\item Problem vorstellen und felt Sense entstehen lassen (Aufmerksamkeit auf Körperreaktion)
					\item Felt Sense mit Wort / Bild / Symbol beschreiben
					\item Stimmigkeit zw. Sense und Symbol überprüfen
					\item Fragen stellen: Warum wird sense ausgelöst? usw.
					\item Heilenden Prozess annehmen und schützen
				\end{itemize}
			\item Wissenschaftliche Fundierung von GT
				\begin{itemize}
					\item Wissenschaftlicher Beirat: ja aber für zu wenig Störungsbilder (affektiv, Angst, Belastung, Anpassung)
					\item Gemeinsamer Bundesausschuss: nur 1 Studie für GT $\rightarrow$ nicht zugelassen
				\end{itemize}
		\end{itemize}

	\section{Interpersonelle Therapie \& CBASP}
		\begin{itemize}
			\item 
		\end{itemize}

	\section{Integrative Ansätze}
		\begin{itemize}
			\item 
		\end{itemize}

	\section{Psychotherapieforschung}
		\begin{itemize}
			\item 
		\end{itemize}



\end{document}
