\documentclass[11pt, paper=a4, twocolumn]{scrartcl}

\usepackage[ngerman]{babel}
\usepackage[utf8]{inputenc}

\usepackage[T1]{fontenc}
\usepackage{mathpazo}

\usepackage{geometry}

\usepackage{mathabx}

\geometry{a4paper, top=20mm, left=15mm, right=15mm, bottom=20mm,
headsep=5mm, footskip=12mm}


\pagenumbering{gobble}

\title{\vspace{-1.25cm}Zusammenfassung Klinische Psychologie\vspace{-0.25cm}}
\date{\vspace{-5ex}}

\newcommand*{\Z}{\mathbb{Z}}

\begin{document}
	\maketitle


	\section{Schizophrenie}
		\begin{itemize}
			\item 
		\end{itemize}

	\section{Somatoforme Störungen}
		\begin{itemize}
			\item 
		\end{itemize}

	\section{Rahmenbedingungen von Psychotherapie}
		\begin{itemize}
			\item Definitionen
				\begin{itemize}
					\item Wissenschaftlich anerkannte therapeutische Verfahren
					\item Feststellung, Heilung oder Linderung von Störungen mit Krankheitswert
					\item Interaktionaler Prozess mit psychologischen Mitteln
				\end{itemize}
			\item Rechtliche Rahmenbedingungen
				\begin{itemize}
					\item Psychotherapeutengesetz: Approbation nötig, regelt Voraussetzungen
					\item Gesetzliche Leistungserstattung nur mit Kassenzulassung
				\end{itemize}
			\item Ausbildung zum Psychologischem Psychotherapeuten (PPT)
				\begin{itemize}
					\item 3 / 5 Jahre
					\item Praktische Tätigkeit: Psychiatriejahr (Diagnostik und Behandlung) + Behandlungsstunden
					\item Theoretische Fortbildung: Vertiefung und praktische Übung
					\item Selbsterfahrung: eigenes Verhalten, Denken, Fühlen überprüfen und pos. verändern
					\item Supervision: Identifizieren und Üben von Optimierungsmöglichkeiten
				\end{itemize}
			\item Antragspflicht: notwendig, zweckmäßig und wirtschaftlich
			\item Gutachterverfahren für LZT
				\begin{itemize}
					\item Liegt Krankheit vor?
					\item Psychotherapie indiziert?
					\item Geplante Therapie zweckmäßig, wirtschaftlich und hinreichend gute Prognose?
					\item Rahmenbedingungen eingehalten?
				\end{itemize}
			\item Gliederung: Beschwerden, Vorgeschichte, psychische / somatische Symptomatik, Störungsmodell, ICD-10, Ziele, Therapieplan
			\item Gutachter gibt Empfehlung, Kasse entscheidet
			\item Stundenkontingente\\
				\begin{tabular}{c|cccc}
					& kurz & lang & max & $\diameter$ (Schulz) \\
					\hline Verhaltensth & 24 & 60 & 80 & 38.6 \\
					Tiefenpsych. & 24 & 60 & 100 & 53.7 \\
					Psychoanalyse & - & 160 & 300 & 106.5
				\end{tabular}
			\item Therapeutische Versorgung
				\begin{itemize}
					\item Ambulant (Praxis, Ambulanzen, Tageskliniken)
					\item Stationär (Psychosomatik, Rehabilitation, Psychiatrie)
					\item 32\% psychische Störungen, davon 63\% nicht versorgt, 18\% der Versorgten bei PPT
					\item Bei Depressionen ca. 50\%
					\item Starker Ost-West Unterschied
					\item Versorgangsgrad ca. 46\%
					\item AU-Tage steigen konstant (11-69\%), Frauen höher als Männer
				\end{itemize}
			\item Ansatz für Approbation nach Studiumsabschluss in Diskussion
			\item Rechtliche Rahmenbedingungen
				\begin{itemize}
					\item Ausbildungs- und Prüfungsverordnung
					\item Berufsordnung des jeweiligen Bundeslands
					\item Landespsychotherapeutenkammer (verfasst Berufsordnung)
					\item Bundespsychotherapeutenkammer
					\item Kassenärztliche Vereinigung
					\item Gemeinsamer Bundesausschuss (erarbeitet PT-Richtlinien)
					\item Schweigepflicht und Datenschutz
				\end{itemize}
		\end{itemize}

	\section{Therapiemotivation}
		\begin{itemize}
			\item Therapiemotivation vs. Veränderungsmotivation
			\item Misserfolg $\leftrightarrow$ Motivationsbeeinträchtigung
			\item Duales Therapiemodell nach Schulte und Eifert
				\begin{itemize}
					\item Methodenstrang
						\begin{itemize}
							\item Identifizieren von Bedingungen
							\item Auswahl von Strategien und Techniken
							\item Umsetzung
						\end{itemize}
					\item Motivationsstrang
						\begin{itemize}
							\item Prüfen von Therapie- / Veränderungsmotivation
							\item Identifizieren von motivationalen Problemen
							\item Einsatz von Methoden zur Motivationsstärkung
						\end{itemize}
				\end{itemize}
			\item Motivationsstärke = ((Nutzen - Kosten) * Durchführbarkeit) + Entschluss der Ausführung
			\item Präskriptives Entschlussmodell
				\begin{itemize}
					\item Option gut genug? Ja $\rightarrow$ Dominanzbildung, Planung, Handlung
					\item Nein $\rightarrow$ suche nach neuer Opt. und Antizipation aversiver Konsequenzen
					\item Nein + emotionale Selbsteffizienz und Einsicht in Ausschließlichkeit, Opfernotwendigkeit 
						$\rightarrow$ Option beste verfügbare?
					\item Ja $\rightarrow$ Dissonanzreduktion (Rechtfertigung durch Langzeitpräferenzen, kognitive Repräsentation) 
						$\rightarrow$ Handlung
				\end{itemize}
			\item Vor- und Nachteilsanalysen
				\begin{itemize}
					\item Tabellenartige Auflistung für spezifische Situation
					\item Kurzfristig: Vorteile (jemand der zuhört) vs. Nachteile (Aussetzen von Ängsten)
					\item Langfristig: Vorteile (Überwindung von Ängsten) vs. Nachteile (weniger Schonung / Ausreden)
				\end{itemize}
			\item Stadien der Veränderungsbereitschaft
				\begin{itemize}
					\item Precontemplation (Problem nicht erkannt)
					\item Contemplation (Veränderungsabsicht)
					\item Determination / Preparation (Schritte planen)
					\item Action (Veränderung vollziehen)
					\item Maintenance (Aufrechterhalten und Rückfall vermeiden)
					\item Relapse (Rückfall bewältigen)
				\end{itemize}
			\item Motivierende Gesprächsführung
				\begin{itemize}
					\item Annahmen: Autonmieverletzung führt zu Reaktanz, ambivalent motiviert, pos. Motive stärken, neg. 
						wertschätzen
					\item 2 Phasen: Aufbau von Veränderungsbereitschaft $\rightarrow$ Stärkung der Selbstverpflichtung
					\item Prinzipien
						\begin{itemize}
							\item Empathie / Wertschätzung: dysfunktionales Verhalten für legitime Ziele
							\item Herausarbeiten von Diskrepanzen: Konsequenzen vs. Ziele
							\item Geschmeidiger Umgang mit Widerstand
							\item Stärkung der Selbsteffizienz: Zuversicht auf Beeinflussung
						\end{itemize}
					\item Interventionen
						\begin{itemize}
							\item Offene Fragen
							\item Aktives und empathisches Zuhören (Paraphrasieren, Zusammenfassen, usw.)
							\item Würdigung (Anerkennen und wertschätzen ALLER Leistungen)
							\item Informationsvermittlung
							\item Non-Direktivität / geleitetes Entdecken
							\item Förderung von change talk (Verstärken von Aussagen mit Veränderungsabsicht)
							\item Konstruktiver Umgang mit Widerstand (überzogenes Widerspiegeln)
							\item Förderung von Handlungsorientierung (konkrete Ziele, commitment fördern)
						\end{itemize}
				\end{itemize}
			\item EPOS (Elaboration positiver Perspektiven)
				\begin{itemize}
					\item Zwei Sitzungen
					\item Vorstellen eines Tags in 5 Jahren, wenn alles nach seinem Willen gelaufen ist
					\item Offene Fragen, mehrere Sinneskanäle
					\item Audioaufnahme für Patient
					\item Zweite Sitzung: imaginativ-holistisch $\rightarrow$ analytisch-zielgerichtet
					\item Zentrale Bilder und dahinterliegende Wünsche herausarbeiten
					\item Wünsche langfristige Ziele?
					\item $\rightarrow$ mittelfristige, kurzfristige Ziele
				\end{itemize}
		\end{itemize}

	\section{KVT-Intro \& Entspannung}
		\begin{itemize}
			\item KVT
				\begin{itemize}
					\item Empirisch belegte Methoden und Theorien
					\item Gegenwarts-, Handlungs-, Problem- und Zielorientiertheit
				\end{itemize}
			\item PMR
				\begin{itemize}
					\item Psychophysiologischer Entspannungsprozess durch Kontrasterleben
					\item Kurzform vs. Langform
					\item Einsatzmöglichkeiten
						\begin{itemize}
							\item Regelmäßige Entspannung (anhaltende Reduktion der Anspannung)
							\item Angewandte Entspannung (Stressbewältigung)
							\item Systematische Desensibilisierung (stufenweise Konfrontation in sensu, reziproke Inhibition)
						\end{itemize}
					\item Ablauf: Einführung, Verkürzung, ohne Anspannung, Selbstinstruktion, verschiedene Kontexte, schnell, 
						Anwendung
					\item Effektivität
						\begin{itemize}
							\item Durch Studien belegt: Anspannungsreduktion, chronische Schmerzzustände, Hypertonie, 
								Angststörung, Schlafstörung, psychosomatische Beschwerden
							\item System. Desens. nachgewiesen aber nur wenn massive Konfrontation nicht möglich
							\item Effektstärken $d=0.58$
						\end{itemize}
				\end{itemize}
			\item Biofeedback
				\begin{itemize}
					\item Rückmeldung physiologischer Parameter durch Geräte
					\item Patient soll Signale beeinflussen $\rightarrow$ auch psych. Prozesse
					\item Evtl. Vermittlung bestimmter Technik
					\item Lernen Beschwerden direkt (Blutdruck) / indirekt (generelle Entspannung) zu beeinflussen
					\item Beeinflussbar: Muskelspannung, Hautleitfähigkeit, Arteriendurchmesser der Schläfenartherie, usw.
					\item Psychophysiologische Überaktivierung als Störungskorrelat
						\begin{itemize}
							\item Erhöhter Grundtonus und Reaktivität
							\item Verringerte Habituation bei Wiederholung
							\item Verzögerte Erholungsphasen
							\item Erhöhte Dishabituation / Reaktivität
						\end{itemize}
					\item Ablauf
						\begin{itemize}
							\item Identifikation von Zielparametern
							\item Eingangsdiagnostik, Mutlikanal-Messung und Einflussübung
							\item Trainingssitzungen (Baseline, Training, Generalisierung)
							\item Kombination mit anderen Interventionen
							\item Abschluss
						\end{itemize}
					\item Metaanalyse bei Migräne belegt stabilen Effekt von ca. $d=0.56$
					\item Spannungskopfschmerzen auch per Metastudie belegt $d=.82$
				\end{itemize}
			\item Imaginative Entspannung
				\begin{itemize}
					\item Verschiedene Verfahren mit unterschiedlichen Zielen
					\item Z.B. Fantasiereise zur Anregung von individuellen Assoziationen und Interpretationen
					\item Problematisch bei gering ausgeprägtem Vorstellungsvermögen
					\item Keine ausreichenden empirischen Belege
				\end{itemize}
			\item Hypnose
				\begin{itemize}
					\item Trancezustand: minimaler Widerstand / maximiale Akzeptanz
					\item Entspannungselemente aber auch weitere Zielsetzungen (Neubewertung, Veränderung von Reaktionsmustern)
					\item Einengung der Wahrnehmung durch Augenfixation oder Suggestion
					\item Hypnotisierbarkeit als Persönlichkeitsmerkmal
					\item Empirische Fundierung
						\begin{itemize}
							\item Gut bei chronischen Schmerzen
							\item Positiv bei Rauchstopp, Schlafstörungen, Angststörungen
							\item Wenig bei Abhängigkeitserkrankungen und Bluthochdruck
						\end{itemize}
				\end{itemize}
			\item Autogenes Training
				\begin{itemize}
					\item Anwender leitet Entspannungszustand selbst ein
					\item Durchführung
						\begin{itemize}
							\item 6 Übungen (Schwere-, Wärme-, Herz-, Atmungs-, Sonnengeflechts-, Stirnkühleübung)
							\item Formeln für relevante Prozesse zur Beeinflussung (Arme schwer usw.)
						\end{itemize}
					\item Empirische Fundierung
						\begin{itemize}
							\item Weniger eindeutig als PMR
							\item Mittlere Effekte für u.a. Kopfschmerzen, Hypertonie, Schlafstörungen
						\end{itemize}
				\end{itemize}
			\item Indikation
				\begin{itemize}
					\item Angst-, Schlaf- und somatoforme Störungen
					\item Geringe Wahrscheinlichkeit von Nebenwirkungen
				\end{itemize}
			\item Kontraindikationen: Hypotonie (niedriger Blutdruck), Herz- oder Atembeschwerden, akute Psychosen
			\item Probleme
				\begin{itemize}
					\item Sexuelle Missbrauchsopfer (Panikattacken durch Setting)
					\item Panikattacken generell (Vermeidung durch Entspannung)
					\item Angst vor Kontrollverlust
				\end{itemize}
		\end{itemize}

	\section{Konfrontationsverfahren}
		\begin{itemize}
			\item Einteilung
				\begin{itemize}
					\item Graduiert: System. Desensibilisierung (sensu), Habituationstraining (vivo)
					\item Massiert: Implosion (sensu), Flooding (vivo)
				\end{itemize}
			\item Systematische Desensibilisierung
				\begin{itemize}
					\item Prinzip der reziproken Hemmung
					\item Anwendung
						\begin{itemize}
							\item Hierarchisierung der Angstsituationen
							\item Lernen von Entspannung
							\item Vorstellung und Entspannung
							\item Wenn angstfrei Aufstieg
						\end{itemize}
					\item Wirkmechanismus hauptsächlich nur Außeinandersetzung mit Angst, nicht Hierarchie
				\end{itemize}
			\item Durchführung
				\begin{itemize}
					\item Diagnostik, Verhaltens- und Bindungsanalyse
					\item Kognitive Vorbereitung: Störungsmodell $\rightarrow$ therapeutisches Vorgehen
					\item Bewusste und freiwillige Entscheidung mit mental präsentiertem Grund
					\item Intensiv-Expo begleitet
					\item Expo allein
					\item Generalisierungstraining
				\end{itemize}
			\item Begründungsmodelle
				\begin{itemize}
					\item VT-Vermeidungs-Rational: Aufrechterhalten von Angst durch Vermeidung, da keine Habituation
					\item Kogn. Rational I: Falsche Bewertung des Stimulus kann durch Vermeidung nicht geprüft werden
					\item Kogn. Rational II: Angst durch Bewertung $\rightarrow$ bewertungsfrei wahrnehmen
				\end{itemize}
			\item Erwartete Angstverläufe als Vorbereitung
			\item Regeln für Exposition
				\begin{itemize}
					\item Zulassen der Ängste
					\item Verzicht auf Vermeidungsstrategien
					\item Angst beobachten ohne zu bewerten
					\item Achten auf kleine Veränderungen
					\item Unterscheidung Realität $\leftrightarrow$ Phantasie
					\item In Situation bleiben bis Angst abfällt
				\end{itemize}
			\item Therapeutenverhalten
				\begin{itemize}
					\item Angst beobachten und auf Skala angeben lassen (auch körperliche Symptome)
					\item Nach Vermeidungsstrategien fragen
					\item Fragen wie man Angst steigern kann und umsetzen
					\item Hinweise auf Angstabfall beschreiben lassen und versuchen trotzdem zu steigern
				\end{itemize}
			\item Weitere Regeln
				\begin{itemize}
					\item Genügend Zeit, kontrollierbare Situation
					\item Erst bei deutlichem Angstabfall beenden
					\item Konstruktive Nachbewertung
					\item Problem: unangenehme Übung als Beleg für Angst
					\item Expositionsrational: Weg aus Angst durch Angst
					\item Kognitives Rational: Bewertung kann sich nur ändern wenn Chance zu sehen dass Erwartungen nicht eintreffen
					\item Aufstellen und Bewerten von Hypothesen
				\end{itemize}
			\item Massiert vs. graduiert: ähnliche Erfolge aber massiert stabiler
			\item Individualisierte vs. standardisierte Pläne: standard besser, v.a. bei unerfahrenen Therapeuten
			\item Studien zur Panikbehandlung
				\begin{itemize}
					\item Barlow et al.: CBT > CBT + PME > PME
					\item Clark et al.: CBT mit Expo > Antidepressiva > graduierte Expo
					\item Metaanalyse: Konfrontation in vivo > kogn. behav. > kogn. > GT
					\item Rief et al.: Intensiv Expo. effektiver als kürzere
				\end{itemize}
			\item Exposition bei Zwangsstörungen
				\begin{itemize}
					\item Problem: Zwangshandlungen nach Expo als Ausgleich
					\item Lösung: Mehr und längere Expos mit Reaktionsverhinderung
					\item Oft nur bis 50\% Angst
				\end{itemize}
			\item Expo bei PTSD
				\begin{itemize}
					\item Problem: Trauma nicht direkt konfrontierbar
					\item Lösung: In sensu und Trauma-assoziierte Stimuli in vivo
					\item Prolonged Exposure: Imagination, Bericht, Tonbandaufzeichnung, in vivo Expo
					\item Stress-Impfungstraining: Entspannung, Rollenspiele
					\item Beste Ergebnisse bei PE
				\end{itemize}
			\item Expo bei Essstörungen
				\begin{itemize}
					\item Expo mit Körper (Spiegel), Nahrungsmittel $\rightarrow$ Prüfung irrationaler Kognitionen
				\end{itemize}
			\item Indikationen für graduiertes Vorgehen
				\begin{itemize}
					\item Körperliche Komplikationen
					\item Fehlende Motivation
				\end{itemize}
		\end{itemize}

	\section{Operante Verfahren und Modelllernen}
		\begin{itemize}
			\item Operante Verfahren
				\begin{itemize}
					\item Arten operanter Konditionierung
						\begin{itemize}
							\item Positiv: Belohnung (Erhöhung), Bestrafung (Verringerung)
							\item Negativ: Flucht (beenden) / Vermeidung (nicht erfolgen), Time Out / Löschung
						\end{itemize}
					\item Verstärkerarten
						\begin{itemize}
							\item Positive vs. negative Verstärker
							\item Primäre Verstärker: Erfüllen Grundbedürfnisse (abhängig von Sättigung)
							\item Sekundäre (konditionierte) Verstärker: assoziiert mit prim. Verstärker (z.B. Lob, 
								Aufmerksamkeit)
							\item Generalisierte Verstärker: Tauschwert (Token)
							\item Soziale Verstärkung: Gratifikation / Sanktion
							\item Selbstverstärkung: z.B. explizites Selbstlob
						\end{itemize}
					\item Therapeutisches Vorgehen
						\begin{itemize}
							\item Shaping: Aufbau des Zielverhaltens
							\item Chaining: ähnlich zu Shaping aber Verstärkung des Abschlusses, nicht Beginn
							\item Fading: Verstärkung des Zielverhaltens unter schrittweiser Ausblendung der Hilfsstimuli
							\item Prompting: Unterstützung des Beginnns der Verhaltensänderung durch (non)verbale 
								Hilfestellung
							\item Intermittierende Verstärkung: (variables) Festigen bereits gezeigten Verhaltens
							\item Direkte Bestrafung: selten langfristig, negative Auswirkung auf Beziehung
							\item Indirekte Bestrafung: Verstärkerentzug bei dysfunkt. Verhalten
							\item Löschung: Verhindern pos. Konsequenzen von neg. Verhalten
							\item Time-Out
							\item Verhaltens- oder Kontingenzverträge: Verhalten $\rightarrow$ Konsequenz
						\end{itemize}
					\item Stimuluskontrolle durch Hinzufügen / Entfernen von Hinweisreizen
					\item Indikation
						\begin{itemize}
							\item Verstärkungspläne für Depression
							\item Bei Abhängigkeiten, Kindern, Anorexie, Schlafstörungen, Demenz
							\item Bei eingeschränkten kognitiven Fähigkeiten
						\end{itemize}
				\end{itemize}
			\item Modelllernen (vgl. Bandura)
				\begin{itemize}
					\item Therapeutisches Vorgehen
						\begin{itemize}
							\item Modell führt Verhalten aus und wird verstärkt (für Patient sichtbar)
							\item Patient führt Verhalten selbst aus und wird ebenfalls verstärkt
							\item Ähnlichkeit von Modell und Patient wichtig
						\end{itemize}
					\item Effektivität
						\begin{itemize}
							\item Mastery-Modell: Modell problemlos $\rightarrow$ Scham bei Patient
							\item Coping-Modell: Modell überwindet anfängliche Probleme
							\item Transfer in Alltag durch Entkopplung und steigende Einsicht des Patienten
							\item Als Mittel durch empirische Untersuchungen belegt aber keine Belege für eigenständige 
								Methode
						\end{itemize}
				\end{itemize}
			\item Rollenspiele
				\begin{itemize}
					\item Operante Techniken und Modelllernen
					\item Diagnostisches vs. therapeutisches Rollenspiel
					\item Problembeschreibung, Festlegen der Situation, Durchführung, Auswertung, Erneutes Spielen und Auswerten, 
						Transfer
				\end{itemize}
		\end{itemize}

	\section{Psychodynamische Ansätze}
		\begin{itemize}
			\item 
		\end{itemize}

	\section{Soziale und emotionale Kompetenz-, Kommunikations- und Problemlösetrainings}
		\begin{itemize}
			\item 
		\end{itemize}

	\section{KVT III: Ellis, Meichenbaum, Beck, Lazarus, Clark, Salkovis}
		\begin{itemize}
			\item 
		\end{itemize}

	\section{Paar, Familientherapie und systemische Ansätze}
		\begin{itemize}
			\item 
		\end{itemize}

	\section{KVT IV: Achtsamkeitsbasierte Verfahren Young; Schematherapie}
		\begin{itemize}
			\item 
		\end{itemize}

	\section{Gesprächstherapie}
		\begin{itemize}
			\item 
		\end{itemize}

	\section{Interpersonelle Therapie; CBASP}
		\begin{itemize}
			\item 
		\end{itemize}

	\section{Integrative Ansätze}
		\begin{itemize}
			\item 
		\end{itemize}

	\section{Psychotherapieforschung}
		\begin{itemize}
			\item 
		\end{itemize}



\end{document}
