\documentclass[11pt, paper=a4, twocolumn]{scrartcl}

\usepackage[ngerman]{babel}
\usepackage[utf8]{inputenc}

\usepackage[T1]{fontenc}
\usepackage{mathpazo}

\usepackage{geometry}

\geometry{a4paper, top=20mm, left=15mm, right=15mm, bottom=20mm,
headsep=5mm, footskip=12mm}


\pagenumbering{gobble}

\title{\vspace{-1.25cm}Zusammenfassung Klinische Psychologie\vspace{-0.25cm}}
\date{\vspace{-5ex}}

\newcommand*{\Z}{\mathbb{Z}}

\begin{document}
	\maketitle


	\section{Schizophrenie}
		\begin{itemize}
			\item 
		\end{itemize}

	\section{Somatoforme Störungen}
		\begin{itemize}
			\item 
		\end{itemize}

	\section{Rahmenbedingungen von Psychotherapie}
		\begin{itemize}
			\item 
		\end{itemize}

	\section{Therapiemotivation}
		\begin{itemize}
			\item 
		\end{itemize}

	\section{KVT I: Entspannung, Reizkonfrontation, Exposition}
		\begin{itemize}
			\item 
		\end{itemize}

	\section{KVT II: Exposition, operante Verfahren, Modelllernen}
		\begin{itemize}
			\item 
		\end{itemize}

	\section{Psychodynamische Ansätze}
		\begin{itemize}
			\item 
		\end{itemize}

	\section{Soziale und emotionale Kompetenz-, Kommunikations- und Problemlösetrainings}
		\begin{itemize}
			\item 
		\end{itemize}

	\section{KVT III: Ellis, Meichenbaum, Beck, Lazarus, Clark, Salkovis}
		\begin{itemize}
			\item 
		\end{itemize}

	\section{Paar, Familientherapie und systemische Ansätze}
		\begin{itemize}
			\item 
		\end{itemize}

	\section{KVT IV: Achtsamkeitsbasierte Verfahren Young; Schematherapie}
		\begin{itemize}
			\item 
		\end{itemize}

	\section{Gesprächstherapie}
		\begin{itemize}
			\item 
		\end{itemize}

	\section{Interpersonelle Therapie; CBASP}
		\begin{itemize}
			\item 
		\end{itemize}

	\section{Integrative Ansätze}
		\begin{itemize}
			\item 
		\end{itemize}

	\section{Psychotherapieforschung}
		\begin{itemize}
			\item 
		\end{itemize}



\end{document}
