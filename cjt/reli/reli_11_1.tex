\documentclass[11pt, paper=a4, twocolumn]{scrartcl}

\usepackage[ngerman]{babel}
\usepackage[utf8]{inputenc}

\usepackage[T1]{fontenc}

\usepackage{geometry}

\geometry{a4paper, top=20mm, left=15mm, right=15mm, bottom=20mm,
headsep=5mm, footskip=12mm}


\pagenumbering{gobble}

\title{\vspace{-1.25cm}Religion 11-1\vspace{-0.25cm}}
\date{\vspace{-5ex}}
\author{samjaza}


\begin{document}
	\maketitle


	\section{Definition}

	\begin{enumerate}
		\item Religio:
			\begin{enumerate}
				\item Re-eligere $\rightarrow$ Entscheidung / Bekehrung
				\item Re-legere $\rightarrow$ Auseinandersetzung, bedenken, nachdenken
				\item Re-ligare $\rightarrow$ Festmachen, Standpunkt, Halt
			\end{enumerate}
		\item Römer:
			\begin{enumerate}
				\item Gewissenhaftigkeit
				\item Offizieller Kult
				\item Gottesfurcht (Respekt)
				\item Zusammenhang: Handeln - Folgen
			\end{enumerate}
		\item Grunddimensionen
			\begin{enumerate}
				\item Transzendenz (Sinnfrage, Götterglaube, Gottesvorstellung, Jenseits)
				\item Kult (Gemeinschaft, Gottesdienst, Meditation / Gebet, Sakramente)
				\item Ethik (Ge- / Verbote, Alltagshandeln, Organisation der Gesellschaft, soziales Handeln)
			\end{enumerate}
	\end{enumerate}
(Beispiel: Faust)

	\section{Transzendenz $\leftrightarrow$ Immanenz}
	
	\begin{itemize}
		\item Transzendenz:
			\begin{itemize}
				\item \glqq{}hinübergehend\grqq{}
				\item Nicht mit Verstand / Wissenschaft zu erfassen, nicht in den Griff zu bekommen
				\item Nicht endgültig aussagbar
				\item Bsp.: Sinnfrage, Jenseits
			\end{itemize}
		\item Immanenz:
			\begin{itemize}
				\item \glqq{}drinnen bleibend\grqq{} $\rightarrow$ das \glqq{}weltliche\grqq{}
				\item Alles erforschbare, was begriffen werden kann
				\item Zutreffend und umfassend beschreibbar
			\end{itemize}
	\end{itemize}

	\section{Grundfragen des Menschen nach Kant}

	\begin{enumerate}
		\item Was kann ich wissen?
			\begin{itemize}
				\item Wie weit reicht der Verstand?
				\item Erkenntnistheorie als Teilbereich der Philosophie
			\end{itemize}
		\item Was soll ich tun?
			\begin{itemize}
				\item moralische Entscheidungen (Ethik)
			\end{itemize}
		\item Was darf ich hoffen?
			\begin{itemize}
				\item Mögliche Jenseitsvorstellungen
				\item Hoffnung auf Erlösung (Schulderfahrung)
				\item Eschatologie (Teilbereich Theologie) Lehre von den letzten Dingen
			\end{itemize}
		\item Was ist der Mensch?
			\begin{itemize}
				\item Mensch: ein Tier oder mehr?
				\item Anthropologie
			\end{itemize}
	\end{enumerate}
	$\rightarrow$ Woher kommen wir? Wohin gehen wir? (Religion als Antwort)

	\section{Religion und Werbung}
	
	\begin{itemize}
		\item Verwendung religiöser Motive zur Ausnutzung der Grundsehnsüchte des Menschen
	\end{itemize}

	\section{Sport als Religionsersatz}

	\begin{itemize}
		\item Kult und Riten
		\item Regeln $\rightarrow$ Gebote
		\item Hymnen
		\item Halt im Alltag
		\item Gemeinschaft
		\item Lehrsätze (Dogmen)
		\item \dots
	\end{itemize}

	\section{Griechische Philosophen}

	\begin{itemize}
		\item Sokrates: Fragen $\rightarrow$ \glqq{}gottlos\grqq{} $\rightarrow$ Giftbecher
		\item Schüler: Platon (Frage \& Antwort: \glqq{}Sokrates\grqq{} antwortet)
		\item Aristoteles andere Vorstellung
		\item Platon: \glqq{}Idee\grqq{} $\rightarrow$ Umsetzung
		\item Aristoteles: Dinge $\rightarrow$ Untersuchung und \glqq{}Idee\grqq{}
	\end{itemize}

	\section{Platons Höhlengleichnis}

	\begin{enumerate}
		\item Menschen in Höhle gefesselt, sehen an Wand Schatten (wenig Erkenntnis)
		\item Einer befreit: geblendet von Feuer, erkennt Höhle
		\item Verlassen der Höhle: nur mit Zwang, nach Zeit erkennen von Teilen der Außenwelt
		\item Versuch der \glqq{}Bekehrung\grqq{} der anderen:\\
			Zurückweisung, Sehnen nach Ausgangssituation
	\end{enumerate}
	$\Rightarrow$ Religion: Ausgang aus Höhle: Leben nach Tod (Erleuchtung)

	\section{Heutige Bedeutung der Religion}

	\begin{itemize}
		\item Interessen der Gesellschaft gerichtet auf:
			\begin{itemize}
				\item Bedürfnisbefriedigung
				\item Wohlstand
				\item Freiheit von Bevormundung
				\item Materiell messbare Ebene
				\item Trennung öffentlich - privat
			\end{itemize}
		\item Religion
			\begin{itemize}
				\item Privatsache
				\item Verliert an sozialer Bedeutung
				\item $\Rightarrow$ Befindet sich im Pluralismus
			\end{itemize}
	\end{itemize}
	$\Rightarrow$ Religiöse Moral / Ethik beschränkt sich auf Privatbereiche\\
	$\Rightarrow$ Gestaltungskraft des Glaubens in der Gesellschaft geht zurück

	\section{Verhältnis der Religionen zueinander}

	\begin{itemize}
		\item Anhänger:\\
		Christen > Islam > Hinduismus > Buddhismus> Judentum
		\item Alter:\\
		Hinduismus > Judentum > Buddhismus > Christentum > Islam
		\item Geschichte Buddha: Blinde \glqq{}ertasten\grqq{} Elefant $\rightarrow$ unterschiedliche Ansichten
		\item Ringparabel (aus religiösen Gründen menschenfreundlich $\rightarrow$ die \glqq{}Wahre\grqq{})
		\item Haltung der katholischen Kirche:
			\begin{itemize}
				\item 2. Vatikanisches Konzil vor 50 Jahren
				\item Anerkennung von allem \glqq{}Wahren und Heiligem\grqq{}
				\item Andere Religionen auf richtigem Weg aber Katholiken kennen das \glqq{}Ziel\grqq{}
				\item Nostra aetate
				\item Inclusivismus
			\end{itemize}
	\end{itemize}

	\section{Mögliche Haltungen gegenüber anderen Religionen}

	\begin{itemize}
		\item Exclusivismus
			\begin{itemize}
				\item Jeder der nicht in der Religion: vom Heil ausgeschlossen
				\item Einzig wahre, andere falscher Weg
			\end{itemize}
		\item Inclusivismus
			\begin{itemize}
				\item Integration anderer Religionen
				\item Auch andere Chance auf Heil aber eigene: bester Weg
			\end{itemize}
		\item Pluralismus
			\begin{itemize}
				\item Alle stehen nebeneinander und sind gleich gültig
				\item Keine Wertung
			\end{itemize}
	\end{itemize}

	\section{Religionsfreiheit}

	\begin{itemize}
		\item Positive Religionsfreiheit
			\begin{itemize}
				\item Recht zu bekennen und zu praktizieren
			\end{itemize}
		\item Negative Religionsfreiheit
			\begin{itemize}
				\item Niemand darf zu Aussage oder Ausübung von Religion zwingen
			\end{itemize}
	\end{itemize}

	\section{Biblische Grundlagen}

	\begin{itemize}
		\item Formale Kriterien: gebundene Sprache, Sprachrhythmus, teilw. Reime, Verse, Parallelismen, Zitate altes Testament
	\end{itemize}

	\section{Aufbau}

	\begin{itemize}
		\item Neues Testament
		\item Matthäus
		\item Markus
		\item Lukas
		\item Johannes
		\item Paulusbriefe + Hebräer
		\item Sieben katholische Briefe
		\item Offenbarung Johannes
		\item Altes Testament
		\item Genesis
		\item Exodus
		\item Levitikus
		\item Numeri
		\item Deuteronomium
		\item Geschichte Israel
		\item Lehrweisheiten und Psalmen
		\item Propheten
	\end{itemize}

	\section{Hebräisch:}

	\begin{itemize}
		\item Fünf Bücher Mose: Torah
		\item Propheten: Nebüm
		\item Schriften: Ketubim
	\end{itemize}

	\section{Interpretation biblischer Texte}

	\begin{itemize}
		\item Wortwörtliches Verständnis unabhängig von der konkreten Situation und eigentlicher Aussageabsicht\\
		$\Rightarrow$ große Brutalität durch Missverständnisse\\
		$\rightarrow$ Fundamentalismus
		\item Subjektive Interpretation je nach Lebenssituation\\
		$\Rightarrow$ alle können was mit anfangen
	\end{itemize}

	\section{Wissenschaftliche Interpretation}

	\begin{itemize}
		\item Wissenschaften $\leftrightarrow$ Geisteswissenschaften
			\begin{itemize}
				\item Forscht
				\item Festgelegte Methoden (Experimente)
				\item Nachvollziehbar entstanden und nachprüfbar
				\item Kann gelehrt und vermittelt werden
			\end{itemize}
		\item Bibelexegese:
			\begin{itemize}
				\item Sprachwissenschaften
				\item Geschichte
				\item Humanwissenschaften, \dots
			\end{itemize}
	\end{itemize}

	\section{Historisch - kritische Methode (Bedeutung zur Zeit der Abfassung)}

	\begin{enumerate}
		\item Textkritik
			\begin{enumerate}
				\item Ursprache (Originaltext)
			\end{enumerate}
		\item Untersuchung der Entstehungsgeschichte
			\begin{enumerate}
				\item Literarkritik
					\begin{itemize}
						\item Unstimmigkeiten, Wiederholungen, Widersprüche, Gedankensprünge
					\end{itemize}
				\item Überlieferungskritik
					\begin{itemize}
						\item Vorschriftliche Textstadien
					\end{itemize}
				\item Quellen und Redaktionskritik
					\begin{itemize}
						\item Stoffsammlung $\rightarrow$ Evangelium
						\item Phasen schriftlicher Entstehungsgeschichte
					\end{itemize}
			\end{enumerate}
		\item Formale und inhaltliche Analyse
			\begin{enumerate}
				\item Formkritik und Gattungskritik
					\begin{itemize}
						\item Gebet, Gleichnis, \dots
						\item Sitz im Leben
						\item Für welche Situation?
					\end{itemize}
				\item Begriffs- und Motivgeschichte
					\begin{itemize}
						\item \glqq{}Sohn Gottes\grqq{}, \glqq{}Jungfrau\grqq{}
					\end{itemize}
			\end{enumerate}
		\item Bestimmung des historischen Orts\\
		(historische Einordnung in Situation, Umwelt, Adressat, wann, wo und von wem?)
		\item Deutende Zusammenfassung
	\end{enumerate}

	\section{Israel im Strom der Zeit}

	\begin{itemize}
		\item Bibel sortiert nach Wichtigkeit nicht Historie
		\item Abraham, Isaak, Jakob: EIN Gott $\rightarrow$ verwandt
		\item Bewahrung des eigenen Glauben und der eigenen Tradition gegen Eroberer
		\item Altes Testament wenig belegt, eher nacheinander
		\item Neues Testament größtenteils historisch belegbar
	\end{itemize}








\end{document}
