\documentclass[11pt, paper=a4, twocolumn]{scrartcl}

\usepackage[ngerman]{babel}
\usepackage[utf8]{inputenc}

\usepackage[T1]{fontenc}

\usepackage{geometry}

\geometry{a4paper, top=20mm, left=15mm, right=15mm, bottom=20mm,
headsep=5mm, footskip=12mm}


\pagenumbering{gobble}

\title{\vspace{-1.25cm}Religion 11-1\vspace{-0.25cm}}
\date{\vspace{-5ex}}



\begin{document}
	\maketitle


	\section{Einführung}
		\begin{itemize}
			\item Datenunabhängigkeit
			\item Schichtenmodell
		\end{itemize}


Religion 12/2
 
Ordnungsmodelle
-	Individualismus
•	Bezogen auf Einzelwesen
•	Grundaussagen:
1.	Jeder Mensch ist nur Individualwesen und steht über der Gemeinschaft
2.	Mensch von Natur aus gesellschaftsfeindlich
3.	Gemeinschaft: Summe an Individuen (Zweckvereinigung zum Eigennutzen)
4.	Gemeinschaft kein Eigenrecht; nur die von Individuen übertragenen (deligierten)
5.	Oberster Grundsatz: freies Spiel der Kräfte (grenzenlose Selbstverwirklichung)
6.	Was Einzelnen nützt, nützt auch Gesellschaft ? Summennutzen darf nicht durch Minderheit vereitelt werden
•	Wirtschaft, Kapitalismus, Liberalismus, „American Dream“
•	? Verabsolutierung der Individualnatur, Leugnung der Sozialnatur
•	Eigennutz verabsolutiert ? Utilitarismus
-	Kollektivismus
•	Grundaussagen:
1.	Mensch nur Sozialwesen ? nur Sozialnatur ? kein Einzelwert ? keine eigenen Rechte ? Gemeinschaft unbedingt über Individuum
2.	Höchster sozialer Wert: Kollektiv; ? Einzelner abhängig davon
3.	[…]
•	Gemeinschaft absolutiert
•	Eigener Rechte und Würde beraubt (muss sich in wirtschaftliche und soziale Prozesse einfügen)
•	Absicherung
-	Personalismus
•	Beruht auf christlichem Menschenbild
•	Grundaussagen:
1.	Mensch Ebenbild Gottes (Gen 1, 27) ? besteht aus Materie + Geist
? Selbsterkenntnis und Selbstbestimmung
2.	? Individualnatur & Sozialnatur
•	Vermeidet Einseitigkeiten der anderen beiden Modelle

Das Menschenbild der katholischen Soziallehre
-	Mensch
•	Kind Gottes (Übernatur)
?	Bindung an Sittengesetz und Gewissen
?	Unveräußerlicher und unantastbarer Eigenwert
?	? Individualnatur
•	Ebenbild Gottes
?	Vernunft, Selbstständigkeit, Naturrechte, usw.
?	Bestimmt, befähigt und verpflichtet in Gesellschaft zu leben
?	? Sozialnatur
•	? Person (vereint Natur und Übernatur) ? Personalismus
Grundprinzipien der katholischen Soziallehre
-	Grundsätze die Verhältnis zwischen Einzelnem und Gesellschaft gelingen lassen
-	Personprinzip / Personalitätsptinzip (~Individualismus)
•	Person Maßstab des gesellschaftlichen Lebens
•	Gemeinschaft aus und für Individuen
•	Gesellschaft und Staat sind für den Einzelnen da, nicht umgekehrt
-	Solidaritätsprinzip
•	Oswald von Nell-Bräuning
•	Grundaussage: Wechselseitige Verbundenheit und Verantwortlichkeit von Individuum und Staat aufgrund von Sozialnatur
•	Sein-, Sollens- und Rechtsgrundsatz
?	? Einzelner und Gesellschaft einander zugeordnet und voneinander abhängig
(„Alle in einem Boot“)
?	Sollensgrundsatz (man SOLL sich verantwortlich zeigen)
(„Einer für alle …“)
?	Individuum und Gesellschaft haben Rechtsanspruch auf solidarische Zusammenarbeit
•	Würdigung:
?	Motive für Solidarität
     Eigeninteresse
     Moralisch
?	Vorteil: es wird jemand geholfen; Respekt & Wertschätzung
-	Gemeinwohlprinzip (bonum comunae)
•	Alle Werte, die zum individuellen und sozialem Leben der Menschen in einer Gesellschaft unbedingt nötig sind
•	Individualwohl nur mit gemeinsamen Mitteln erreichbar
•	Grundaussagen:
1.	Unter bestimmten Bedingungen Gemeinnutz > Eigennutz
2.	Gemeinwohl eigene Wirklichkeit (>SEinzelwohle)
3.	Gemeinwohl Vorrang soweit, solange man Mitglied dieser Gesellschaft ist
4.	Minderheitenschutz (Grenze in Grundrechten und Menschenwürde)
5.	Setzt Vorhandensein einer allgemein anerkannten Autorität voraus
•	Würdigung
?	Schutz des Gemeinwohl vor egoistischen Einzelinteressen
?	Entwicklungsfähigkeit der Gemeinschaft (muss nicht alles schon besitzen)
?	Vergleich mit Körper (Organe können nicht egoistisch sein)
-	Subsidiaritätsprinzip („Hilfe“)
•	Gemeinschaft muss alles tun was der Einzelne kann aber darf das was er selbst kann nicht machen
•	„Hilfe zur Selbsthilfe“
•	Grundaussagen:
1.	Entzugsverbot: So wenig Staat wie möglich, so viel Staat wie nötig ? Recht auf Eigentätigkeit der Einzelnen
2.	Hilfsgebot: Recht und Pflicht zur subsidiaren Assistenz (wers braucht dem wird geholfen)
3.	Subsidiäre Reduktion: Probleme auf möglichst niedrigem Organisationsniveau lösen
4.	Pflicht zur Selbsthilfe: Jeder muss Angelegenheiten so weit wie möglich selbst regeln
5.	Kompetenzverteilung: Jeder Aufgabe die er auch kann
•	Würdigung:
?	Solidarität <-> Subsidiarität (klärt wie weit 1. Zu gehen hat)
?	Begrenzt und dezentralisiert staatliche Gewalt
?	Verhindert Überforderung bestimmter Institutionen aber auch dass sie sich überall raushalten

-	Nachhaltigkeit
•	Alles auf Dauer angelegt
•	Schonender Umgang mit Ressourcen (auch menschlich)
Soziale Frage und Kirche
-	Anfänge soziale Frage: Kirchliche (Kolping) (und gebildete)
-	Lage Familie 19. Jh.
•	Geringes Einkommen
•	Beide Eltern (+Kinder) müssen arbeiten
•	Schlechter Lebensstandard
•	Keine soziale Absicherung
-	Definition Soziale Frage
•	Frage nach Fehlern und Mängeln der bestehenden gesellschaftlichen Ordnung und nach Mitteln ihnen abzuhelfen
•	Frage nach Diagnose und Therapie
•	Entweder Leiden erträglich machen (Sozialpolitik) oder Ursachen beheben (Sozialreform)
•	Kirche fordert dass neben den Betroffenen auch Staat und Kirche mit eingreift
-	Warum keine Antwort?
•	Fehlbarkeit des Menschen (Erbsünde)
•	Kein Paradies auf Erden möglich)
Ursprünge
-	Adolf Kolping (1813 – 1865)
•	Schuster dann Priester
•	Arbeite in Bedrängnis wegen industrieller Fertigung + Walz mit 16
•	Geistliche + persönliche Betreuung
•	Bezug auf Handwerker, nicht Industrie
-	Wilhelm Emanuel von Ketteler (1811 – 1877)
•	Jurist, Anfang 30 Priester
•	Predigt über große sozialen Fragen der Gegenwart
•	Bischof von Mainz
•	Fordert staatlichen Arbeiterschutz, Verot der Kinderarbeit, Sonntagsruhe
•	Fördert Arbeitervereine (kirchliche)
-	Päpstliche Sozialenzykliken
•	Rundschreiben an Bischöfe (soziales)
•	1.: 1891 – Rerum Novarum
?	Dringlichkeit Lösung Arbeiterfrage
?	Güter sollen ALLEN Menschen dienen
? fordert Erleichterung Eigentumsbildung
?	Früher Europa, heute Welt
•	2.: 1931 – Quadragesino anno
?	Gesellschaftsreform aus Sicht des Evangeliums
?	Subsidiaritätsprinzip formuliert
?	Berufsständische Gesellschaftsordnung
•	3.: 1961 – Mater et magistra
?	Wirtsch. Ausgleich zw. Den Völkern
?	Mitbeteiligung an Produktionsvermögen für Arbeitnehmer
•	4.: 1967 – Populorum progressio
?	An „alle Menschen guten Willens“
?	„Entwicklungshilfe ist Friedensarbeit“
?	Solidarität in Politik und Wirtschaft
Zukunftsprognosen
-	„Club of Rome“
•	„Grenzen des Wachstums“ (Ressourcen, Energie, …)
•	Zunahme Nord-Süd-Gefälle
•	Änderung biologische Natur des Menschen
•	Zunahme psychischer und psychosomatischer Krankheiten
•	ABC-Waffen (Atom, Bio, Chemie)
•	[…]
-	Futurologie
•	Kombination mehrerer Einzelwissenschaften
? Zukunftsplanung
Utopien
-	„Plan ohne reale Grundlage“
? entweder absolut oder aktuell nicht möglich
-	Formen
•	Soziale, technische, wirtschaftliche, politische
•	Absolute / relative
•	Positive / negative (Dystopie)
-	Dystopie
•	Greift bestehende Probleme auf und extrapoliert sie
•	Explizite Kritik (Utopie: implizite Kritik)
-	Positiv
•	Motivation zu Fortschritt
•	Zeigen Verbesserungsbedarf
•	Notwendig zur verantwortlichen Zukunftsgestaltung
•	Dystopien als Warnung
-	Negativ
•	Utopien blickwinkelabhängig Dystopie
? Gefahr Ideologisierung
•	Gefahr der Frustration / Resignation
? kann in Aggression umschlagen
Politeia (Platon)
-	Ungestrafte Gelegenheiten „müssen“ genutzt werden
? Gesetze / Kontrolle nötig
-	Drei Stände
•	Herrscher (Vernunft durch Erziehung
? Weisheit)
•	Wächter (Mut ? Tapferkeit)
•	Arbeiter (Trieb ? Besonnenheit)
•	? Gerechtigkeit
-	Freiheit der Kunst <-> sittliches Ziel
-	Vereinigung von Macht und Philosophie
-	Umwälzung von OBEN
Thomas Morus – Utopia
-	Erzählung von Inselstaat
-	Keine Gleichheit mit Privatbesitz
-	Alle arbeiten, kein Luxus
-	Regiert von gewähltem Fürst
-	Glauben an Vernunft
-	Familie als wichtiges Thema
Aldous Huxley – Brave new world
-	„Zucht“ von Kindern
-	Community, identity, stability
-	Konditionierung
-	Spontane Befriedigung der Bedürfnisse (auch freie Sexualität)
-	„Flucht“ vor Realität durch Droge Soma
-	Indianerreservat mit „normalen“ Menschen
? fordert Recht nach Schlechtem
Die Reich-Gottes-Botschaft Jesu
-	Verkündigung und Handeln Jesu
•	Verkündigung
?	Evangelium
?	Bezug auf alte Traditionen
?	Bergpredigt
?	Gleichnisse (Senfkorn, Einladung zu Hochzeit ? annehmen)
•	Handeln
?	Heilungen ? Neuanfang
?	Für ALLE da
-	Eschatologischer Vorbehalt
•	Lehre von den letzten Dingen
•	Himmel ~ Reich Gottes
•	Reich Gottes schon angebrochen aber noch nicht vollendet (schon + noch nicht)
•	Verheißung aber keine (volle) Erfüllung
•	Wachstumsgleichnisse
•	? Letzte Vollendung nur von Gott
(<-> Utopie durch Mensch)
-	Reich-Gottes-Botschaft Jesu
•	Seligpreisungen
? in R.G. ist Mensch wie er sein SOLLTE
? kehrt um
-	Altes Testament
•	Theokratie ? Jahwe als Anführer
•	König immer Gott verantwortlich
? „Kritik“ an bestehenden Herrschaftsformen
•	Entwicklung der Eschatologie v.a. durch Propheten nach Exil
•	Jeder kann kommen ? Heilsuniversalismus
Zukunft als Kommen / Werden
-	Frage nach der Zukunft
•	Normalerweise nur absehbare Teilbereiche
•	In Grenzsituationen (Liebe, Krankheit, Tod):
Frage nach „letzten“ Zukunft
-	Zukunft als Werden (futurum)
•	Vorausschaubar, planbar, machbar
•	Extrapolation vorhandener Möglichkeiten
•	Futurologie
-	Zukunft als Kommen (adventus)
•	Nicht plan-/machbar, da von freien Entscheidungen anderer abhängig
•	Mensch kann „gesamtes“ nicht in den Griff bekommen







\end{document}
