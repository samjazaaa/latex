\documentclass[11pt, paper=a4, twocolumn]{scrartcl}

\usepackage[ngerman]{babel}
\usepackage[utf8]{inputenc}

\usepackage[T1]{fontenc}

\usepackage{geometry}

\geometry{a4paper, top=20mm, left=15mm, right=15mm, bottom=20mm,
headsep=5mm, footskip=12mm}


\pagenumbering{gobble}

\title{\vspace{-1.25cm}Religion 11-1\vspace{-0.25cm}}
\date{\vspace{-5ex}}



\begin{document}
	\maketitle


	\section{Einführung}
		\begin{itemize}
			\item Datenunabhängigkeit
			\item Schichtenmodell
		\end{itemize}




Moralkonzepte

1.	Norm
-	Wortherkunft
•	Altgr. Maßstab ? Messen
•	Richtschnur ? Ausrichtung, Orientierung
-	Soziologisch
•	Erwartungscharakter (gleiche Situation: jeder gleiches Verhalten)
•	Verletzung mit Sanktionen belegt (meist inoffiziell)
-	Philosophisch
•	Regel die angibt was sein und geschehen SOLL (Richtlinie Ziel - vorher)
•	Maßstab der Beurteilung und Bewertung (nachher)
•	Aktueller: Allgemein verbindliche Verhaltens- und Handlungsregel
? Je allgemeiner / abstrakter eine Norm desto weitreichender, zeitloser und verbindlicher – Je konkreter desto eingeschränkter und weniger verbindlich
-	Allgemeine Grundnormen: verpflichtend und unwandelbar
(nicht töten, klauen etc. ~ 10 Gebote) ? „Kategorischer Imperativ“ als Grundlage
-	Bestehen <-> Anerkennen <-> Durchführung
2.	Gesetz
-	Norm deren Verletzung gerichtlich verfolgt (sanktioniert) werden kann
3.	Wert
-	ABSTRAKTES Ziel menschlichen Wünschens und Strebens
-	Hochgeschätzte abstrakte Handlungsorientierung ? nie ganz erreichbar
•	Bsp.: Freiheit, Gerechtigkeit, Liebe, Treue
4.	Gut
-	KONKRETES Ziel menschlichen Handelns
-	Hochgeschätztes sinnliches, materielles Handlungsziel ? konkret erreichbar
•	Bsp.: Fernseher, Abitur, Gesundheit, Wissen, Wohlstand
5.	Funktionen von Normen
-	Ersatzfunktion (ersetzt fehlende Instinktgebundenheit ? Orientierung)
-	Entlastungsfunktion (Schützt vor Überforderung und Entscheidungsdruck)
-	Schutzfunktion (Schützen Einzelnen vor Anforderungen / Gefahren durch andere)
-	Gelingensfunktion (Ermöglicht Einordnung des Einzelnen in Gesellschaft im Spannungsfeld zwischen Integration und Selbstverwirklichung)
6.	Beurteilung 
Sachliche Beurteilung

•	Richtig – Falsch
•	Sachurteil
•	Je nach Sachlage nach Anwendung des passenden Vorzugsprinzips
•	ERGEBNIS wird beurteilt

Moralische Beurteilung
•	Gut – Böse
•	(sittliches) Werturteil
•	Ziele und Absichten werden beurteilt
•	Objektive (an sich) <-> subjektive (für mich) Werte
•	Sittlich gut: Orientierung an objektiven Werten ( dagegen wenden)





\end{document}
