\documentclass[11pt, paper=a4, twocolumn]{scrartcl}

\usepackage[ngerman]{babel}
\usepackage[utf8]{inputenc}

\usepackage[T1]{fontenc}

\usepackage{geometry}

\geometry{a4paper, top=20mm, left=15mm, right=15mm, bottom=20mm,
headsep=5mm, footskip=12mm}


\pagenumbering{gobble}

\title{\vspace{-1.25cm}Religion 11-1\vspace{-0.25cm}}
\date{\vspace{-5ex}}



\begin{document}
	\maketitle


	\section{Einführung}
		\begin{itemize}
			\item Datenunabhängigkeit
			\item Schichtenmodell
		\end{itemize}



Religion 11/2

 
Das Gottesbild Jesu
-	Heilungserzählungen
o	Kontakt
o	Vergebung Sünden
o	Heilung
o	Wirkung
o	Begeisterung
-	Exemplarisches Handeln
-	Nähe Gottes
-	Zeigt Menschen Wege auf
-	Jesus als Vorbild
Jesus der Sohn Gottes
-	Gottes Gegenwart in Jesu
o	Heilt, bringt in Gesellschaft, verkündet Willen
o	Mehr als ein Prophet
o	Teil der Botschaft
-	„Sohn Gottes“
o	AT: Könige Söhne Gottes
o	Herrscher als Sohn Gottes
-	Anwendung auf Jesus
o	Nie selbst
o	4 Phasen
?	Paulus im Römerbrief:
seit Auferstehung
?	Matthäus & Lukas:
seit Geburt / Zeugung
?	Johannes:
Wort wird Fleisch ? Präexistenz
Evolutionärer Glaube
-	Exklusiver Glaube (nur auf wenige beschränkt)
-	Bundesglaube (Liebe Gottes an bestimmte Bedingungen geknüpft)
-	Prophetischer Glaube (Gott liebt alle aber aus der Ferne)
-	Inkarnatorischer Glaube (Gott in Leben und Beziehungen und allem)
Einflussfaktoren auf Gottesvorstellungen
-	Mutter ? Vertrauen / Misstrauen
-	Grenzsituationen
-	? erfüllt Bedürfnisse
Religionsgeschichtliche Entwicklung
-	Polytheismus
o	Mehrere Götter für verschiedene Bereiche (Natur, Lebensereignisse, Städte)
o	Erst Gegenstände direkt dann Abbilder von Göttern verehrt
o	Nomaden wenige, Ackerbau viele
-	Dualismus
o	2 gegensätzliche Kräfte
o	?<->?; geistlich <-> materiell;
gut <-> böse
o	Geistlich oft höher gestellt
-	Pantheismus
o	Gott in ALLEM ? keine Transzendenz
-	Henotheismus
o	Gibt mehrere aber nur einer wird verehrt und bekommt alle Eigenschaften der anderen
-	Monotheismus
o	Absolut: EIN Gott in EINER Person
o	Relativ: ein Gott in drei Beziehungen
Gotteserkenntnis
-	Empirie
-	Reflexion
-	Schlussfolgerung
-	? Syllogismus
Antike
-	Platon: Evidenzbeweis
o	Ideen vor den echten Dingen
o	Höchste Idee: Gott
-	Aristoteles: Bewegungsbeweis
o	Jede Bewegung braucht Ursache
o	Erster unbewegter Beweger: Gott
-	Stoa: Zweckmäßigkeitsbeweis
o	Buch nicht durch zufällige Buchstaben
o	? Ordnung der Welt braucht Urheber (Gott)
-	M. T. Cicero: Ethnologischer Beweis
o	Kein Volk ohne Götter
o	Glaube allgemein / angeboren
o	? Gott
-	Aurelius Augustinus: Gottesbeweis des A.
o	Konsens über wahres, gutes und schönes
o	Nicht von Tiere und auch nicht von Menschen (da keine kulturellen Unterschiede)
o	Höchstes gutes, wahres, schönes ? Gott
Mittelalter
-	Anselm von Canterbury: Ontologischer B.
o	Real > gedacht
o	Höchstes Gedachtes ist Gott
o	Muss existieren weil sonst größeres reales
-	Thomas von Aquin: Fünf Wege zu Gott
o	Empirie -> Axiom -> Gott
o	Bewegungsbeweis
?	Gibt Bewegung
?	Alles von was andrem bewegt
?	Erster unbewegter Beweger
o	Kausalitätsbeweis
?	Ursache & Wirkung
?	Jede Wirkung braucht Ursache
?	Erste. unverursachte Ursache
o	Kontingenzbeweis
?	Welt ist kontingent (nicht notwendig exisiterend)
?	Kann sich nicht selsbt Existenz geben
?	Notwendiges Wesen das unerschaffen ist hat Welt erschaffen
o	Stufenbeweis
?	Gibt gute, wahre und schöne Dinge
?	Gibt höchste Stufe
?	Gott
o	Finalitätsbeweis
?	Gibt Ordnung und Zweckmäßigkeit
?	Nicht von selbst sondern durch denkenden Geist
?	Gott
Beurteilung der alten Beweise
-	Gott transzendent ? kann mit menschlichen Mitteln nicht erklärt werden
-	Syllogismus: Axiome nicht beweisbar und Ausschluss des Rückgriffs ins Unendliche
-	Theoretisch ? keine Antwort auf Menschheitsfragen
Immanuel Kant
-	Begründung statt Beweis
-	Muss Gott geben damit der Mensch ein moralischer ist
? Gottespostulat
-	Unbedingt verpflichtendes Sittengesetz (kategorischer Imperativ)
-	1. Postulat (praktische Vernunft): menschliche Willensfreiheit
o	Konflikt Neigung <-> Pflicht
o	Nicht endgültig ? etwas danach
-	2. Postulat: Unsterblichkeit der Seele
o	Mensch kann sich Unsterblichkeit nicht selbst geben ? jemand der diese garantiert
-	3. Postulat: Existenz Gottes
Dogma von der natürlichen Erkennbarkeit Gottes
-	1. Vatikanisches Konzil
-	Gott KANN erkannt werden
-	Anfang und Ziel
-	Durch menschliche Vernunft erkennbar
-	Aus der Natur /dem Geschaffenen
-	SICHER erkannt werden
Religionskritik
Ludwig Feuerbach (1804-1872)
-	Gott Projektion des menschlichen Wesens und seiner Wunschträume nach Unsterblichkeit und Glück
-	Suchen wesentliches nicht in uns sondern außerhalb
-	Eigenschaften Gottes sind unsere, nur vergrößert
-	Projektion materiell nicht existent
Karl Marx (1818-1883)
-	Unterstützt Feuerbach aber fehlende Konsequenzen in gesellschaftlicher Umsetzung
-	Menschliche Autonomie ? Ablegen eines Glaubens
-	Mensch soziales Wesen
-	Religion durch Staat und Gesellschaft produziert („gelenkte Illusion“, „Jenseitsvertröstung“, „Opium des Volkes“)
-	? verhindert Analyse, klares Denken, Entwickeln von Lösungen ? zementieren Verhältnisse
-	Mensch ökonomisches und materielles Wesen ? materielle Lage > geistige
-	Mensch ist ein Gemeinschaftswesen (autonom nur in Gruppe)
-	Mensch ist ein geschichtliches Wesen (Veränderung durch vorherrschende Triebkräfte)
-	Mensch ist ein werdendes Wesen (Selbstentfremdung ? Mensch sollte sich durch Arbeit selbst erschaffen)
-	? soll sich emanzipieren und aus Missständen befreien
Kritik
-	Transzendent ? nicht beweis- oder widerlegbar
-	Zeigt dass nur eine bestimmte Vorstellung von Gott falsch ist
Friedrich Nietzsche (1844-1900)
-	Gott behindert menschliche Freiheit & Autonomie ? Menschen „töten“ Gott
-	? Leben wird ziel- und orientierungslos
nichts mit der Religion, Moral und Lebenssinn ? Nihilismus
-	Mensch muss diese Stelle selbst ausfüllen
? Übermensch
o	Raubgierige Bestie
o	Streben nach Macht und Herrschaft
o	Strebt nach Freiheit (keine moralischen Regeln)
o	Sinnlich triebhaftes Wesen (Dionysos) ? Genuss, Lust, Rücksichtslosigkeit
o	Leben als Kampf (Recht des Stärkeren)
o	Nächstenliebe als Schwäche
o	? aggressiv und destruktiv
-	Keine Unterscheidung gut – böse
-	Christliche „Sklavenmoral“
-	Übermensch tötet Gott und setzt sich an dessen Stelle und entwirft neue Moral
Vier Gründe nach Feuerbach
-	Erstaunen über eigene Kräfte ? Ergebnis mangelnder seelischer Integrationskraft
-	Begrenzte und abhängige Wesen ? Streben nach Absolutheit und Freiheit ? suchen nach überlegender Macht als Hilfe
-	Unerklärbare Dinge ? Suche nach immateriellen Ursachen da keine materiellen bekannt (halten geistige Beschränktheit nicht aus)
-	Mensch schreibt eigene Mängel der Menschheit zu ? kein Vertrauen in Gattung ? Gattung wird Gott
Das christliche Menschenbild
Philosophische Grundlagen
-	Mehrdimensionales Wesen
o	Lebendige Einheit von Leib, Geist und Seele
o	Humanwissenschaften betrachten nur EINE Seite ? einseitig
-	Personalität des Menschen
o	Verstand, Vernunft nur menschlich
o	Wille und Entscheidungsfreiheit statt Instinkte und Triebe
o	Gemüt (Emotionen, Begabungen usw.) steuerbar
-	Kulturfähigkeit
o	Sprache, Sitten und Bräuche, Kunst, Religion
-	Sozialnatur des Menschen
o	Mensch als „wir“ Wesen
o	Selbsttranszendenz (Weltoffenheit)
o	Gestaltet sich und die Welt
o	Andere spielen wichtige Rolle
Biblischer Hintergrund
-	AT: Geschöpflichkeit
o	Kann nicht selbst schaffen und ist endlich
o	Ebenbild Gottes (große Ähnlichkeit ? jeder ein bisschen Gott und dessen Vertreter ? Möglichkeit frei zu gestalten)
o	Jedoch alles nur begrenzt (u.A. Autonomie)
o	Mensch als Verantwortungsträger (Pflege der Welt) und Dialogführer Gottes
-	NT
o	Jesus als Ideal des Menschen und Umgang mit diesen
o	Vater unser ? Mensch unter Schutz Gottes
o	Erwartung der Vollendung des Lebens bei Gott






\end{document}
