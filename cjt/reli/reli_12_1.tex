\documentclass[11pt, paper=a4, twocolumn]{scrartcl}

\usepackage[ngerman]{babel}
\usepackage[utf8]{inputenc}

\usepackage[T1]{fontenc}

\usepackage{geometry}

\geometry{a4paper, top=20mm, left=15mm, right=15mm, bottom=20mm,
headsep=5mm, footskip=12mm}


\pagenumbering{gobble}

\title{\vspace{-1.25cm}Religion 11-1\vspace{-0.25cm}}
\date{\vspace{-5ex}}



\begin{document}
	\maketitle


	\section{Einführung}
		\begin{itemize}
			\item Datenunabhängigkeit
			\item Schichtenmodell
		\end{itemize}



Religion 12/1
 
Moralkonzepte
-	Siehe extra Blatt
Moralische Entscheidung
-	1. Ziel bewerten
•	Gute <-> schlechte Ziele
-	2. Mittel / Wege überlegen und auswählen
•	Zweck heiligt nicht die Mittel
•	Adäquater Informationsstand nötig um den ethisch richtigen Weg zu wählen (Nebeneffekte)
-	3. Handlung umsetzen
Herkunft von Normen(philosophisch)
-	Biologische Naturordnung als Normenquelle
•	Gut: alles was der Ordnung der sichtbaren Natur entspricht
•	Problem: Schwächere unterliegen (Sozialdarwinismus)
-	Natürliches Sittengesetz
•	Nach Verstand / Vernunft die Natur analysieren
•	? auch Schwächere eine Chance
•	Aristoteles, Thomas v. Aqu.
-	Nützlichkeit und Annehmlichkeit – Utilitarismus
•	Gut: alles was größtmöglicher Zahl von Menschen größtmöglichen Nutzen bringt (Erfolg, Glück, …)
•	Minderheit vernachlässigt
•	Definition von Nutzen und größtmöglicher Anzahl problematisch
-	Hedonismus
•	Gut ist was Spaß macht / Lust schafft
•	Probleme für sich selbst und andere
-	Unbedingtheit der Pflicht als Normenquelle
•	Kategorischer Imperativ / „Goldene Regel“
•	Pflichten gegenüber Gesellschaft; muss Handlungen auch von anderen erwarten
Herkunft von Normen (Grundwerte)
-	„angeboren“ <-> diskutierbar
-	Nach menschlicher Persönlichkeit
-	Normen nach Verhalten der Mehrheit (Normative Kraft des Faktischen)
Problem: viele machen was „falsches“ ? richtig?
-	Normen beeinflussen Verhalten
-	Gesetzgeber als Normenquelle: Rechtspositivismus
Problem: Gesetzgeber nicht immer moralisch richtig

Herkunft von Normen (theologisch)
-	Illuminationsmodell
•	Mensch durch Gott erleuchtet ? erkennt intuitiv Normen
•	Unkontrollierbar und zu individualistisch
•	Eher als „Gewissensentscheidung“
-	Biblizistisches Modell
•	Bibel als direkte Normenquelle (unfehlbar und verpflichtend)
•	Problem: Weiterentwickelte Gesellschaft; nicht wörtlich zu interpretieren
•	Liefert nur grundlegende Entscheidungsfundamente
-	Traditionalistisches Modell
•	Kirchliche Traditionen als Normenquelle
•	Problem: Hintergründe und Aktualität der Traditionen
-	Lehramtsmodell
•	Kirchliches Lehramt als Normenquelle
•	Papst ist unfehlbar wenn er es beansprucht und in Übereinstimmung mit der Kirche redet (Mariä Himmelfahrt)
Konstanz und Variabilität von Normen
-	Einzige Konstanz: Wille Gottes (lex aeterna)
-	Natürliches Sittengesetz (aus Natur mit Verstand ? Wille Gottes)
-	Hierarchie





-	Das positive göttliche Gesetz
•	Von Gott gegebene Gesetze (Moral-, Zeremonial- und Iudizialgesetze)
•	Im neuen Testament verschärft durch Jesus
-	Positives menschliches Recht
•	Gesetzgeber: Staat; auch Kirchenrecht (innerhalb der Kirche)
-	Staatliches Recht
•	Muss sittlichen Ansprüchen genügen aber NICHT Moral
? darf nur sittlich erlaubtes fordern
•	Muss gerecht sein
(z.B. Gleichheitsgrundsatz)
•	Muss notwendig oder nützlich sein
? keine unnötigen Gesetze
•	Muss einhaltbar / erfüllbar sein
? Machbarkeit
-	Gültigkeitsdauer
•	Bezug zur Zeit (Kranzgeld)
•	Offizielle Ausnahme ? Dispens
•	Epikie (Erkenntnis dass ein Gesetz für diese Situation nicht gilt)
Ethische Modelle der Entscheidung
-	Werte-Ethik
•	Werte und Güter als eins
•	Mensch erkennt Werte aus Wirklichkeit
•	Ziel: Werte schützen, fördern und umsetzen
•	Rangordnung von Werten:
1.	Leben und lebensnotwendige Dinge (Nahrung, Kleidung, …)
2.	Werte des sinnlichen Fühlens oder des Angenehmen (ästhetische Bedürfnisse)
3.	Werte des edlen (Mut, Tapferkeit, …)
4.	Werte des geistigen (schönes, wahres und gutes) ? Bildung, Verstand, Kunst
5.	Liebe
-	Situationsethik
•	Das Sein hat in sich keinerlei Werte
? sinnlos (Sartre – Existenzialismus)
•	? muss eigene Ethik je nach Anforderung und Situation immer neu entwerfen ohne Norm als Entscheidungshilfe
? willkürgefahr
-	Gesinnungsethik (deontologische)
•	Handlungen die IMMER schlecht sind
? immer ausnahmslos zu meiden
•	Eine Gesinnung der man direkt folgt
? einfache Entscheidungen aber häufig Ausnahmen nötig
-	Verantwortungsethik (teleologische)
•	Zweck / Ziel der Handlung entscheidend
•	Verantwortungsethik beurteilt auch Weg
•	Jede Handlung muss ausschließlich von ihren Folgen her sittlich beurteilt werden wobei auch die Folgen des Weges mit zu bedenken sind
•	Bewusstsein für alle Folgen des Handelns und Verantwortung für diese
? auch für Informationsstand verantwortlich ? Informationspflicht
•	Bei unübersehbaren Folgen ist die Handlung zu unterlassen
•	Evtl. Inkaufnahme eines Übels zum Schutz etwas wichtigeren
•	Güterabwägung: Konkurrenz verschiedener Güter und Werte, Dringlichkeit, Eigenwohl, gemein-, Gesamtwohl
Dekalog
-	1. Tafel (Gott)
•	1. Henotheismus, Monolatrie
•	(2. Kein Gottesbild; Christentum Ausnahme, da Jesus sichtbar geworden als Mensch)
•	2. Name Gottes nicht missbrauchen
•	3. Sabbat ehren ? funktionierende Gesellschaft (bei Juden 2. Tafel)
-	2. Tafel (Mensch)
•	4. Eltern ehren
•	5. Nicht morden
•	6. Nicht Ehe brechen
•	7. Nicht Stehlen
•	8. Nicht falsch gegen andere Aussagen
•	9. Nicht nach Frau des Anderen verlangen
•	10. Nicht nach Haus des Anderen verlangen
-	(Be-)Deutung
•	Zweite Tafel eher Zusammenfassung von bereits vorhandenen rechtlichen und sittlichen Traditionen
•	„Gebrauchsanleitung“ für Verhalten in der Welt
•	Wegweiser in die Freiheit ? keine Einengung sondern sichern der erlangten Freiheit (Präambel)
•	Eher Richtlinien, die immer neu und angepasst zu deuten sind (z.B. Töten)
Die Bergpredigt
-	Mt.: Erste und zentrale Rede Jesu (drei Kapitel)
-	Wahrscheinlich erst im Nachhinein zusammengefügt
-	„Vater Unser“ als Zentrum
-	Bei Lukas kürzer – Feldrede
-	Bergsymbolik ? Nähe zu Gott (vgl. Moses, Eliah)
-	Interpretation des Dekalogs
•	6.: Allein bei Gedanke Auge ausreißen
? Provokation ? Diskussionen
•	Mord: auch schon Planung falsch
•	Generell: Verschärfung
-	Jesus stellt sich auf Ebene Mose und sogar Gott („Ich aber sage euch …“)
-	Keine Aufhebung der alten Gesetze und Traditionen
-	Unerfüllbar?
•	(nur für Jüngerkreis Jesu)
•	Interimsethik bis zur Endzeit
•	Utopisten wortwörtlich
•	Unerfüllbar ? Angewiesenheit auf Gottes Erbarmen verdeutlicht
-	„Bedingungen für Reich Gottes“
Naturrecht vs. Rechtspositivismus
-	Naturrecht
•	Übergeordnetes, immer geltendes Naturrecht
•	Positivierten Normen vorgelagert
•	RechtsINHALT absolut
•	Richter kann vernünftigem Ermessen Vorrang vor geschriebenem Gesetz geben
•	Gefahr: Rechtsunsicherheit, Willkür
-	Rechtspositivismus
•	Positivierte Normen
•	RechtsFORM absolut
•	Bedingung: formell korrekt erzeugt
•	Richter streng an die Gesetze gebunden
•	Gefahr: „ungerechte“ Gesetze
-	Problematiken
•	Naturrecht müsste endgültig formuliert werden aber jede Formulierung ist zeitbedingt
•	Rechtspositivismus kann unsittliche Ordnungen errichten
•	Mensch kann vollkommene Ordnung des Naturrechts weder vollständig erfassen noch formulieren
•	Mensch kann nur Widerspruch von Naturrecht erkennen
•	Naturrecht nur als Maßstab für positives Recht
Menschenrechte
-	Entstehung
•	Naturrecht in der Antike: alle unter gleichen Bedingungen leben
•	Christliches Naturrecht: alle gleich weil alle Ebenbilder Gottes und Sünder ? jedoch nicht rechtliche oder soziale Gleichheit abgeleitet sondern nur  innere Freiheit
•	Ständische Freiheitstradition MA:
Leben, Freiheit und Eigentum nur durch Gerichtsurteil ? Widerstandsrecht gegen Herrscher die das Recht brachen
•	Rationalistisches Naturrecht in Aufklärung:
?	Säkularisierung des Naturrechts
?	John Milton: Recht auf Selbstbestimmtheit
?	John Locke: Mensch erkennt Rechtsgrundsätze durch Vernunft; Vorstaatliche Rechte vorausgesetzt
? Leben, Freiheit, Eigentum
-	Verwirklichung
•	England: Revolution
•	Nordamerika: Bill of Rights, Independence
•	Deutschland: Verfassung
-	Menschenrechte
•	1. Würde Mensch
•	2. Gleichstellung
•	3. Lebensrecht (frei & sicher)
•	4. Anti-Sklaverei
•	5. Anti-Folter
•	6. Rechtsfähigkeit
•	7. Schutz vor Diskriminierung
•	8. Rechtssicherheit
•	9. Willkürverbot
•	10. Neutrale Rechtsprechung
•	11. Unschuldsvermutung
•	12. Recht auf Privatsphäre
•	13. Freizügigkeit
•	14. Asylrecht
•	15. Recht auf Staatsangehörigkeit
•	16. Recht Ehe & Familie
•	17. Recht Eigentum
•	18. Gedanken- / Religionsfreiheit
•	19. Meinungs- / Informationsfreiheit
•	20. Vereinigungsfreiheit
•	21. Demokratie
•	22. Soziale Sicherheit
•	23. Recht auf Arbeit
•	24. Recht auf Freizeit
•	25. Lebensstandard
•	26. Bildung
•	27. Kultur & Kunst
•	28. Internationale Ordnung
•	29. Pflichten
•	30. Abschaffungsverbot
-	Parallelen
•	Dekalog: viele (1. Und 2. Indirekt)
•	Grundgesetz: mehr Einschränkungen und Zusammenfassung; andere Schwerpunkte (Sklaven und Folter nicht enthalten)
Katholische Soziallehre
-	Caritas
•	Akute Notfallhilfe
•	Kurzfristig
•	Einzelfälle
•	Symptombehandlung
-	Soziallehre
•	Änderung Verhältnisse
•	Langfristig, dauerhaft
•	? Reformen, Wurzelbehandlung






\end{document}
